%% Generated by Sphinx.
\def\sphinxdocclass{report}
\documentclass[letterpaper,10pt,english]{sphinxmanual}
\ifdefined\pdfpxdimen
   \let\sphinxpxdimen\pdfpxdimen\else\newdimen\sphinxpxdimen
\fi \sphinxpxdimen=.75bp\relax
\ifdefined\pdfimageresolution
    \pdfimageresolution= \numexpr \dimexpr1in\relax/\sphinxpxdimen\relax
\fi
%% let collapsible pdf bookmarks panel have high depth per default
\PassOptionsToPackage{bookmarksdepth=5}{hyperref}

\PassOptionsToPackage{booktabs}{sphinx}
\PassOptionsToPackage{colorrows}{sphinx}

\PassOptionsToPackage{warn}{textcomp}
\usepackage[utf8]{inputenc}
\ifdefined\DeclareUnicodeCharacter
% support both utf8 and utf8x syntaxes
  \ifdefined\DeclareUnicodeCharacterAsOptional
    \def\sphinxDUC#1{\DeclareUnicodeCharacter{"#1}}
  \else
    \let\sphinxDUC\DeclareUnicodeCharacter
  \fi
  \sphinxDUC{00A0}{\nobreakspace}
  \sphinxDUC{2500}{\sphinxunichar{2500}}
  \sphinxDUC{2502}{\sphinxunichar{2502}}
  \sphinxDUC{2514}{\sphinxunichar{2514}}
  \sphinxDUC{251C}{\sphinxunichar{251C}}
  \sphinxDUC{2572}{\textbackslash}
\fi
\usepackage{cmap}
\usepackage[T1]{fontenc}
\usepackage{amsmath,amssymb,amstext}
\usepackage{babel}



\usepackage{tgtermes}
\usepackage{tgheros}
\renewcommand{\ttdefault}{txtt}



\usepackage[Bjarne]{fncychap}
\usepackage{sphinx}

\fvset{fontsize=auto}
\usepackage{geometry}


% Include hyperref last.
\usepackage{hyperref}
% Fix anchor placement for figures with captions.
\usepackage{hypcap}% it must be loaded after hyperref.
% Set up styles of URL: it should be placed after hyperref.
\urlstyle{same}

\addto\captionsenglish{\renewcommand{\contentsname}{Table of Contents:}}

\usepackage{sphinxmessages}
\setcounter{tocdepth}{1}



\title{Canadian Payroll Administration}
\date{Jun 22, 2025}
\release{HR}
\author{Alexandre Bobkov}
\newcommand{\sphinxlogo}{\vbox{}}
\renewcommand{\releasename}{Release}
\makeindex
\begin{document}

\ifdefined\shorthandoff
  \ifnum\catcode`\=\string=\active\shorthandoff{=}\fi
  \ifnum\catcode`\"=\active\shorthandoff{"}\fi
\fi

\pagestyle{empty}
\sphinxmaketitle
\pagestyle{plain}
\sphinxtableofcontents
\pagestyle{normal}
\phantomsection\label{\detokenize{index::doc}}


\sphinxstepscope


\chapter{INTRODUCTION}
\label{\detokenize{introduction:introduction}}\label{\detokenize{introduction::doc}}

\section{Payroll Legal Framework}
\label{\detokenize{introduction:payroll-legal-framework}}
\sphinxAtStartPar
The Canadian Payroll Administration system is designed to ensure compliance with the legal framework governing payroll in Canada. This includes adherence to federal and provincial regulations regarding employee compensation, deductions, and reporting requirements.
The system is built to handle various payroll scenarios, including different employment types, tax calculations, and benefit deductions, while ensuring that all transactions are accurately recorded and reported in accordance with the law.

\sphinxstepscope


\chapter{Payroll Accounting}
\label{\detokenize{payroll_accounting:payroll-accounting}}\label{\detokenize{payroll_accounting::doc}}

\section{Journal Entries}
\label{\detokenize{payroll_accounting:journal-entries}}

\subsection{Accounting Recap}
\label{\detokenize{payroll_accounting:accounting-recap}}
\sphinxAtStartPar
\(\Sigma \text{ Total Debits} = \Sigma \text{ Total Credits}\)

\sphinxAtStartPar
\({Assets} = {Liabilities} + Equity\)

\sphinxAtStartPar
Furthermore, we know that:
\begin{equation*}
\begin{split}Equity = Revenue - Expenses\end{split}
\end{equation*}
\sphinxAtStartPar
, which leads us to:
\begin{equation*}
\begin{split}Assets = Liabilities + (Revenues - Expenses)\end{split}
\end{equation*}
\sphinxAtStartPar
Payroll accounting is a critical component of the Canadian Payroll Administration system. It involves the systematic recording, analysis, and reporting of payroll transactions to ensure that all financial aspects of employee compensation are accurately reflected in the organization’s financial statements.
Payroll accounting includes the management of employee wages, tax withholdings, benefit deductions, and other payroll\sphinxhyphen{}related expenses. The system is designed to automate these processes, ensuring accuracy and compliance with Canadian payroll regulations.


\subsection{Journal Entries}
\label{\detokenize{payroll_accounting:id1}}
\sphinxAtStartPar
Journal entries are a key part of payroll accounting, as they document the financial impact of payroll transactions on the organization’s accounts. Each payroll run generates a series of journal entries that reflect the distribution of wages, taxes, and deductions across various accounts.
These entries are essential for maintaining accurate financial records and ensuring that the organization’s financial statements reflect the true cost of employee compensation. The Canadian Payroll Administration system automates the generation of these journal entries, reducing the risk of errors and ensuring compliance with accounting standards.

\begin{DUlineblock}{0em}
\item[] DR    Payroll Expenses    \$10,500.00
\item[]
\begin{DUlineblock}{\DUlineblockindent}
\item[] CR  Payroll Payable   \$10,500.00
\end{DUlineblock}
\end{DUlineblock}

\sphinxstepscope


\chapter{REVIEW QUESTIONS}
\label{\detokenize{review_questions:review-questions}}\label{\detokenize{review_questions::doc}}
\sphinxAtStartPar
This section contains review questions for the material covered in the course. These questions are designed to test your understanding and help reinforce the concepts learned.


\section{New Employee Information}
\label{\detokenize{review_questions:new-employee-information}}
\sphinxstepscope


\chapter{REFERENCES}
\label{\detokenize{references:references}}\label{\detokenize{references::doc}}
\sphinxstepscope


\chapter{Errors and Errata}
\label{\detokenize{errata:errors-and-errata}}\label{\detokenize{errata::doc}}

\chapter{Glossary}
\label{\detokenize{index:glossary}}\begin{itemize}
\item {} 
\sphinxAtStartPar
\DUrole{xref}{\DUrole{std}{\DUrole{std-ref}{genindex}}}

\end{itemize}



\renewcommand{\indexname}{Index}
\printindex
\end{document}