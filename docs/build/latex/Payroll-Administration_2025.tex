%% Generated by Sphinx.
\def\sphinxdocclass{report}
\documentclass[letterpaper,10pt,english]{sphinxmanual}
\ifdefined\pdfpxdimen
   \let\sphinxpxdimen\pdfpxdimen\else\newdimen\sphinxpxdimen
\fi \sphinxpxdimen=.75bp\relax
\ifdefined\pdfimageresolution
    \pdfimageresolution= \numexpr \dimexpr1in\relax/\sphinxpxdimen\relax
\fi
%% let collapsible pdf bookmarks panel have high depth per default
\PassOptionsToPackage{bookmarksdepth=5}{hyperref}

\PassOptionsToPackage{booktabs}{sphinx}
\PassOptionsToPackage{colorrows}{sphinx}

\PassOptionsToPackage{warn}{textcomp}
\usepackage[utf8]{inputenc}
\ifdefined\DeclareUnicodeCharacter
% support both utf8 and utf8x syntaxes
  \ifdefined\DeclareUnicodeCharacterAsOptional
    \def\sphinxDUC#1{\DeclareUnicodeCharacter{"#1}}
  \else
    \let\sphinxDUC\DeclareUnicodeCharacter
  \fi
  \sphinxDUC{00A0}{\nobreakspace}
  \sphinxDUC{2500}{\sphinxunichar{2500}}
  \sphinxDUC{2502}{\sphinxunichar{2502}}
  \sphinxDUC{2514}{\sphinxunichar{2514}}
  \sphinxDUC{251C}{\sphinxunichar{251C}}
  \sphinxDUC{2572}{\textbackslash}
\fi
\usepackage{cmap}
\usepackage[T1]{fontenc}
\usepackage{amsmath,amssymb,amstext}
\usepackage{babel}



\usepackage{tgtermes}
\usepackage{tgheros}
\renewcommand{\ttdefault}{txtt}



\usepackage[Bjarne]{fncychap}
\usepackage{sphinx}

\fvset{fontsize=auto}
\usepackage{geometry}


% Include hyperref last.
\usepackage{hyperref}
% Fix anchor placement for figures with captions.
\usepackage{hypcap}% it must be loaded after hyperref.
% Set up styles of URL: it should be placed after hyperref.
\urlstyle{same}

\addto\captionsenglish{\renewcommand{\contentsname}{Table of Contents:}}

\usepackage{sphinxmessages}
\setcounter{tocdepth}{2}



\title{Canadian Payroll Administration (2025)}
\date{Jul 18, 2025}
\release{Fall 2025}
\author{Alexandre Bobkov}
\newcommand{\sphinxlogo}{\vbox{}}
\renewcommand{\releasename}{Release}
\makeindex
\begin{document}

\ifdefined\shorthandoff
  \ifnum\catcode`\=\string=\active\shorthandoff{=}\fi
  \ifnum\catcode`\"=\active\shorthandoff{"}\fi
\fi

\pagestyle{empty}
\sphinxmaketitle
\pagestyle{plain}
\sphinxtableofcontents
\pagestyle{normal}
\phantomsection\label{\detokenize{index::doc}}


\sphinxAtStartPar
Alexander Bobkov (Alex) is the author of this comprehensive and practical study guide for Payroll Administration,
drawing on nearly two decades of hands\sphinxhyphen{}on experience in the accounting field. From 2005 to 2022, Alexander successfully
operated his own accounting firm, offering bookkeeping, accounting, and payroll services to a diverse clientele in the National Capital Regions.
With a rich educational background that spans from a college diploma to a Master’s degree in Business, he brings both academic insight
and practical expertise to his work. For the past five years, Alexander has focused specifically on the payroll sector. This study guide reflects
his long\sphinxhyphen{}standing goal: to help professional bookkeepers and business managers to build a solid foundation in payroll administration while easing
the anxiety often associated with its complexity. Designed to be clear, practical, and empowering, the guide equips readers with the skills needed
to confidently perform essential payroll functions encountered in day\sphinxhyphen{}to\sphinxhyphen{}day operations.

\sphinxstepscope


\chapter{PREFACE}
\label{\detokenize{preface:preface}}\label{\detokenize{preface::doc}}
\sphinxAtStartPar
Through this material, students will gain a comprehensive understanding of core payroll principles and practices. They will explore legislative compliance requirements and the role of key regulatory bodies that govern payroll operations in Canada.

\sphinxAtStartPar
Students will learn how to:
\begin{itemize}
\item {} 
\sphinxAtStartPar
Accurately calculate net pay for salaried, hourly, commissioned, and contract employees.

\item {} 
\sphinxAtStartPar
Identify and meet payroll\sphinxhyphen{}related obligations for businesses.

\item {} 
\sphinxAtStartPar
Navigate the administrative aspects of human resource management that intersect with payroll responsibilities.

\item {} 
\sphinxAtStartPar
Apply payroll procedures using computerized payroll software through practical, hands\sphinxhyphen{}on exercises.

\item {} 
\sphinxAtStartPar
Payroll’s responsibilities from hiring through to termination.

\item {} 
\sphinxAtStartPar
Payroll compliance legislation in practical scenarios.

\item {} 
\sphinxAtStartPar
Individual pay calculation process.

\end{itemize}


\section{Learning Outcomes}
\label{\detokenize{preface:learning-outcomes}}
\sphinxAtStartPar
The material of this study guide aim to make students to be be able to:
\begin{itemize}
\item {} 
\sphinxAtStartPar
Calculate regular individual pay

\item {} 
\sphinxAtStartPar
Calculate non\sphinxhyphen{}regular individual pay

\item {} 
\sphinxAtStartPar
Calculate termination payments

\item {} 
\sphinxAtStartPar
Complete a Record of Employment (ROE)

\item {} 
\sphinxAtStartPar
Apply federal and provincial legislation to payroll, including:
\sphinxhyphen{} The Canada Pension Plan Act
\sphinxhyphen{} The Employment Insurance Act
\sphinxhyphen{} The Income Tax Act
\sphinxhyphen{} Employment Standards legislation
\sphinxhyphen{} Workers’ Compensation Acts
\sphinxhyphen{} Québec\sphinxhyphen{}specific legislation

\end{itemize}


\section{Recommended Course Material}
\label{\detokenize{preface:recommended-course-material}}

\section{Material Structure Overview}
\label{\detokenize{preface:material-structure-overview}}\begin{enumerate}
\sphinxsetlistlabels{\arabic}{enumi}{enumii}{}{.}%
\item {} 
\sphinxAtStartPar
Introduction to Canadian Payroll

\item {} 
\sphinxAtStartPar
Labour and Employment Standards

\item {} 
\sphinxAtStartPar
Accounting for Payroll

\item {} 
\sphinxAtStartPar
Calculating Gross Pay

\item {} 
\sphinxAtStartPar
Pensionable, Insurable, and Taxable Earnings

\item {} 
\sphinxAtStartPar
Calculating Net Pay

\item {} 
\sphinxAtStartPar
Calculating Employer’s Source Deduction Remittances

\item {} 
\sphinxAtStartPar
Termination of Employment:
\begin{itemize}
\item {} 
\sphinxAtStartPar
Record of Employment (ROE)

\item {} 
\sphinxAtStartPar
Termination Payments

\item {} 
\sphinxAtStartPar
Retirement Pay

\end{itemize}

\end{enumerate}

\sphinxAtStartPar
In other words, the material covers the foundational knowledge and technical skills needed to confidently perform payroll tasks in a variety of employment settings.

\sphinxstepscope


\chapter{INTRODUCTION}
\label{\detokenize{introduction:introduction}}\label{\detokenize{introduction::doc}}

\section{Outcomes}
\label{\detokenize{introduction:outcomes}}
\sphinxAtStartPar
Applying federal and provincial payroll legislation, regulations, and policies to ensure compliance with the legal framework governing payroll in Canada.
\begin{itemize}
\item {} 
\sphinxAtStartPar
CPP/QPP

\item {} 
\sphinxAtStartPar
EI

\item {} 
\sphinxAtStartPar
Income Tax (Federal, ON and QC)

\end{itemize}

\sphinxAtStartPar
Calculating regular individual pay

\sphinxAtStartPar
Calculating non\sphinxhyphen{}regular individual pay

\sphinxAtStartPar
Calculating termination pay

\sphinxAtStartPar
Completing a Record of Employment (ROE)


\subsection{Payroll Legal Framework}
\label{\detokenize{introduction:payroll-legal-framework}}
\sphinxAtStartPar
The Canadian Payroll Administration system is designed to ensure compliance with the legal framework governing payroll in Canada. This includes adherence to federal and provincial regulations regarding employee compensation, deductions, and reporting requirements.
The system is built to handle various payroll scenarios, including different employment types, tax calculations, and benefit deductions, while ensuring that all transactions are accurately recorded and reported in accordance with the law.

\sphinxstepscope


\chapter{TERMINOLOGY}
\label{\detokenize{terminology:terminology}}\label{\detokenize{terminology::doc}}

\section{Pensionable Earnings}
\label{\detokenize{terminology:pensionable-earnings}}

\section{Insurable Earnings}
\label{\detokenize{terminology:insurable-earnings}}
\sphinxstepscope


\chapter{PAYROLL COMPLIANCE AND REGULATIONS}
\label{\detokenize{compliance:payroll-compliance-and-regulations}}\label{\detokenize{compliance::doc}}
\sphinxAtStartPar
\sphinxstylestrong{LEARNING OBJECTIVES}

\sphinxAtStartPar
This chapter provides a comprehensive introduction to the fundamentals of payroll compliance and regulations in Canada.
It outlines the key stakeholders involved, the core objectives of payroll, and the legal frameworks that shape payroll
operations. The differences between federal and provincial/territorial jurisdictions are clearly explained, with emphasis on
how each level of government influences payroll administration. The chapter also examines the Canada Revenue Agency’s
criteria for determining whether an individual is considered an employee or self\sphinxhyphen{}employed, providing essential context for
accurate classification and compliance.

\sphinxAtStartPar
Topics covered in this chapter are:
\begin{enumerate}
\sphinxsetlistlabels{\arabic}{enumi}{enumii}{}{.}%
\item {} 
\sphinxAtStartPar
Identify four uses of the term payroll

\item {} 
\sphinxAtStartPar
Describe payroll’s objectives

\item {} 
\sphinxAtStartPar
Describe who payroll’s stakeholders are

\item {} 
\sphinxAtStartPar
Differentiate between federal and provincial/territorial jurisdictions

\item {} 
\sphinxAtStartPar
Explain how each stakeholder affects payroll processes and procedures

\item {} 
\sphinxAtStartPar
Apply the Canada Revenue Agency’s factors for determining whether an individual is an employee or self\sphinxhyphen{}employed

\end{enumerate}


\section{Introduction}
\label{\detokenize{compliance:introduction}}
\sphinxAtStartPar
Payroll is an essential operational function in any organization that employs staff, ensuring individuals are compensated
appropriately for their work. The payroll process is subject to a robust framework of legislation. Both federal and
provincial/territorial governments enact regulations that oversee various elements of payroll administration,
including compensation practices, taxation of employee benefits, and the protection of worker rights. These legal obligations
help establish consistency, accountability, and fairness in how employee remuneration is managed across sectors.

\sphinxAtStartPar
For the purposes of this course, payroll refers specifically to the process of compensating employees for work performed
within the framework of an employer\sphinxhyphen{}employee relationship. Individuals who are self\sphinxhyphen{}employed or work as contractors submit
invoices for their services and are paid through accounts payable, not through the payroll system — and therefore are not
considered employees. This chapter explains how to assess whether a true employee\sphinxhyphen{}employer relationship exists. Once that
relationship is confirmed, the appropriate method of payment can be accurately identified and applied.

\sphinxAtStartPar
Various levels of the governments offer specific criteria that help determine whether an employee\sphinxhyphen{}employer
relationship is in place. Understanding the nature of this relationship is essential when evaluating a worker’s status within
an organization. Once the relationship is clearly identified, it becomes possible to ensure that payments made to the
individual are handled in full compliance with applicable legislation. This distinction plays a vital role in payroll
administration and helps prevent legal or financial errors tied to misclassification.


\section{Payroll Objectives}
\label{\detokenize{compliance:payroll-objectives}}
\sphinxAtStartPar
The payroll function plays a vital role in every organization, with its primary objective being to ensure employees are
compensated accurately and on time, in accordance with legislative requirements throughout the full annual payroll cycle.

\sphinxAtStartPar
Employees expect to receive their pay as scheduled and in the method agreed upon with their employer, whether by cheque or
direct deposit. Beyond ensuring timely payment, payroll practitioners are also responsible for effectively communicating
payroll\sphinxhyphen{}related information to all relevant stakeholders, supporting transparency, compliance, and organizational
accountability.

\sphinxAtStartPar
\sphinxstylestrong{Payroll} is the process of paying employees in exchange for the work they perform. The
term payroll can refer to:
\begin{itemize}
\item {} 
\sphinxAtStartPar
the department that administers the payroll

\item {} 
\sphinxAtStartPar
the total number of people employed by an organization

\item {} 
\sphinxAtStartPar
the wages and salaries paid out in a year

\item {} 
\sphinxAtStartPar
a list of employees to be paid and the amount due to each

\end{itemize}

\sphinxAtStartPar
\sphinxstylestrong{Legislation} refers to laws enacted by a legislative body. In Canada there are many legislative
sources that payroll practitioners must comply with at two separate levels ─ the federal and
the provincial/territorial governments. Later in the chapter we will explore the compliance
requirements for the various pieces of legislation from these sources.

\sphinxAtStartPar
\sphinxstylestrong{Compliance} is the observance of official requirements. For payroll, this means
performing payroll functions according to federal and provincial/territorial legislative and
non\sphinxhyphen{}governmental stakeholder requirements.

\sphinxAtStartPar
The legislative requirements are termed \sphinxstylestrong{statutory}. This means they are enacted, created, or
regulated by statute, a law enacted by the legislative branch of a government. Fines and
penalties can be imposed if an organization is not in compliance with the legislative
requirements in each jurisdiction.

\sphinxAtStartPar
When working with federal and provincial or territorial government agencies, payroll professionals must be well\sphinxhyphen{}versed in
the various laws and regulations that govern payroll operations, as well as the compliance requirements specific to each.
It is their responsibility to ensure the organization adheres to all applicable legislation, thereby minimizing the risk of
fines or legal penalties.

\sphinxAtStartPar
In addition to government regulations, payroll practitioners must also comply with obligations set forth by non\sphinxhyphen{}government
stakeholders. These may include collective agreements with unions, group insurance policies, pension plans, and other
contractual arrangements. Maintaining compliance across all stakeholder requirements is essential to the integrity and
effectiveness of the payroll function.


\section{Responsibilities and Functions of Payroll}
\label{\detokenize{compliance:responsibilities-and-functions-of-payroll}}
\sphinxAtStartPar
The responsibilities of a payroll practitioner can vary significantly based on the size and structure of the organization,
the jurisdictions in which employees are paid, and the presence of other supporting departments such as human resources,
finance, and administration. These factors influence both the scope and complexity of payroll duties within the organization.

\sphinxAtStartPar
In small and medium\sphinxhyphen{}sized organizations, payroll practitioners often take on multiple roles that would typically be divided
among separate departments in larger companies. Their responsibilities may extend beyond payroll to include tasks such as
employee recruitment, human resources policy development, benefits administration, accounts payable and receivable, budgeting,
and general administrative functions. In these settings, a broad and thorough knowledge of all assigned areas is essential,
along with an understanding of the resources available to seek advice or guidance when needed.

\sphinxAtStartPar
In contrast, larger organizations tend to maintain dedicated payroll departments, supported by separate teams for human
resources, accounting, and administration. Even within these multi\sphinxhyphen{}departmental structures, payroll professionals must possess
a strong understanding of the employee life cycle. From onboarding through to termination, each stage carries specific
implications for pay processing and reporting, requiring close coordination and specialized expertise.

\sphinxAtStartPar
The payroll department in a large organization may have:
\begin{itemize}
\item {} 
\sphinxAtStartPar
payroll administrators who are responsible for entering payroll data into the system and making required payroll remittances

\item {} 
\sphinxAtStartPar
payroll coordinators who are responsible for preparing the payroll journal entries and reconciling the payroll related accounts

\item {} 
\sphinxAtStartPar
payroll managers who manage the payroll function, the payroll staff and represent payroll at the management level

\end{itemize}

\sphinxAtStartPar
\sphinxstylestrong{Content Knowledge}

\sphinxAtStartPar
Payroll normally requires performing the following duties:
\begin{itemize}
\item {} 
\sphinxAtStartPar
Payroll Compliance Legislation: the Income Tax Act, the Employment Insurance Act, the Canada Pension Plan Act, Employment/Labour Standards, privacy legislation, Workers’ Compensation and provincial/territorial payroll\sphinxhyphen{}specific legislation

\item {} 
\sphinxAtStartPar
Payroll Processes: the remuneration and deduction components of payroll and how to use these components to calculate a net pay in both regular and non\sphinxhyphen{}regular circumstances

\item {} 
\sphinxAtStartPar
Payroll Reporting: how to calculate and remit amounts due to government agencies, insurance companies, unions and other third parties. In addition, payroll reporting includes accounting for payroll expenses and accruals to internal financial systems and federal and provincial/territorial year\sphinxhyphen{}end reporting.

\end{itemize}

\sphinxAtStartPar
\sphinxstylestrong{Technical Skills}

\sphinxAtStartPar
Payroll professionals must possess a strong set of technical skills to perform their roles effectively. These include
proficiency in payroll software and financial systems, as well as competence in commonly used computer applications such as
spreadsheets, databases, and word processing programs. Mastery of these tools ensures accurate processing, reporting, and
management of payroll\sphinxhyphen{}related data.

\sphinxAtStartPar
As organizations evolve and adapt to new technologies and reporting requirements, payroll and business systems are frequently
updated or replaced. Therefore, it is essential for payroll personnel to remain flexible and open to change. A successful
payroll practitioner should demonstrate a willingness to embrace continuous learning and stay current with system upgrades and
best practices. This adaptability not only enhances performance but also supports long\sphinxhyphen{}term career growth in an ever\sphinxhyphen{}changing
professional landscape.

\sphinxAtStartPar
\sphinxstylestrong{Personal and Professional Skills}

\sphinxAtStartPar
The following personal and professional skills will assist payroll professionals in dealing with
the various stakeholders involved in the payroll process:
\begin{itemize}
\item {} 
\sphinxAtStartPar
written communication skills, such as preparing employee emails and memos, management reports, policies and procedures and correspondence with various levels of government

\item {} 
\sphinxAtStartPar
verbal communication skills, to be able to respond to internal and external stakeholder inquiries

\item {} 
\sphinxAtStartPar
the ability to read, understand and interpret legal terminology found in documents such as collective agreements, benefit contracts and government regulations

\item {} 
\sphinxAtStartPar
excellent mathematical skills to perform various calculations

\item {} 
\sphinxAtStartPar
problem solving, decision\sphinxhyphen{}making, time management and organizational skills

\end{itemize}

\sphinxAtStartPar
\sphinxstylestrong{Behavioural and Ethical Standards}

\sphinxAtStartPar
Professional behaviour and ethical conduct are critical components of an effective payroll practitioner’s skill set. In this
role, individuals must demonstrate trustworthiness, given the constant potential for fraud. Attention to detail is essential,
making conscientiousness a valued trait.

\sphinxAtStartPar
Payroll professionals handle sensitive personal and financial data, so discretion is non\sphinxhyphen{}negotiable. They must also be
tactful when interacting with employees, particularly in conversations involving financial concerns, which may be emotionally
charged. Perceptiveness helps practitioners understand multiple perspectives in complex situations.

\sphinxAtStartPar
The ability to work under pressure is key, especially when managing absolute deadlines. Sound judgment and common sense allow
practitioners to identify problems quickly and implement effective solutions. Finally, maintaining objectivity and a factual
approach when responding to questions and inquiries ensures fair and consistent communication across the organization.

\sphinxAtStartPar
Effective payroll professionals should be:
\begin{itemize}
\item {} 
\sphinxAtStartPar
trustworthy, as the potential for fraud is ever present

\item {} 
\sphinxAtStartPar
conscientious, with a keen attention to detail

\item {} 
\sphinxAtStartPar
discreet, due to the confidential nature of information being handled

\item {} 
\sphinxAtStartPar
tactful in dealing with employees who can be very sensitive when discussing their financial issues

\item {} 
\sphinxAtStartPar
perceptive, able to understand all sides of an issue

\item {} 
\sphinxAtStartPar
able to work under the pressures of absolute deadlines

\item {} 
\sphinxAtStartPar
able to use common sense in order to recognize problems quickly and apply sound solutions

\item {} 
\sphinxAtStartPar
able to remain objective and maintain a factual perspective when dealing with questions and inquiries

\end{itemize}


\section{Payroll Stakeholders}
\label{\detokenize{compliance:payroll-stakeholders}}
\sphinxAtStartPar
Stakeholders refer to the individuals, groups, and organizations—both within and outside the company—that have a vested
interest in the operations and outcomes of the payroll department. These stakeholders can be viewed as internal customers,
and payroll practitioners are encouraged to adopt a proactive, service\sphinxhyphen{}oriented approach in meeting their needs and
expectations.

\sphinxAtStartPar
Payroll management stakeholders are the federal and provincial/territorial governments, the
internal stakeholders and the external stakeholders. Internal stakeholders include employees,
employers and other departments within the organization. External stakeholders include
benefit carriers, courts, unions, pension providers, charities, third party administrators and
outsource/software vendors.


\subsection{Government Stakeholders}
\label{\detokenize{compliance:government-stakeholders}}
\sphinxAtStartPar
Government legislation establishes the rules and regulations that govern payroll practices, particularly in relation to
employee compensation. It is therefore essential for payroll practitioners to understand both the scope and the origin of all
payroll\sphinxhyphen{}related laws.

\sphinxAtStartPar
Canada is ruled by a federal government with ten largely self\sphinxhyphen{}governing provinces and three
territories controlled by the federal government. Payroll practitioners have to be compliant
not only with the federal government legislation, but with the provincial and territorial
governments’ legislation as well.

\sphinxAtStartPar
As a result, payroll departments are directly influenced by legislative developments at both the federal and provincial or
territorial levels, making ongoing legal awareness a critical component of payroll management.

\sphinxAtStartPar
The federal parliament has the power to make laws for the peace, order and good government
of Canada. The federal cabinet is responsible for most of the legislation introduced by
parliament, and has the sole power to prepare and introduce tax legislation involving the
expenditure of public money.

\sphinxAtStartPar
The provincial/territorial legislatures have power over direct taxation in the province or
territory for the purposes of natural resources, prisons (except for federal penitentiaries),
charitable institutions, hospitals (except marine hospitals), municipal institutions, education,
licences for provincial/territorial and municipal revenue purposes, local works, incorporation
of provincial/territorial organizations, the creation of courts and the administration of justice,
fines and penalties for breaking provincial/territorial laws.

\sphinxAtStartPar
Both the federal and provincial/territorial governments have power over agriculture,
immigration and certain aspects of natural resources. Should their laws conflict, federal law
prevails.

\sphinxAtStartPar
In the case of old age, disability, and survivor’s pensions, again both the federal and
provincial/territorial governments have power. In this instance, if their laws conflict, the
provincial/territorial power prevails.

\sphinxAtStartPar
The federal government cannot transfer any of its powers to a provincial/territorial
legislature, nor can a provincial/territorial legislature transfer any of its powers to the federal
government. The federal government can, however, delegate the administration of a federal
act to a provincial/territorial agency, and a provincial/territorial legislature can delegate the
administration of a provincial/territorial act to a federal agency.
\begin{quote}

\sphinxAtStartPar
As all provinces and territories (except Québec) have delegated the administration of the
collection of income tax deductions to the federal government, the Canada Revenue Agency
(CRA) collects income tax withheld from employees under both federal and
provincial/territorial requirements. Québec collects its provincial income tax directly.
\end{quote}


\subsection{Federal Government}
\label{\detokenize{compliance:federal-government}}
\sphinxAtStartPar
The Constitution Act of 1867 outlined the division of legislative power and authority between
federal and provincial/territorial jurisdictional governments. The exclusive legislative
authority of the Parliament of Canada extends to all matters regarding:
\begin{itemize}
\item {} 
\sphinxAtStartPar
regulation of trade and commerce

\item {} 
\sphinxAtStartPar
Employment Insurance

\item {} 
\sphinxAtStartPar
postal service

\item {} 
\sphinxAtStartPar
fixing and providing salaries and allowances for civil and other officers of the Government of Canada

\item {} 
\sphinxAtStartPar
navigation and shipping

\item {} 
\sphinxAtStartPar
ferries between a province and any British or foreign country or between two provinces

\item {} 
\sphinxAtStartPar
criminal law, except the Constitution of Courts of Criminal Jurisdiction, but including the Procedure in Criminal Matters

\item {} 
\sphinxAtStartPar
anything not specifically assigned to the provinces under this Act

\end{itemize}

\sphinxAtStartPar
The Canada Labour Code is legislation that consolidates certain statutes respecting labour.
Part I deals with Industrial Relations, Part II deals with Occupational Health and Safety and
Part III deals with Labour Standards. The primary objective of Part III is to establish and
protect employees’ and employers’ rights to fair and equitable conditions of employment.
Part III provisions establish minimum requirements concerning the working conditions of
employees under federal jurisdiction in the following industries and organizations:
\begin{itemize}
\item {} 
\sphinxAtStartPar
industries and undertakings of inter\sphinxhyphen{}provincial/territorial, national, or international nature, that is, transportation, communications, radio and television broadcasting, banking, uranium mining, grain elevators, and flour and feed operations

\item {} 
\sphinxAtStartPar
organizations whose operations have been declared for the general advantage of Canada or two or more provinces, and such Crown corporations as Canada Post Corporation, and the Canadian Broadcasting Corporation (CBC)

\end{itemize}


\subsection{Provincial/Territorial Governments}
\label{\detokenize{compliance:provincial-territorial-governments}}
\sphinxAtStartPar
Under the Constitution Act of 1867, the exclusive legislative authority of the provinces and
territories exists over:
\begin{itemize}
\item {} 
\sphinxAtStartPar
all laws regarding property and civil rights, which give the provinces/territories the authority to enact legislation to establish employment standards for working conditions

\item {} 
\sphinxAtStartPar
employment in manufacturing, mining, construction, wholesale and retail trade, service industries, local businesses and any industry or occupation not specifically covered under federal jurisdiction

\end{itemize}

\sphinxAtStartPar
Canada’s division of authority between federal and provincial or territorial governments directly influences payroll
practices, particularly in relation to employment and labour standards. These standards are governed independently by
each province and territory, and outline key rules related to workplace conditions.

\sphinxAtStartPar
Among the issues addressed are hours of work, minimum wage, overtime eligibility, vacation entitlements, and termination pay.
Because each jurisdiction sets its own legislation, payroll practitioners must ensure compliance with the specific
requirements applicable to the location where the employee works. Navigating these variations is an essential aspect of
effective and lawful payroll administration.
\begin{quote}

\sphinxAtStartPar
\sphinxstylestrong{Example:}

\sphinxAtStartPar
The Gap is a retail business with stores across Canada. The workers in each store are
governed under the employment/labour standards legislated in the jurisdiction in which they
work. For example, the minimum general hourly wage in effect January 1, 2020 (which is
governed by provincial/territorial employment/labour standards) is higher in Ontario than in
Prince Edward Island. An employee working in Ontario would receive a higher hourly
minimum wage than an employee with the same position in Prince Edward Island.

\sphinxAtStartPar
Employers must follow the employment/labour standards legislated by the jurisdiction in
which their employees work, unless they are governed by federal labour standards. Federal
labour standards apply to certain industries and organizations, regardless of where the
employees work.
\end{quote}

\sphinxAtStartPar
The person or persons performing the payroll function must clearly understand under which
employment/labour standards jurisdiction the employees of the organization fall.
Organizations may have some employees who fall under federal jurisdiction and another
group of employees who fall under provincial/territorial legislation.


\subsection{Internal Stakeholders}
\label{\detokenize{compliance:internal-stakeholders}}
\sphinxAtStartPar
Internal stakeholders are the people and departments within the organization that rely on the payroll function to operate
effectively. They form the core audience served by payroll and include employees who depend on accurate and timely
compensation, employers who oversee workforce management, and other internal teams—such as human resources, finance,
and operations—that collaborate closely with payroll for data sharing, planning, and compliance. These stakeholders play a
direct role in shaping how payroll services are delivered and supported across the organization.

\sphinxAtStartPar
\sphinxstylestrong{Employers} \sphinxhyphen{} Management may require certain information from payroll to make sound
business decisions.

\sphinxAtStartPar
\sphinxstylestrong{Employees} \sphinxhyphen{} Employees require that their pay is received in a timely and accurate manner to
meet personal obligations. Employees must also be assured that their personal information is
kept confidential.

\sphinxAtStartPar
\sphinxstylestrong{Other departments} \sphinxhyphen{} Many departments interact with payroll, either for information or
reporting. According to the Canadian Payroll Association’s 2020 National Payroll Week
(NPW) Payroll Professional Research Survey, fifty\sphinxhyphen{}five percent of payroll practitioners
report through the finance department and thirty\sphinxhyphen{}two percent report through the human
resources department. Information such as general ledger posting, payroll and benefit costs
and salary information must flow between payroll, human resources and finance in formats
needed for their various requirements.

\sphinxAtStartPar
In addition, other departments such as contracts and manufacturing often need payroll
information for budgeting, analytical and quality purposes.


\subsection{External Stakeholders}
\label{\detokenize{compliance:external-stakeholders}}
\sphinxAtStartPar
External stakeholders are entities outside of both the organization and government that maintain a collaborative or
service\sphinxhyphen{}based relationship with the payroll function. These may include benefit providers, insurance carriers, pension
plan administrators, unions, and third\sphinxhyphen{}party service vendors. Although not formally part of the company or regulatory bodies,
their involvement directly impacts payroll operations.

\sphinxAtStartPar
Ensuring compliance with external stakeholder requirements is a key duty of the payroll department. This often includes
verifying data, meeting contractual obligations, and coordinating financial transactions. In many cases, payroll
professionals must initiate cheque requests through accounts payable and submit accompanying documentation to these
organizations to fulfill obligations accurately and on time. Maintaining strong communication and attention to detail with
external partners is essential for smooth and compliant payroll administration.

\sphinxAtStartPar
\sphinxstylestrong{Benefit Carriers} are insurance companies that provide benefit coverage to employees.
Payroll is responsible for deducting and remitting premiums for the insurance coverage to the
carriers and for providing reports on employee enrolment and coverage levels.

\sphinxAtStartPar
\sphinxstylestrong{Courts and the CRA} require payroll to accurately deduct and remit amounts ordered to be
withheld through garnishments, third party demands, requirements to pay and support
deduction orders.

\sphinxAtStartPar
\sphinxstylestrong{Unions} require that payroll accurately deduct and remit union dues and initiation fees, and to
ensure that the terms of the collective agreement are adhered to. It is estimated that just under
one\sphinxhyphen{}third of the workforce in Canada belongs to a trade union. Payroll professionals must be
familiar with the role and activities of trade unions and the responsibilities of the employer
and the payroll department in a unionized environment.

\sphinxAtStartPar
\sphinxstylestrong{Pension Providers} are third party pension plan providers that may require payroll to provide
enrolment reports on participating employees and length of service calculations, and to remit
employee deductions and employer contributions

\sphinxAtStartPar
\sphinxstylestrong{Charities} have arrangements with some organizations to facilitate employee donations
through payroll deductions. Payroll is responsible for remitting these deductions to the
charity.

\sphinxAtStartPar
\sphinxstylestrong{Third Party Administrators} are organizations that affect the administration of the payroll
function. Examples of these external stakeholders are banking institutions or benefit
organizations that offer Group Registered Retirement Saving Plans (RRSP). Payroll is
responsible for deducting any employee contributions and remitting employer and employee
contributions to the plan administrator.

\sphinxAtStartPar
\sphinxstylestrong{Outsource/Software vendors} are payroll service providers or payroll software vendors that
work with the payroll department to ensure the payroll is being processed accurately and
efficiently.


\section{Legislations and Regulations}
\label{\detokenize{compliance:legislations-and-regulations}}
\sphinxAtStartPar
Federal and provincial/territorial legislation, and amendments to existing legislation and
regulations, can affect the operations of a payroll department, as the requirement to comply
with the new or amended legislation must be satisfied.

\sphinxAtStartPar
It is important to note the difference between legislation and regulatio. \sphinxstylestrong{Legislation} determines what the rules are, while \sphinxstylestrong{regulations} determine how the rules are to be
applied.

\sphinxAtStartPar
The methods for calculating income tax deductions are specified by the federal government through regulations.
\begin{quote}

\sphinxAtStartPar
\sphinxstyleemphasis{Example:}

\sphinxAtStartPar
\sphinxstyleemphasis{The Income Tax Act}

\sphinxAtStartPar
The legislation: Specifies that employers are required to withhold income tax from employees.

\sphinxAtStartPar
The regulation: Specifies the taxation methods that should be used for non\sphinxhyphen{}periodic payments such as bonuses, retroactive pay increases, lump sum payments, etc.

\sphinxAtStartPar
\sphinxstyleemphasis{Non\sphinxhyphen{}periodic bonus payments}

\sphinxAtStartPar
Where a payment in respect of a bonus is made by an employer to an employee whose total remuneration (including the bonus) from the employer
may reasonably be expected to exceed \$5,000 in the taxation year of the employee in which the payment is made, the amount to be deducted or withheld by the employer is dictated
through a calculation prescribed in the regulation within the Act.
\end{quote}

\sphinxAtStartPar
Legislative changes can present significant challenges for payroll departments, especially when implemented mid\sphinxhyphen{}year or
applied retroactively. These adjustments often require updates to individual payroll records, additional reconciliation
efforts, and revisions to year\sphinxhyphen{}end balancing procedures, placing extra demands on payroll professionals.

\sphinxAtStartPar
Labour legislation in particular is subject to frequent modifications, including amendments, repeals, and revisions.
Therefore, it is critical for payroll practitioners to remain informed about the laws and regulatory updates relevant to
each jurisdiction in which their organization operates.

\sphinxAtStartPar
Legislative changes are typically communicated through public media. In larger organizations, updates may also be shared
internally by human resources, tax specialists, or legal departments. Regardless of the organization’s size, payroll
professionals should take a proactive role in monitoring relevant developments and ensuring that all affected parties are
made aware of any changes. A variety of resources—such as government publications, industry newsletters, professional
associations, and online portals—can support this ongoing effort to stay informed and maintain compliance.

\sphinxAtStartPar
The following are some of the available resources:
\begin{itemize}
\item {} 
\sphinxAtStartPar
The Canadian Payroll Association offers a phone and email information service, Payroll InfoLine, for members’ payroll related questions. The Association also has a website for members, www.payroll.ca, that contains guidelines, legislative updates and other useful payroll related information. As well, the Association is available on Twitter(@cdnpayroll), LinkedIn (The Canadian Payroll Association) and Facebook (@canadianpayroll).

\item {} 
\sphinxAtStartPar
The Canada Revenue Agency (CRA) produces guides, publications, Income Tax Bulletins, folios and Circulars, posts news bulletins and enables participation on an electronic mailing list with e\sphinxhyphen{}mail alerts for new content to the Canada.ca website.

\item {} 
\sphinxAtStartPar
The Revenu Québec (RQ) website provides guides, publications, bulletins, forms, online services and enables participation on an electronic mailing list with e\sphinxhyphen{}mail notifications of tax news articles \sphinxhyphen{} \sphinxurl{https://www.revenuquebec.ca/en/}

\item {} 
\sphinxAtStartPar
Employment/labour standards (federal, provincial and territorial) publications and websites. Each jurisdiction has a website providing information on their employment/labour standards. For example, the websites for Alberta and Québec are: Alberta \sphinxhyphen{} \sphinxurl{https://www.alberta.ca/employment-standards.aspx} Québec \sphinxhyphen{} www.cnt.gouv.qc.ca/en

\item {} 
\sphinxAtStartPar
Employment and Social Development Canada (ESDC) and Service Canada (SC) publications including information regarding the Employment Insurance (EI) program and the Social Insurance Number \sphinxhyphen{} www.canada.ca

\item {} 
\sphinxAtStartPar
CCH Canada Limited publishes a series of volumes on employment and labour law, pensions and benefits, etc., that supplies information on legislation with regular updates as changes become law \sphinxhyphen{} www.cch.ca

\item {} 
\sphinxAtStartPar
Carswell publishes The Canadian Payroll Manual and offers a phone and email service to subscribers \sphinxhyphen{} www.carswell.com

\end{itemize}


\subsection{Legislative Compliance}
\label{\detokenize{compliance:legislative-compliance}}
\sphinxAtStartPar
Payroll plays a critical role not only in ensuring that employees are paid accurately and on time, but also in
supporting and maintaining compliance with numerous government regulations. This includes legislative obligations related to
payroll source deductions, Canada Pension Plan contributions, Employment Insurance premiums, and both federal and
provincial/territorial income tax withholdings. When these obligations are not met, employers may face serious consequences,
including financial penalties or legal enforcement actions designed to ensure compliance.

\sphinxAtStartPar
Penalties such as fines, interest charges, and legal sanctions often result from audits or investigations into
non\sphinxhyphen{}compliance. In more severe cases, enforcement measures may include seizure of bank accounts or assets, and fines.

\sphinxAtStartPar
To monitor and enforce these requirements, government agencies utilize a range of tracking systems. Some, such as those
used to oversee source deduction remittances, rely on strict reporting schedules that create a consistent audit trail.
Failure to meet these time\sphinxhyphen{}sensitive obligations typically triggers a swift response and the imposition of penalties.

\sphinxAtStartPar
For reporting requirements that are less frequent or ongoing, the consequences of non\sphinxhyphen{}compliance may not be immediate.
However, they can lead to scrutiny from auditors or other officials tasked with verifying that payroll practices align with
current legislative standards. Payroll professionals must remain vigilant and informed to protect their organization from
financial and legal risk.
\begin{quote}

\sphinxAtStartPar
\sphinxstyleemphasis{Example:}

\sphinxAtStartPar
\sphinxstylestrong{Record of Employment (ROE)} issuance
Failure to issue a ROE within the established deadlines may result in a visit from an investigative officer from Service Canada.

\sphinxAtStartPar
The Canada Revenue Agency’s Pensionable and Insurable Earnings Review (PIER) is an annual compliance review system. This system utilizes the data provided on the T4
information slips issued at year\sphinxhyphen{}end to validate the amounts of CPP contributions and EI premiums deducted by employers, and identifies any remittance deficiencies.
\end{quote}


\subsection{Self\sphinxhyphen{}Assessment}
\label{\detokenize{compliance:self-assessment}}
\sphinxAtStartPar
Both the federal and provincial/territorial tax systems in Canada operate on the principle of self\sphinxhyphen{}assessment. Under this system, taxpayers and their representatives—including employers—are responsible for accurately calculating, reporting, and remitting taxes and other required contributions by the prescribed deadlines.

\sphinxAtStartPar
The Canada Revenue Agency (CRA) and Revenu Québec (RQ) serve as administrators of these systems, ensuring that individuals and organizations remain compliant and that all amounts owed are properly paid.

\sphinxAtStartPar
Importantly, both agencies acknowledge the right of taxpayers to organize their financial affairs in a way that minimizes their tax liability, provided it remains within legal boundaries. While tax planning is permitted, tax evasion—such as failing to report income, neglecting to remit amounts due, or submitting false information—is strictly prohibited and subject to enforcement actions.


\section{The Employee\sphinxhyphen{}Employer Relationship}
\label{\detokenize{compliance:the-employee-employer-relationship}}
\sphinxAtStartPar
Determining the nature of the working relationship between an individual and an organization is essential in all employment situations. Whether the individual is classified as an employee or self\sphinxhyphen{}employed directly affects the statutory withholding requirements and the organization’s compliance with applicable legislation. To support this assessment, the Canada Revenue Agency (CRA) provides a set of guidelines designed to help distinguish between the two classifications. Importantly, the decision is not made by the worker but must be based on objective criteria and legal standards.

\sphinxAtStartPar
Payroll practitioners play an important role in promoting awareness of this distinction throughout the organization. By proactively communicating the significance of establishing a valid employee\sphinxhyphen{}employer relationship, payroll professionals help ensure that employment classifications are accurate and compliant.

\sphinxAtStartPar
Once an employee\sphinxhyphen{}employer relationship is confirmed, the payroll department becomes responsible for meeting compliance obligations related to statutory withholdings. This includes deducting the appropriate amounts—such as income tax, Canada Pension Plan contributions, and Employment Insurance premiums—from employee pay and remitting them to the government within the required timelines. Proper classification and adherence to these rules are key to maintaining legal and financial accountability.

\sphinxAtStartPar
Where an employee\sphinxhyphen{}employer relationship exists, the CRA requires the employer to:
\begin{itemize}
\item {} 
\sphinxAtStartPar
register with the Canada Revenue Agency for a Business Number (BN)

\item {} 
\sphinxAtStartPar
withhold the statutory deductions of income tax, Canada Pension Plan (CPP) contributions, and Employment Insurance (EI) premiums on amounts paid to employees

\item {} 
\sphinxAtStartPar
remit the amounts withheld as well as the required employer’s share of CPP contributions and EI premiums to the Canada Revenue Agency

\item {} 
\sphinxAtStartPar
report the employees’ income and deductions on the appropriate information return

\item {} 
\sphinxAtStartPar
give the employees copies of their T4 slips by the end of February of the following calendar year

\end{itemize}

\sphinxAtStartPar
Information on the factors to consider when determining whether an employee\sphinxhyphen{}employer relationship exists can be found in the
Canada Revenue Agency guide, Employee or Self\sphinxhyphen{}Employed? \sphinxhyphen{} RC4110. The guide is available on the CRA’s website,
\sphinxurl{https://www.canada.ca/en/revenue-agency.html}.


\subsection{Contract of Service (Employment)}
\label{\detokenize{compliance:contract-of-service-employment}}
\sphinxAtStartPar
A \sphinxstylestrong{contract of service} is an arrangement whereby an individual (the employee) agrees to
work on a full\sphinxhyphen{}time or part\sphinxhyphen{}time basis for an employer for a specified or indeterminate period
of time.

\sphinxAtStartPar
Under a contract of service, one party serves another in return for a salary or some other form
of remuneration.


\subsection{Contract for Service (Subcontracting)}
\label{\detokenize{compliance:contract-for-service-subcontracting}}
\sphinxAtStartPar
A \sphinxstylestrong{contract for service} is a business relationship whereby one party agrees to perform certain
specific work stipulated in the contract for another party. It usually calls for the
accomplishment of a clearly defined task but does not normally require that the contracting
party do anything him/herself. A person who carries out a contract for service may be
considered a contract worker, a self\sphinxhyphen{}employed person or an independent contractor.

\sphinxAtStartPar
A business relationship is formed when a self\sphinxhyphen{}employed individual enters into a verbal or written agreement to complete
specific work for a payer in exchange for compensation. This arrangement does not establish an employer\sphinxhyphen{}employee relationship;
instead, it represents a contract for services.

\sphinxAtStartPar
Under this type of agreement, the self\sphinxhyphen{}employed individual is responsible for delivering a final result within an agreed
timeframe, using methods of their own choosing. They are not subject to the direction or supervision of the payer while
completing the work, and they retain autonomy over how tasks are executed. In most cases, the payer does not participate in
or influence the work process, meaning control over the work lies entirely with the self\sphinxhyphen{}employed individual. This structure
reflects a high level of independence and flexibility, distinguishing it clearly from traditional employment relationships.

\sphinxAtStartPar
Under a contract for service, a self\sphinxhyphen{}employed individual accepts both the potential for profit and the risk of financial loss.
Prior to engagement, the individual agrees on the total cost of the work, uses personal tools and equipment, and assumes full
responsibility for how the work is performed. This means the individual bears any unforeseen costs or challenges that arise
during the project. Conversely, if the work is completed more efficiently than expected, the financial gain—through retained
profits—is greater.

\sphinxAtStartPar
Organizations often utilize contracts for service when they require tasks or projects that fall outside their normal business
operations. While the relationship between a payer and a self\sphinxhyphen{}employed contractor may resemble that of an employer and
employee, there is a key distinction. In a contract for service, the payer specifies the desired outcome or deliverable, but
not how the work should be completed. In contrast, a contract of service allows the employer to direct both the tasks and the
method by which they are carried out, establishing a more controlled, employee\sphinxhyphen{}based relationship.

\sphinxAtStartPar
Under a contract for service, the payer exercises general oversight to ensure that the agreed work is completed as specified.
However, this oversight does not extend to controlling how the work is performed. The self\sphinxhyphen{}employed individual retains
autonomy over the methods used to complete the tasks. Receiving general instructions from a project manager or similar
representative does not establish an employer\sphinxhyphen{}employee relationship.

\sphinxAtStartPar
An employee\sphinxhyphen{}employer relationship is recognized when an organization has the authority to direct and control both the work and
the manner in which it is performed. If there is uncertainty about whether such a relationship exists, the Canada Revenue
Agency (CRA) recommends submitting Form CPT1 — Request for a CPP/EI Ruling: Employee or Self\sphinxhyphen{}Employed? — to obtain
clarification. A sample of this form can be found at the end of this section.

\sphinxAtStartPar
Independent contractors or self\sphinxhyphen{}employed individuals are not classified as employees if no employer\sphinxhyphen{}employee relationship is
present. They typically submit invoices and receive payment through accounts payable. However, the act of submitting an
invoice alone is not sufficient to confirm self\sphinxhyphen{}employment status. Proper assessment of the working relationship is essential
sto ensure accurate classification and compliance with tax and labor regulations.


\subsection{Factors Determining the Type of Contract}
\label{\detokenize{compliance:factors-determining-the-type-of-contract}}
\sphinxAtStartPar
The CRA uses a two\sphinxhyphen{}step approach to examine the relationship between the worker and the
payer for relationships outside the province of Québec. The approach used for relationships
in the province of Québec will be discussed in a later chapter.

\sphinxAtStartPar
\sphinxstylestrong{Step 1:}
The first step is to establish what the intent was when the worker and the payer entered into
the working arrangement. Did they intend to enter into an employee\sphinxhyphen{}employer relationship
(contract of service) or did they intend to enter into a business relationship (contract for
service). The CRA must determine not only how the working relationship has been defined
but why it was defined that way.

\sphinxAtStartPar
\sphinxstylestrong{Step 2:}
The CRA then considers certain factors when determining if a contract of service or a
contract for service exists. In order to understand the working relationship and verify that the
intent of the worker and the payer is reflected in the facts, they will ask a series of questions
that relate to the following factors:
\begin{itemize}
\item {} 
\sphinxAtStartPar
the level of control the payer has over the worker

\item {} 
\sphinxAtStartPar
whether or not the worker provides the tools and equipment

\item {} 
\sphinxAtStartPar
whether the worker can subcontract the work or hire assistants

\item {} 
\sphinxAtStartPar
the degree of financial risk taken by the worker

\item {} 
\sphinxAtStartPar
the degree of responsibility for investment and management held by the worker

\item {} 
\sphinxAtStartPar
the worker’s opportunity for profit

\item {} 
\sphinxAtStartPar
any other relevant factors, such as written contracts

\end{itemize}

\sphinxAtStartPar
The CRA will look at the answers independently and then together and consider whether or
not they reflect the intent that was originally stated. Considered individually, the response to
each of these questions is not conclusive; however, when weighed together, certain
conclusions may be drawn. When there is no common intent, the CRA will decide if the
answers are more consistent with a contract of service or a contract for service.
Each of these factors will be discussed in the material and indicators showing whether the
worker is an employee or self\sphinxhyphen{}employed will be provided.


\subsection{Control}
\label{\detokenize{compliance:control}}
\sphinxAtStartPar
One of the key factors in determining a worker’s status is the extent to which the payer has the ability, authority, or
right to control both what work is performed and how it is carried out. Equally important is the level of independence the
worker maintains in performing their duties.

\sphinxAtStartPar
In evaluating the relationship, both the payer’s oversight of the worker’s day\sphinxhyphen{}to\sphinxhyphen{}day activities and their overall
influence are assessed. However, it is the payer’s right to exercise control—rather than whether that control is actively
used—that holds the most weight in determining the nature of the working relationship.

\sphinxAtStartPar
Worker is an \sphinxstyleemphasis{Employee} when:
\begin{itemize}
\item {} 
\sphinxAtStartPar
The relationship is one of subordination.

\item {} 
\sphinxAtStartPar
The payer will often direct, scrutinize, and effectively control many elements of how the work is performed.

\item {} 
\sphinxAtStartPar
The payer controls both the results of the work and the method used to do the work.

\item {} 
\sphinxAtStartPar
The payer determines what jobs the worker will do.

\item {} 
\sphinxAtStartPar
The worker receives training or direction from the payer on how to do the work.

\end{itemize}

\sphinxAtStartPar
Worker is a \sphinxstyleemphasis{Self\sphinxhyphen{}Employed} when:
\begin{itemize}
\item {} 
\sphinxAtStartPar
Individual usually works independently, does not have anyone overseeing them.

\item {} 
\sphinxAtStartPar
The worker is usually free to work when and for whom they choose and may provide their services to different payers at the same time.

\item {} 
\sphinxAtStartPar
The worker can accept or refuse work from the payer.

\item {} 
\sphinxAtStartPar
The working relationship between the payer and the worker does not present a degree of continuity, loyalty, security, subordination, or integration.

\end{itemize}


\subsection{Tools and Equipment}
\label{\detokenize{compliance:tools-and-equipment}}
\sphinxAtStartPar
Ownership of tools and equipment is not a definitive factor in determining the nature of a working relationship or the type
of contract in place. While self\sphinxhyphen{}employed individuals often use their own tools to fulfill contractual obligations—making
such ownership common in business relationships—this alone does not confirm self\sphinxhyphen{}employment. Employees may also be required
to supply their own tools, depending on the trade or occupation.

\sphinxAtStartPar
In typical employee\sphinxhyphen{}employer relationships, the employer provides the necessary equipment and assumes the costs associated
with its use, including maintenance, insurance, transportation, rental fees, and operational expenses such as fuel. However,
in certain industries—such as automotive repair, painting, carpentry, and technical fields like software development or
surveying—it is standard practice for employees to use their own tools or specialized instruments.

\sphinxAtStartPar
In contrast, self\sphinxhyphen{}employed individuals not only supply their own equipment but also bear the associated costs. Significant
financial investment in tools—especially those that require ongoing maintenance or replacement—can suggest a business
relationship, as these individuals assume both the potential for profit and the risk of loss.

\sphinxAtStartPar
Ultimately, the key consideration is the scale of the investment and the financial responsibility related to repairs,
replacement, and insurance, rather than mere ownership itself.

\sphinxAtStartPar
The worker is an employee when:
\begin{itemize}
\item {} 
\sphinxAtStartPar
The payer supplies most of the tools and equipment.

\item {} 
\sphinxAtStartPar
The payer retains the right of use over the tools and equipment provided to the worker.

\item {} 
\sphinxAtStartPar
The worker supplies the tools and equipment and the payer reimburses the worker for their use

\end{itemize}

\sphinxAtStartPar
The worker is a self\sphinxhyphen{}employed individual when:
\begin{itemize}
\item {} 
\sphinxAtStartPar
The worker provides the tools and equipment required and is responsible for the cost of repairs, insurance and maintenance and retains the right over the use of these assets.

\item {} 
\sphinxAtStartPar
The worker supplies his or her own workspace, is responsible for the costs to maintain it, and does substantial work from that site.

\end{itemize}


\subsection{Subcontracting Work or Hiring Assistants}
\label{\detokenize{compliance:subcontracting-work-or-hiring-assistants}}
\sphinxAtStartPar
As subcontracting work or hiring assistants can affect a worker’s chance of profit or risk of loss, this can help determine
the type of business relationship.

\sphinxAtStartPar
The worker is an employee when:
\begin{itemize}
\item {} 
\sphinxAtStartPar
The worker cannot hire helpers or assistants.

\item {} 
\sphinxAtStartPar
The worker must perform the services personally.

\end{itemize}

\sphinxAtStartPar
The worker is a self\sphinxhyphen{}employed individual when:
\begin{itemize}
\item {} 
\sphinxAtStartPar
The worker does not have to perform the service personally.

\item {} 
\sphinxAtStartPar
They can hire another party to complete the work, without consulting with the payer.

\end{itemize}


\subsection{Financial Risk}
\label{\detokenize{compliance:financial-risk}}
\sphinxAtStartPar
When evaluating the nature of a working relationship, the Canada Revenue Agency (CRA) considers whether the individual
incurs fixed, ongoing costs or unreimbursed expenses. In traditional employee arrangements, employers typically reimburse
expenses that arise as part of the job—for example, travel or business\sphinxhyphen{}related costs.

\sphinxAtStartPar
In contrast, self\sphinxhyphen{}employed individuals often assume greater financial risk by covering recurring operational costs
regardless of whether active work is being performed. These may include equipment leasing, office space rental, or other
business overheads. While both employees and contractors may receive reimbursement for certain expenses, the CRA places
particular emphasis on identifying costs that are not reimbursed. The presence of such expenses may indicate a business
relationship, reflecting the independence and financial responsibility characteristic of self\sphinxhyphen{}employment.

\sphinxAtStartPar
The worker is an employee when:
\begin{itemize}
\item {} 
\sphinxAtStartPar
The worker is not usually responsible for any operating expenses.

\item {} 
\sphinxAtStartPar
The worker is not financially liable if he or she does not fulfill the obligations of the contract.

\item {} 
\sphinxAtStartPar
The payer determines and controls the method and amount of pay.

\end{itemize}

\sphinxAtStartPar
The worker is a self\sphinxhyphen{}employed individual when:
\begin{itemize}
\item {} 
\sphinxAtStartPar
The worker is financially liable if he or she does not fulfill the obligations of the contract.

\item {} 
\sphinxAtStartPar
The worker does not receive any protection or benefits from the payer.

\item {} 
\sphinxAtStartPar
The worker hires helpers to assist and pays them.

\item {} 
\sphinxAtStartPar
The worker advertises the services offered.

\end{itemize}


\subsection{Responsibility for Investment and Management}
\label{\detokenize{compliance:responsibility-for-investment-and-management}}
\sphinxAtStartPar
When assessing whether a business relationship exists, one important indicator is the worker’s financial investment in the
services they provide. If an individual is required to invest in equipment, materials, or other resources to complete the
work, this suggests the presence of a contract for service rather than an employment relationship.

\sphinxAtStartPar
Another key factor is decision\sphinxhyphen{}making authority related to financial outcomes. When the worker independently makes business
decisions that influence their profit or loss—such as pricing, project selection, or service delivery methods—it further
supports the classification of a self\sphinxhyphen{}employed individual operating under a business arrangement. These characteristics
reflect the autonomy and financial risk typically associated with self\sphinxhyphen{}employment.

\sphinxAtStartPar
The worker is an employee when:
\begin{itemize}
\item {} 
\sphinxAtStartPar
The worker has no capital investment in the business.

\item {} 
\sphinxAtStartPar
The worker does not have a business presence.

\end{itemize}

\sphinxAtStartPar
The worker is a self\sphinxhyphen{}employed individual when:
\begin{itemize}
\item {} 
\sphinxAtStartPar
The worker has capital investment, manages his or her staff, hires and pays individuals to help perform the work, and has established a business presence.

\end{itemize}


\subsection{Opportunity for Profit}
\label{\detokenize{compliance:opportunity-for-profit}}
\sphinxAtStartPar
A business relationship is often indicated when a worker has the ability to realize a profit or incur a loss, reflecting their
control over the financial and operational aspects of the services they provide. Self\sphinxhyphen{}employed individuals typically
negotiate their own rates, choose which contracts to accept, and may take on multiple contracts simultaneously. To fulfill
contractual obligations, they often incur and manage expenses, which directly influence their potential for profit.

\sphinxAtStartPar
In contrast, employees generally do not bear financial risk or benefit from profit. While commission\sphinxhyphen{}based employees may
increase their earnings through performance, this does not represent profit in the traditional sense, as it does not reflect
income earned beyond expenses. Moreover, employees do not typically share in a business’s profits or losses.

\sphinxAtStartPar
When assessing worker classification, the Canada Revenue Agency (CRA) considers the extent to which the individual controls
their revenue and expenses. Another key factor is the method of payment: employees are usually compensated at a fixed rate
based on a consistent pay schedule (e.g., hourly, weekly, or annually). Self\sphinxhyphen{}employed individuals, however, are often paid a
flat rate for a specific job, especially when they absorb related costs—an arrangement that commonly signals a business
relationship.

\sphinxAtStartPar
The worker is an employee when:
\begin{itemize}
\item {} 
\sphinxAtStartPar
The worker is not in a position to realize a business profit or loss.

\item {} 
\sphinxAtStartPar
The worker is entitled to benefit plans that are normally only offered to employees.

\end{itemize}

\sphinxAtStartPar
The worker is a self\sphinxhyphen{}employed individual when:
\begin{itemize}
\item {} 
\sphinxAtStartPar
The worker is compensated by a flat fee.

\item {} 
\sphinxAtStartPar
The worker can hire and pay a substitute.

\end{itemize}

\sphinxAtStartPar
The worker is an employee when:

\sphinxAtStartPar
The worker is a self\sphinxhyphen{}employed individual when:


\section{Review Summary}
\label{\detokenize{compliance:review-summary}}
\sphinxAtStartPar
The core purpose of the payroll function within any organization is to ensure employees are compensated accurately and
punctually, in accordance with all applicable legislation throughout the full annual payroll cycle. This essential function
supports employee satisfaction, regulatory compliance, and overall operational efficiency.

\sphinxAtStartPar
Payroll refers to the systematic process of remunerating employees for their services. It involves calculating earnings,
applying deductions for taxes and benefits, and issuing payments through approved channels. Precision in these processes is
critical to avoid financial discrepancies and foster organizational trust.

\sphinxAtStartPar
Legislation encompasses the legal framework enacted by federal, provincial, and territorial bodies that governs payroll
activities. This includes tax laws, employment standards, and workplace rights. Compliance means adhering to these legal
requirements to prevent penalties and uphold ethical business practices.

\sphinxAtStartPar
To execute payroll duties effectively, practitioners must possess comprehensive knowledge of payroll legislation, operational
processes, and reporting obligations. Beyond technical expertise, strong interpersonal and professional skills are essential,
enabling practitioners to adapt to legislative changes and uphold standards of accountability.

\sphinxAtStartPar
Stakeholders—both internal and external—have a vested interest in the payroll function’s integrity and outcomes. Internally,
this includes employees, employers, and interconnected departments such as human resources and finance. Externally,
stakeholders may include benefit providers, unions, pension administrators, charitable organizations, legal entities, and
software vendors. Their interaction with payroll processes influences expectations around accuracy, compliance, and data
coordination.

\sphinxAtStartPar
Payroll governance is shaped by both federal and provincial/territorial authority. The federal government enacts legislation
that applies nationally, particularly for industries operating across provinces or those serving a broader national interest.
Provincial and territorial governments regulate regional matters such as civil rights, property, and employment standards
within local industries. Any sector not under federal oversight typically falls under provincial or territorial jurisdiction.

\sphinxAtStartPar
Employers are obliged to comply with the labour and employment standards applicable to the jurisdiction in which their
employees work—unless federal laws take precedence. Where legislation mandates compliance, enforcement may include financial
penalties or legal action to ensure accountability.

\sphinxAtStartPar
Employment relationships are defined through contractual arrangements. A contract of service refers to a traditional
employer\sphinxhyphen{}employee relationship, where an individual commits to working for an employer—either on a full\sphinxhyphen{}time or part\sphinxhyphen{}time
basis—for a specified or ongoing period. The employer has authority over both the duties and how they are executed.

\sphinxAtStartPar
Conversely, a contract for service reflects a business arrangement where an independent contractor agrees to perform specific
tasks, with discretion over how the work is completed. This signifies a client\sphinxhyphen{}provider relationship rather than an employment
one.

\sphinxAtStartPar
To assess worker classification—particularly outside Québec—the Canada Revenue Agency (CRA) employs a two\sphinxhyphen{}step evaluation. Key
considerations include:

\sphinxAtStartPar
Control: Whether the payer holds the right to determine what work is done and how it is executed.

\sphinxAtStartPar
Independence: The degree of autonomy exercised by the worker.

\sphinxAtStartPar
Ownership of Tools: Significant investment in tools and equipment, along with maintenance and insurance responsibilities, may indicate a business relationship.

\sphinxAtStartPar
Financial Risk: Ongoing operational costs or unreimbursed expenses reflect a higher likelihood of self\sphinxhyphen{}employment.

\sphinxAtStartPar
Revenue Control: The ability to manage pricing, accept multiple contracts, and influence earnings supports classification under a contract for service.

\sphinxAtStartPar
Collectively, these factors guide proper categorization for legal and tax purposes, helping organizations ensure compliance and mitigate potential risk.


\section{Review Questions}
\label{\detokenize{compliance:review-questions}}
\sphinxAtStartPar
What is the primary objective of the payroll department?
\begin{quote}

\sphinxAtStartPar
The primary objective of the payroll department is to pay employees accurately and
on time, in compliance with the legislative requirements for a full annual payroll
cycle.
\end{quote}

\sphinxAtStartPar
List four definitions of payroll.
\begin{itemize}
\item {} 
\sphinxAtStartPar
the department that administers the payroll

\item {} 
\sphinxAtStartPar
the total number of people employed by an organization

\item {} 
\sphinxAtStartPar
the wages and salaries paid out in a year

\item {} 
\sphinxAtStartPar
a list of employees to be paid and the amount due to each

\end{itemize}

\sphinxAtStartPar
List the three types of payroll management stakeholders and provide an example of each.
\begin{quote}

\sphinxAtStartPar
Payroll management stakeholders are government (federal and provincial/territorial), internal
(employees, employers and other departments) and external (benefit carriers, courts, unions, pension
providers, charities, third party administrators and outsource/software vendors).
\end{quote}

\sphinxAtStartPar
Explain the difference between legislation and regulation.
\begin{quote}

\sphinxAtStartPar
Legislation determines what the rules are, while regulations determine how the rules are to be applied.
\end{quote}

\sphinxAtStartPar
What are two examples of sources of information that you use (or could use) to keep upto\sphinxhyphen{}date on payroll compliance changes?
\begin{quote}

\sphinxAtStartPar
The Canadian Payroll Association offers Payroll InfoLine, a phone\sphinxhyphen{}in and e\sphinxhyphen{}mail information service for members
\begin{quote}
\begin{itemize}
\item {} 
\sphinxAtStartPar
The Canada Revenue Agency (CRA) produces guides, publications and Income Tax Bulletins, folios and Circulars, posts news bulletins and enables

\end{itemize}

\sphinxAtStartPar
participation on an electronic mailing list with e\sphinxhyphen{}mail alerts for new content to the site
\sphinxhyphen{} The Revenu Québec (RQ) website provides guides, publications, bulletins, forms, online services and enables participation on an electronic mailing list with e\sphinxhyphen{}mail notifications of tax news articles
\sphinxhyphen{} Employment/labour standards (federal, provincial and territorial) publications and websites
\sphinxhyphen{} Employment and Social Development Canada (ESDC) and Service Canada (SC) publications including information regarding the Employment Insurance (EI) program and the Social Insurance Number
\sphinxhyphen{} CCH Canada Limited publishes a series of volumes on employment and labour law, pensions and benefits, etc., that supplies information on legislation with regular updates as changes become law
\sphinxhyphen{} Carswell publishes The Canadian Payroll Manual and offers a phone\sphinxhyphen{}in service to subscribers
\end{quote}

\sphinxAtStartPar
Copies of legislation are available from the printing offices of the federal, provincial and territorial governments as well as through government websites.
\end{quote}

\sphinxAtStartPar
List three external stakeholders and explain their compliance requirements.
\begin{quote}

\sphinxAtStartPar
Benefit Carriers \sphinxhyphen{} Payroll is responsible for deducting and remitting premiums for the insurance coverage to the carriers and for providing reports on employee enrolment and coverage levels.
Courts and the CRA \sphinxhyphen{} Payroll must accurately deduct and remit amounts ordered to be withheld through garnishments, third party demands, requirements to pay and support deduction orders.
Unions \sphinxhyphen{} Payroll must accurately deduct and remit union dues and initiation fees, and ensure that the terms of the collective agreement are adhered to.
Pension Providers \sphinxhyphen{} Third party pension plan providers may require payroll to provide enrolment reports on participating employees and length of service calculations, and to remit employee deductions and employer contributions.
\end{quote}

\sphinxAtStartPar
Indicate the jurisdiction the following employees fall under:
\begin{itemize}
\item {} 
\sphinxAtStartPar
Canada Post Corporation (F)

\item {} 
\sphinxAtStartPar
An insurance company (P)

\item {} 
\sphinxAtStartPar
A uranium mining company (F)

\item {} 
\sphinxAtStartPar
Canadian Broadcasting Corporation (F)

\item {} 
\sphinxAtStartPar
A retail department store with locations in every province (P)

\item {} 
\sphinxAtStartPar
A chartered bank (F)

\end{itemize}

\sphinxAtStartPar
What is the difference between a contract of service and a contract for service?
\begin{quote}

\sphinxAtStartPar
A contract of service is an arrangement whereby an individual (the employee) agrees to work on a full\sphinxhyphen{}time or part\sphinxhyphen{}time basis for an employer for a specified or indeterminate period of time.

\sphinxAtStartPar
A contract for service is a business relationship whereby one party agrees to perform certain specific work stipulated in the contract for another party.
\end{quote}

\sphinxAtStartPar
What are the factors that the Canada Revenue Agency (CRA) considers when
determining if a contract of service or a contract for service exists?

\sphinxAtStartPar
Please consider the following scenario.
\begin{quote}

\sphinxAtStartPar
You are a payroll professional working for a large manufacturing company. Your
organization has had many change initiatives over the last number of years including
three mergers and two large group terminations. Your company endorses the use of
consultants rather than growing the number of permanent employees.

\sphinxAtStartPar
Write a memo to your supervisor, who is the Chief Financial Officer of the company, to
explain why your role must coordinate with the Accounts Payable Department to ensure
that these payments are being handled correctly. Please prepare your answer in a separate
document.

\sphinxAtStartPar
\sphinxstyleemphasis{At the last weekly Finance meeting, Tom and I discussed the increase in the number of contractor invoices being
processed through accounts payable (AP). We have some concerns as to whether these individuals would be considered truly
selfemployed by the Canada Revenue Agency (CRA), or whether the CRA would determine them to be employees.}

\sphinxAtStartPar
\sphinxstyleemphasis{If the worker is considered self\sphinxhyphen{}employed, then payment, on submission of an invoice, will continue to be handled by AP. If, however, the worker is considered an
employee, they would have to be set up on payroll, as they would be in receipt of income from employment, subject to all legislated statutory withholdings.}

\sphinxAtStartPar
\sphinxstyleemphasis{I have attached the CRA’s form Request for a CPP/EI Ruling \sphinxhyphen{} Employee or SelfEmployed? \sphinxhyphen{} CPT1 for your information. This form can be completed by the
company and sent with supporting documentation, such as the terms and conditions of the contract, for a ruling from the CRA on the individual’s status.}

\sphinxAtStartPar
\sphinxstyleemphasis{I think that Payroll must coordinate with the Accounts Payable Department to ensure that these payments are being handled correctly.}

\sphinxAtStartPar
\sphinxstyleemphasis{Tom and I would be pleased to meet with you to ensure the company is in
compliance with all legislative requirements. Would you be available next Friday
morning at 10:00 to discuss?}
\end{quote}

\sphinxstepscope

\sphinxAtStartPar
Membership or participation in the Canada Pension Plan (CPP) and Employment Insurance
Plan (EI) is compulsory for certain types of employment. As a person responsible for the payroll
you need to know which employees must participate in these plans, what amounts to withhold
from employees and how much the employer will have to remit or send to the Canada
Revenue Agency (CRA).

\sphinxAtStartPar
Payroll plays a pivotal role in administering statutory deductions, specifically the collection and remittance of
Canada Pension Plan (CPP) contributions and Employment Insurance (EI) premiums. These mandatory deductions, along with the
employer’s matching portion, must be accurately submitted to the Canada Revenue Agency (CRA) within prescribed timelines.

\sphinxAtStartPar
This chapter outlines the essential criteria used to identify pensionable and insurable earnings, and provides detailed
guidance on calculating both employee deductions and employer contributions for regular and non\sphinxhyphen{}regular pay periods.

\sphinxAtStartPar
In accordance with federal legislation, CPP contributions are the \sphinxstylestrong{first deduction} applied to employment income,
followed by EI premiums. Because these deductions are mandated by statute, they are classified as \sphinxstylestrong{statutory deductions},
underscoring their legal significance and the employer’s obligation to ensure full compliance.

\sphinxAtStartPar
\sphinxstylestrong{Learning Objectives}

\sphinxAtStartPar
Upon completion of this chapter, you should be able to explain:
\begin{enumerate}
\sphinxsetlistlabels{\arabic}{enumi}{enumii}{}{.}%
\item {} 
\sphinxAtStartPar
The requirements and calculations for Canada Pension Plan contributions

\item {} 
\sphinxAtStartPar
The requirements and calculations for Employment Insurance premiums

\item {} 
\sphinxAtStartPar
What Service Canada uses the information on a Record of Employment for

\end{enumerate}

\sphinxAtStartPar
This chapter will cover the following topics:
\begin{quote}
\begin{enumerate}
\sphinxsetlistlabels{\arabic}{enumi}{enumii}{}{.}%
\item {} 
\sphinxAtStartPar
Identify the following Canada Pension Plan components:

\end{enumerate}
\begin{itemize}
\item {} 
\sphinxAtStartPar
Who must contribute to the Canada Pension Plan

\item {} 
\sphinxAtStartPar
Types of employment subject to Canada Pension Plan contributions

\item {} 
\sphinxAtStartPar
Types of employment not subject to Canada Pension Plan contributions

\item {} 
\sphinxAtStartPar
Payments and benefits subject to Canada Pension Plan contributions

\item {} 
\sphinxAtStartPar
Payments and benefits not subject to Canada Pension Plan contributions

\end{itemize}
\begin{enumerate}
\sphinxsetlistlabels{\arabic}{enumi}{enumii}{}{.}%
\setcounter{enumi}{1}
\item {} 
\sphinxAtStartPar
Calculate Canada Pension Plan contributions at an individual level

\item {} 
\sphinxAtStartPar
Identify the following Employment Insurance components:

\end{enumerate}
\begin{itemize}
\item {} 
\sphinxAtStartPar
Who must pay Employment Insurance premiums

\item {} 
\sphinxAtStartPar
Types of employment subject to Employment Insurance premiums

\item {} 
\sphinxAtStartPar
Types of employment not subject to Employment Insurance premiums

\item {} 
\sphinxAtStartPar
Payments and benefits subject to Employment Insurance premiums

\item {} 
\sphinxAtStartPar
Payments and benefits not subject to Employment Insurance premiums

\end{itemize}
\begin{enumerate}
\sphinxsetlistlabels{\arabic}{enumi}{enumii}{}{.}%
\setcounter{enumi}{3}
\item {} 
\sphinxAtStartPar
Calculate Employment Insurance premiums at an individual level

\item {} 
\sphinxAtStartPar
Describe the purpose of a Record of Employment

\item {} 
\sphinxAtStartPar
Identify when the Record of Employment must be completed

\end{enumerate}
\end{quote}


\chapter{Canada Pension Plan}
\label{\detokenize{cpp-and-ei:canada-pension-plan}}\label{\detokenize{cpp-and-ei::doc}}
\sphinxAtStartPar
Objective of this section is to enable you to identify the following Canada Pension Plan components:
\begin{itemize}
\item {} 
\sphinxAtStartPar
Who must contribute to the Canada Pension Plan

\item {} 
\sphinxAtStartPar
Types of employment subject to Canada Pension Plan contributions

\item {} 
\sphinxAtStartPar
Types of employment not subject to Canada Pension Plan contributions

\item {} 
\sphinxAtStartPar
Payments and benefits subject to Canada Pension Plan contributions

\item {} 
\sphinxAtStartPar
Payments and benefits not subject to Canada Pension Plan contributions

\item {} 
\sphinxAtStartPar
Calculate Canada Pension Plan contributions at an individual level

\end{itemize}

\sphinxAtStartPar
The \sphinxstylestrong{Canada Pension Plan} (CPP) is a federally legislated social insurance program established under the Canada Pension Plan
Act. Its primary purpose is to provide financial protection to contributors and their families in the event of retirement,
disability, or death. The program is funded through mandatory payroll deductions from employees, which are matched equally by
employers. These employee contributions are specifically referred to as Canada Pension Plan contributions.

\sphinxAtStartPar
In addition to the CPP, employers may offer private or non\sphinxhyphen{}government pension plans, which may also involve payroll
deductions from employees. The specific payroll withholding requirements for these supplementary pension plans will be discussed
in more detail in the later chapters; it is important to note that the CPP is often one of multiple retirement savings
vehicles available within an organization’s compensation structure.

\begin{sphinxadmonition}{note}{Note:}
\sphinxAtStartPar
Employers located in Quebec are responsible for deducting Québec Pension Plan (QPP) contributions, instead
of CPP contributions, from their Québec employees and remitting those contributions to Revenu Québec (RQ).
\end{sphinxadmonition}

\sphinxAtStartPar
The Canada Pension Plan (CPP) was designed as an income replacement program for individuals who have been in pensionable
employment during their working life. A CPP retirement pension is a monthly benefit paid to people who have contributed to
the Canada Pension Plan. The pension is designed to replace about 25 percent of the earnings on which a person’s contributions
were based. Individuals can apply for their CPP retirement pension when they turn 60.

\sphinxAtStartPar
There are three Canada Pension Plan benefits:
\begin{itemize}
\item {} 
\sphinxAtStartPar
retirement pension

\item {} 
\sphinxAtStartPar
disability benefits (for contributors with a disability and their dependent children)

\item {} 
\sphinxAtStartPar
survivor benefits (including the death benefit, the survivor’s pension and the children’s benefit)

\end{itemize}

\sphinxAtStartPar
The CPP operates throughout Canada while the province of Québec administers its own program for workers in Québec called the
\sphinxstylestrong{Québec Pension Plan} (QPP). The two plans work together to ensure that all contributors are protected, no matter where the
individual lives. Québec Pension Plan requirements will be covered later in this course.


\section{Who Must Contribute to the Canada Pension Plan}
\label{\detokenize{cpp-and-ei:who-must-contribute-to-the-canada-pension-plan}}
\sphinxAtStartPar
The CPP is a \sphinxstylestrong{contributory plan}. This means that all costs are covered by the financial contributions paid by employees,
employers and self\sphinxhyphen{}employed workers, and from revenue earned on CPP investments. The CPP is not funded through general tax
revenues.

\sphinxAtStartPar
Canada Pension Plan contributions must be withheld from employees who:
\begin{quote}
\begin{enumerate}
\sphinxsetlistlabels{\arabic}{enumi}{enumii}{}{.}%
\item {} 
\sphinxAtStartPar
CPP contributions must be withheld from employees who have reached the age of 18 but are under the age of 70.

\item {} 
\sphinxAtStartPar
CPP contributions must be withheld from employees who are in pensionable employment.

\item {} 
\sphinxAtStartPar
CPP contributions must be withheld from employees in pensionable employment who are not considered to be disabled by either Service Canada or Retraite Québec.

\end{enumerate}

\sphinxAtStartPar
4. CPP contributions must be withheld from employees who are 65 years of age but are under the age of 70 and are in receipt of the Canada or Québec Pension Plan retirement
pension, but have not filed an election to stop paying CPP contributions.
\end{quote}

\sphinxAtStartPar
In principle, employees who do not fall within the categories listed previously would not make CPP contributions. However, it
is not always clear what constitutes pensionable earnings and pensionable employment. To clarify eligibility, the CRA has
developed a list of the types of employment that are not subject to CPP contributions. This information can also be found in
the Employers’ Guide \sphinxhyphen{} Payroll Deductions and Remittances \sphinxhyphen{} T4001, which is published by the CRA.

\sphinxAtStartPar
The following types of employment are excluded by legislation and therefore do not constitute pensionable employment. Payments arising from such employment are not subject
to CPP contributions:
\begin{itemize}
\item {} 
\sphinxAtStartPar
employment in agriculture, or an agricultural enterprise, horticulture, fishing, hunting, trapping, forestry, logging, or lumbering, by an employer:
\begin{itemize}
\item {} 
\sphinxAtStartPar
who pays the employee less than \$250 in cash remuneration in a calendar year; or

\item {} 
\sphinxAtStartPar
employs the employee for a period of less than 25 working days in the same year on terms providing for payment of cash remuneration—the working days do not have to be consecutive

\end{itemize}

\item {} 
\sphinxAtStartPar
employment of a casual nature other than for the purpose of the employer’s usual trade or business

\item {} 
\sphinxAtStartPar
employment of a person, other than as an entertainer, in connection with a circus, fair, parade, carnival, exposition, exhibition, or other similar activity, if that person is:
\begin{itemize}
\item {} 
\sphinxAtStartPar
not regularly employed by that employer, and

\item {} 
\sphinxAtStartPar
employed by that employer for less than seven days in a year

\end{itemize}

\item {} 
\sphinxAtStartPar
employment of a person by a government body as an election worker, if that person:
\begin{itemize}
\item {} 
\sphinxAtStartPar
is not a regular employee of the government body, and

\item {} 
\sphinxAtStartPar
works for less than 35 hours in a calendar year

\end{itemize}

\item {} 
\sphinxAtStartPar
employment as a teacher on exchange from a foreign country

\item {} 
\sphinxAtStartPar
employment of a spouse or common\sphinxhyphen{}law partner if the employer cannot deduct the remuneration paid as an expense under the Income Tax Act

\item {} 
\sphinxAtStartPar
employment of a member of a religious order who has taken a vow of perpetual poverty. This applies whether the remuneration is paid directly to the order or paid by the member to the order.

\item {} 
\sphinxAtStartPar
employment for which no cash remuneration is paid, where the employee is the child of, or is maintained by, the employer

\item {} 
\sphinxAtStartPar
employment of a person who helps the employer in a disaster or in a rescue operation if the employee is not regularly employed by the employer

\end{itemize}


\chapter{Employment Insurance}
\label{\detokenize{cpp-and-ei:employment-insurance}}
\sphinxAtStartPar
Objective of this section is to enable you to identify the following Employment Insurance components:
\begin{itemize}
\item {} 
\sphinxAtStartPar
Who must pay Employment Insurance premiums

\item {} 
\sphinxAtStartPar
Types of employment subject to Employment Insurance premiums

\item {} 
\sphinxAtStartPar
Types of employment not subject to Employment Insurance premiums

\item {} 
\sphinxAtStartPar
Payments and benefits subject to Employment Insurance premiums

\item {} 
\sphinxAtStartPar
Payments and benefits not subject to Employment Insurance premiums

\item {} 
\sphinxAtStartPar
Calculate Employment Insurance premiums at an individual level

\end{itemize}

\sphinxAtStartPar
\sphinxstylestrong{Employment Insurance} (EI) is a federally legislated social insurance program established under the Employment Insurance Act.
It provides temporary financial support to individuals who are unemployed while seeking new employment or engaging in skill
development. In addition to regular benefits, EI offers special provisions for workers who take leave due to significant life
events such as illness, pregnancy, caring for a newborn or newly adopted child, supporting a critically ill or injured person,
or tending to a family member facing a serious health condition with a risk of death.

\sphinxAtStartPar
The EI program is funded through payroll contributions made by employees, known as Employment Insurance premiums. Employers
also contribute by paying a premium that is calculated based on their employees’ deductions.

\sphinxAtStartPar
While Employment Insurance is a government\sphinxhyphen{}mandated program, it may not be the only insurance plan available in the workplace.
Many organizations offer private or non\sphinxhyphen{}government insurance options such as life and disability coverage, which are funded by
employers, employees, or both. Although this chapter focuses specifically on the federally legislated EI program, additional
information about private insurance plans will be covered in the later chapters.


\chapter{Record of Employment}
\label{\detokenize{cpp-and-ei:record-of-employment}}
\sphinxAtStartPar
The \sphinxstylestrong{Record of Employment} (ROE) is the form used by Service Canada to determine an individual’s qualification to collect
Employment Insurance benefits when their employment is interrupted, how much the benefit will be and how long they will
collect it. As payroll is responsible for completing the ROE, the form will be illustrated in this chapter, along with an
explanation of what payroll information must be tracked for ROE reporting purposes.

\sphinxstepscope


\chapter{DETERMINING ANNUAL AND PAY PERIOD EARNINGS}
\label{\detokenize{compensation:determining-annual-and-pay-period-earnings}}\label{\detokenize{compensation::doc}}

\section{Employment Income}
\label{\detokenize{compensation:employment-income}}

\section{Allowances}
\label{\detokenize{compensation:allowances}}

\section{Expenses}
\label{\detokenize{compensation:expenses}}

\section{Benefits}
\label{\detokenize{compensation:benefits}}
\sphinxstepscope


\chapter{OBNOARDING EMPLOYEE}
\label{\detokenize{onboarding_employee:obnoarding-employee}}\label{\detokenize{onboarding_employee::doc}}
\sphinxAtStartPar
In the context of Canadian payroll administration, onboarding an employee refers to the formal process of integrating a new hire into
both the organizational and payroll systems. It ensures that the employee is properly registered, legally compliant, and ready to be paid
accurately and on time.

\sphinxAtStartPar
Key Steps on Onboarding an Employee:
\begin{itemize}
\item {} 
\sphinxAtStartPar
Collect Required Personal Information: Includes full legal name, address, date of birth, and Social Insurance Number (SIN). The SIN is

\end{itemize}

\sphinxAtStartPar
critical for tax reporting to the CRA (Canada Revenue Agency).
\begin{itemize}
\item {} 
\sphinxAtStartPar
Obtain Federal \& Provincial Tax Forms: New employees must complete Form TD1 (Federal and possibly a Provincial version) to declare tax

\end{itemize}

\sphinxAtStartPar
credits and determine income tax withholdings.
\begin{itemize}
\item {} 
\sphinxAtStartPar
Set Up Banking Info for Direct Deposit: Employees usually provide a void cheque or bank form to set up electronic payments.

\item {} 
\sphinxAtStartPar
Register the Employee in the Payroll System: Involves entering all personal and job\sphinxhyphen{}related data, assigning a payroll ID, and verifying

\end{itemize}

\sphinxAtStartPar
employment status (e.g. full\sphinxhyphen{}time, part\sphinxhyphen{}time, contract).
\begin{itemize}
\item {} 
\sphinxAtStartPar
Enroll in Benefits or Pension Programs: If applicable, the employee may be signed up for group insurance, retirement savings plans (like

\end{itemize}

\sphinxAtStartPar
RRSP or pension plans), and other benefits. These deductions must be accurately reflected in payroll.
\begin{itemize}
\item {} 
\sphinxAtStartPar
Assign Statutory Deductions that Employers must withhold and remit
\begin{itemize}
\item {} 
\sphinxAtStartPar
CPP (Canada Pension Plan)

\item {} 
\sphinxAtStartPar
EI (Employment Insurance)

\item {} 
\sphinxAtStartPar
Income Tax (based on TD1 form)

\end{itemize}

\item {} 
\sphinxAtStartPar
Confirm Employment Agreement \& Start Date

\end{itemize}

\sphinxAtStartPar
Compliance \& Record Keeping
\begin{itemize}
\item {} 
\sphinxAtStartPar
Employers in Canada are responsible for keeping accurate records of employee data, pay stubs, deductions, and remittances for

\end{itemize}

\sphinxAtStartPar
at least 6 years.
\begin{itemize}
\item {} 
\sphinxAtStartPar
If audited by CRA, these documents must be readily available.

\item {} 
\sphinxAtStartPar
Employers must also provide T4 slips by end of February each year to summarize annual earnings and deductions for tax filing.

\end{itemize}
\begin{quote}

\sphinxAtStartPar
Employment Standards Requirements
\end{quote}


\bigskip\hrule\bigskip


\sphinxAtStartPar
Each province/territory, as well as the federal government, sets minimum employment standards, including:
\begin{itemize}
\item {} 
\sphinxAtStartPar
Minimum wage

\item {} 
\sphinxAtStartPar
Minimum age (may also be governed by other legislation)

\item {} 
\sphinxAtStartPar
Required pay statement information:
\sphinxhyphen{} Employee name
\sphinxhyphen{} Pay period date
\sphinxhyphen{} Rates of pay and hours worked
\sphinxhyphen{} Gross earnings
\sphinxhyphen{} Itemized deductions
\sphinxhyphen{} Net pay

\end{itemize}


\section{Internal Forms}
\label{\detokenize{onboarding_employee:internal-forms}}
\sphinxAtStartPar
Typical commencement package forms include:
\begin{itemize}
\item {} 
\sphinxAtStartPar
Authorization for hiring

\item {} 
\sphinxAtStartPar
Direct deposit agreement

\item {} 
\sphinxAtStartPar
Union membership application

\item {} 
\sphinxAtStartPar
Benefits enrollment (e.g., health/dental, pension)

\item {} 
\sphinxAtStartPar
Confidentiality agreement

\end{itemize}


\subsection{Authorization for Hiring}
\label{\detokenize{onboarding_employee:authorization-for-hiring}}
\sphinxAtStartPar
This internal document includes:
\begin{itemize}
\item {} 
\sphinxAtStartPar
New employee’s basic info

\item {} 
\sphinxAtStartPar
Start date, department, salary

\item {} 
\sphinxAtStartPar
Probation details

\item {} 
\sphinxAtStartPar
Hiring authority’s signature

\end{itemize}

\sphinxAtStartPar
\sphinxstylestrong{Important:} Employer must obtain a valid SIN. A SIN starting with 9 must have a valid expiry date and associated work permit.


\subsection{Union Membership}
\label{\detokenize{onboarding_employee:union-membership}}
\sphinxAtStartPar
For unionized workplaces:
\begin{itemize}
\item {} 
\sphinxAtStartPar
Amount of union dues to be deducted

\item {} 
\sphinxAtStartPar
Employees signature authorization for deduction

\item {} 
\sphinxAtStartPar
Exemptions may apply, but dues equivalent still required

\end{itemize}


\subsection{Benefit Enrollment Forms}
\label{\detokenize{onboarding_employee:benefit-enrollment-forms}}
\sphinxAtStartPar
Forms cover group insurance and pension plans:
\begin{itemize}
\item {} 
\sphinxAtStartPar
Employee indicates coverage type

\item {} 
\sphinxAtStartPar
Signatures authorize payroll deductions

\end{itemize}


\subsection{Confidentiality Agreement}
\label{\detokenize{onboarding_employee:confidentiality-agreement}}
\sphinxAtStartPar
A legally binding agreement protecting sensitive company info:
\begin{itemize}
\item {} 
\sphinxAtStartPar
Defines proprietary data

\item {} 
\sphinxAtStartPar
Outlines responsibilities, penalties, and timeframe

\end{itemize}


\section{Required Federal and Provincial/Territorial Forms}
\label{\detokenize{onboarding_employee:required-federal-and-provincial-territorial-forms}}
\sphinxAtStartPar
\sphinxstylestrong{Purpose:} Determine correct income tax withholdings.

\sphinxAtStartPar
Forms:
\begin{itemize}
\item {} 
\sphinxAtStartPar
TD1 (Federal)

\item {} 
\sphinxAtStartPar
TD1 (Provincial/Territorial)

\item {} 
\sphinxAtStartPar
TP\sphinxhyphen{}1015.3\sphinxhyphen{}V (Québec employees)

\end{itemize}

\sphinxAtStartPar
\sphinxstylestrong{Provincial/territorial withholding} is based on \sphinxstyleemphasis{province of employment}, but tax liability is based on \sphinxstyleemphasis{province of residence}.

\sphinxAtStartPar
\sphinxstylestrong{Adjustments:}
\begin{itemize}
\item {} 
\sphinxAtStartPar
Request extra withholding via TD1 or TP\sphinxhyphen{}1015.3\sphinxhyphen{}V

\item {} 
\sphinxAtStartPar
Request reduction using CRA Form T1213 or RQ Form TP\sphinxhyphen{}1016\sphinxhyphen{}V

\end{itemize}

\sphinxAtStartPar
Essential Info on All Forms:
\begin{itemize}
\item {} 
\sphinxAtStartPar
Employee name

\item {} 
\sphinxAtStartPar
Date of birth

\item {} 
\sphinxAtStartPar
Social Insurance Number

\end{itemize}


\subsection{Tax Credits (TD1)}
\label{\detokenize{onboarding_employee:tax-credits-td1}}\begin{enumerate}
\sphinxsetlistlabels{\arabic}{enumi}{enumii}{}{.}%
\item {} 
\sphinxAtStartPar
Basic personal amount

\item {} 
\sphinxAtStartPar
Canada caregiver (infirm children)

\item {} 
\sphinxAtStartPar
Age amount

\item {} 
\sphinxAtStartPar
Pension income

\item {} 
\sphinxAtStartPar
Tuition

\item {} 
\sphinxAtStartPar
Disability

\item {} 
\sphinxAtStartPar
Spouse/common\sphinxhyphen{}law partner amount

\item {} 
\sphinxAtStartPar
Eligible dependant

\item {} 
\sphinxAtStartPar
Caregiver for infirm spouse or dependant

\item {} 
\sphinxAtStartPar
Caregiver for dependant age 18+

\item {} 
\sphinxAtStartPar
Transfers from spouse

\item {} 
\sphinxAtStartPar
Transfers from dependant

\item {} 
\sphinxAtStartPar
Total

\end{enumerate}

\sphinxAtStartPar
Additional Instructions:
\begin{itemize}
\item {} 
\sphinxAtStartPar
Fill out TD1 only if claiming more than basic credit

\item {} 
\sphinxAtStartPar
Québec employees must always complete TP\sphinxhyphen{}1015.3\sphinxhyphen{}V

\end{itemize}


\subsection{Tax Credits (TP\sphinxhyphen{}1015.3\sphinxhyphen{}V \sphinxhyphen{} Québec)}
\label{\detokenize{onboarding_employee:tax-credits-tp-1015-3-v-quebec}}\begin{itemize}
\item {} 
\sphinxAtStartPar
Basic amount

\item {} 
\sphinxAtStartPar
Transfer from spouse

\item {} 
\sphinxAtStartPar
Amount for dependants

\item {} 
\sphinxAtStartPar
Impairment in mental/physical function

\item {} 
\sphinxAtStartPar
Age amount, retirement income, living alone

\item {} 
\sphinxAtStartPar
Career extension

\end{itemize}

\sphinxAtStartPar
Deductions:
\begin{itemize}
\item {} 
\sphinxAtStartPar
Remote area housing

\item {} 
\sphinxAtStartPar
Deductible support payments

\end{itemize}


\section{Entering Employee Information into Sage50}
\label{\detokenize{onboarding_employee:entering-employee-information-into-sage50}}
\sphinxAtStartPar
To enter a new employee into the Sage 50 Payroll module (Canada edition), start by navigating to the Employees \& Payroll section in the
Home window. Right\sphinxhyphen{}click the Employees icon and choose “Add Employee” to begin creating employee’s record. Input the employee’s full legal
name. Then, proceed to fill in the personal and payroll details across several tabs: the Personal tab for birth date and contact info, the Taxes tab to select the appropriate provincial
tax table, the Income tab to configure their pay frequency, and the Deductions tab to define benefit or pension deductions. You’ll also
want to enter their bank details for direct deposit. For compliance, be sure to complete and store signed TD1 forms (Federal and Provincial)
separately, as Sage50 does not automatically generate these. You’ll also need to set up EI, CPP, and Income Tax deductions and link them to remittance vendors in the system. Once all
information is reviewed for accuracy, save and close the record to finalize setup. If you prefer a guided approach, Sage50 also offers
an Employee Wizard to walk you through these steps.

\begin{sphinxadmonition}{important}{Important:}
\sphinxAtStartPar
To maintain accuracy and compliance in Sage 50 Payroll, carefully verify that all employee information entered into the system,
including full legal name, Social Insurance Number (SIN), residential address, and compensation details, matches the data provided on
official documentation such as the signed employment contract and government\sphinxhyphen{}issued identification (e.g. driver’s licence, Employment Contract).
Double\sphinxhyphen{}checking these entries helps prevent administrative errors and ensures that payroll records remain consistent with legal and
regulatory standards.
\end{sphinxadmonition}

\noindent\sphinxincludegraphics{{onboarding-employee_001}.png}

\noindent\sphinxincludegraphics{{onboarding-employee_002}.png}

\noindent\sphinxincludegraphics{{onboarding-employee_003}.png}

\noindent\sphinxincludegraphics{{onboarding-employee_004}.png}

\noindent\sphinxincludegraphics{{onboarding-employee_005}.png}

\noindent\sphinxincludegraphics{{onboarding-employee_006}.png}


\subsection{Review Questions}
\label{\detokenize{onboarding_employee:review-questions}}\begin{enumerate}
\sphinxsetlistlabels{\arabic}{enumi}{enumii}{}{.}%
\item {} 
\sphinxAtStartPar
What is the significance of accurately entering the “Date Hired” field when setting up a new employee profile in Sage 50?

\sphinxAtStartPar
\sphinxstyleemphasis{Accurately entering the “Date Hired” in Sage 50 is a critical step in ensuring the integrity of payroll records and overall HR
compliance. This field defines the employee’s official start date, which determines pay cycle alignment, benefit entitlement periods,
and the correct application of mandatory deductions such as CPP and EI. It also plays a pivotal role in historical payroll
reporting—including audit readiness and the generation of year\sphinxhyphen{}end T4 slips. Furthermore, the hire date is essential when preparing a
Record of Employment (ROE), as it establishes the starting point for the employee’s insurable earnings and service duration.}

\end{enumerate}

\sphinxAtStartPar
2. Within the scope of Payroll Administration, how should the department ethically and legally respond when a supervisor requests access
to an employee’s date of birth for the purpose of workplace recognition, given that this personal information is already held by payroll?
\begin{quote}

\sphinxAtStartPar
\sphinxstyleemphasis{Under Canadian payroll administration and the Personal Information Protection and Electronic Documents Act (PIPEDA), sharing an
employee’s date of birth — even for positive intentions like workplace recognition, is not legally or ethically appropriate.}

\sphinxAtStartPar
\sphinxstyleemphasis{PIPEDA requires employers to:}
\begin{itemize}
\item {} 
\sphinxAtStartPar
Limit the collection, use, and disclosure of personal information to what is necessary for clearly identified business purposes.

\item {} 
\sphinxAtStartPar
Obtain meaningful consent before using personal data for any purpose beyond what it was originally collected for—such as payroll or benefits administration.

\item {} 
\sphinxAtStartPar
Protect employee privacy by restricting access to personal information on a strict need\sphinxhyphen{}to\sphinxhyphen{}know basis.

\end{itemize}

\sphinxAtStartPar
\sphinxstyleemphasis{In this case, using the date of birth for celebrations or acknowledgments is outside the scope of payroll processing. Even if the
Payroll department holds this information, it cannot be disclosed to supervisors or other staff.}
\end{quote}


\section{Content Review Highlights}
\label{\detokenize{onboarding_employee:content-review-highlights}}\begin{itemize}
\item {} 
\sphinxAtStartPar
Consent is required for personal info collection

\item {} 
\sphinxAtStartPar
TD1 and TP\sphinxhyphen{}1015.3\sphinxhyphen{}V are used to calculate source deductions

\item {} 
\sphinxAtStartPar
Claim amounts may differ between federal and provincial forms

\item {} 
\sphinxAtStartPar
Employers must keep the forms on file (do not send to CRA/RQ)

\end{itemize}


\section{Review Questions (Sample)}
\label{\detokenize{onboarding_employee:review-questions-sample}}\begin{enumerate}
\sphinxsetlistlabels{\arabic}{enumi}{enumii}{}{.}%
\item {} 
\sphinxAtStartPar
What does an offer letter signature signify?

\item {} 
\sphinxAtStartPar
What documents are included in a commencement package?

\item {} 
\sphinxAtStartPar
Name three common internal forms

\item {} 
\sphinxAtStartPar
What must payroll verify on a hiring form?

\item {} 
\sphinxAtStartPar
What must be checked for SINs starting with “9”?

\item {} 
\sphinxAtStartPar
True/False: Union dues can be deducted without consent.

\item {} 
\sphinxAtStartPar
What authorizes benefit premium deductions?

\end{enumerate}


\section{Example Evaluations}
\label{\detokenize{onboarding_employee:example-evaluations}}
\sphinxAtStartPar
\sphinxstylestrong{Gloria Meyer (Alberta):}
\sphinxhyphen{} Claimed: Basic, eligible dependant, transferred tuition
\sphinxhyphen{} Appears accurate

\sphinxAtStartPar
\sphinxstylestrong{Luc Laframboise (Québec):}
\sphinxhyphen{} Claimed: Basic, spouse, dependant in school, tuition transfer
\sphinxhyphen{} Appropriate provincial and federal claims made

\sphinxAtStartPar
\sphinxstylestrong{Ingrid Johansson (Alberta, Single Parent):}
\sphinxhyphen{} Claimed credits for two children
\sphinxhyphen{} \sphinxstylestrong{Overclaimed} dependant credit \textendash{} only one is eligible
\sphinxhyphen{} Needs correction on federal and AB TD1 forms

\begin{sphinxadmonition}{note}{ONBOARDING EMPLOYEE EXERCISE}

\sphinxAtStartPar
Using MS Forms, create a questionaire for gathering all required information for onboarding a new employee at Quebec\sphinxhyphen{}based company for the payroll purposes.
\end{sphinxadmonition}

\sphinxstepscope


\chapter{Payroll Accounting}
\label{\detokenize{payroll_accounting:payroll-accounting}}\label{\detokenize{payroll_accounting::doc}}

\section{Journal Entries}
\label{\detokenize{payroll_accounting:journal-entries}}

\subsection{Accounting Recap}
\label{\detokenize{payroll_accounting:accounting-recap}}
\begin{center}\(\Sigma \text{ Total Debits} = \Sigma \text{ Total Credits}\)
\end{center}
\begin{center}\(\text{Assets} = \text{Liabilities} + \text{Equity}\)
\end{center}\begin{equation}\label{equation:payroll_accounting:AccountingEquation}
\begin{split}Assets = Liabilities + Equity\end{split}
\end{equation}
\sphinxAtStartPar
Furthermore, we know that:

\begin{center}\(\text{Equity = Revenue - Expenses}\)
, which leads us to:
\end{center}
\begin{center}\(\text{Assets = Liabilities + (Revenues - Expenses)}\)
\end{center}
\sphinxAtStartPar
Accounting equation \eqref{equation:payroll_accounting:AccountingEquation}

\sphinxAtStartPar
Payroll accounting is a critical component of the Canadian Payroll Administration system. It involves the systematic recording, analysis, and reporting of payroll transactions to ensure that all financial aspects of employee compensation are accurately reflected in the organization’s financial statements.
Payroll accounting includes the management of employee wages, tax withholdings, benefit deductions, and other payroll\sphinxhyphen{}related expenses. The system is designed to automate these processes, ensuring accuracy and compliance with Canadian payroll regulations.


\subsection{Journal Entries}
\label{\detokenize{payroll_accounting:id1}}
\sphinxAtStartPar
Journal entries are a key part of payroll accounting, as they document the financial impact of payroll transactions on the organization’s accounts. Each payroll run generates a series of journal entries that reflect the distribution of wages, taxes, and deductions across various accounts.
These entries are essential for maintaining accurate financial records and ensuring that the organization’s financial statements reflect the true cost of employee compensation. The Canadian Payroll Administration system automates the generation of these journal entries, reducing the risk of errors and ensuring compliance with accounting standards.

\begin{DUlineblock}{0em}
\item[] DR    Payroll Expenses    \$10,500.00
\item[]
\begin{DUlineblock}{\DUlineblockindent}
\item[] CR  Payroll Payable   \$10,500.00
\end{DUlineblock}
\end{DUlineblock}

\sphinxstepscope


\chapter{REVIEW QUESTIONS}
\label{\detokenize{review_questions:review-questions}}\label{\detokenize{review_questions::doc}}
\sphinxAtStartPar
This section contains review questions for the material covered in the course. These questions are designed to test your understanding and help reinforce the concepts learned.


\section{New Employee Information}
\label{\detokenize{review_questions:new-employee-information}}
\sphinxAtStartPar
Which one of the following is correct?
\begin{itemize}
\item {} 
\sphinxAtStartPar
\sphinxstylestrong{a.} Choice A

\item {} 
\sphinxAtStartPar
\sphinxstylestrong{b.} Choice B

\item {} 
\sphinxAtStartPar
\sphinxstylestrong{c.} Choice C

\end{itemize}

\sphinxstepscope


\chapter{TD1}
\label{\detokenize{TD1:td1}}\label{\detokenize{TD1::doc}}

\section{TD1 \sphinxhyphen{} 2025 Personal Tax Credits Return}
\label{\detokenize{TD1:td1-2025-personal-tax-credits-return}}

\begin{savenotes}\sphinxattablestart
\sphinxthistablewithglobalstyle
\centering
\begin{tabulary}{\linewidth}[t]{TT}
\sphinxtoprule
\sphinxtableatstartofbodyhook
\sphinxAtStartPar
Last name
&\\
\sphinxhline
\sphinxAtStartPar
First name and initial(s)
&\\
\sphinxhline
\sphinxAtStartPar
Date of birth (Year/Month/Day)
&\\
\sphinxhline
\sphinxAtStartPar
Employee number
&\\
\sphinxhline
\sphinxAtStartPar
Address
&
\sphinxAtStartPar
Postal Code
\\
\sphinxbottomrule
\end{tabulary}
\sphinxtableafterendhook\par
\sphinxattableend\end{savenotes}

\sphinxAtStartPar
1. Basic personal amount \sphinxhyphen{} Every resident of Canada can enter a basic
personal amount of \$16,129. However, if your net income from all sources will
be greater than \$177,882 and you enter \$16,129, you may have an amount owing
on your income tax and benefit return at the end of the tax year. If your
income from all sources will be greater than \$177,882 you have the option to
calculate a partial claim. To do so, fill in the appropriate section of Form
TD1\sphinxhyphen{}WS, Worksheet for the 2025 Personal Tax Credits Return, and enter the
calculated amount here. \textasciicircum{}

\sphinxAtStartPar
2. Canada caregiver amount for infirm children under age 18 \sphinxhyphen{} Only one parent
may claim \$2,687 for each infirm child born in 2008 or later who lives with
both parents throughout the year. If the child does not live with both
parents throughout the year, the parent who has the right to claim the
“Amount for an eligible dependant” on line 8 may also claim the Canada
caregiver amount for the child. \textasciicircum{}

\sphinxAtStartPar
3. Age amount \sphinxhyphen{} If you will be 65 or older on December 31, 2025, and your net
income for the year from all sources will be \$45,522 or less, enter \$9,028.
You may enter a partial amount if your net income for the year will be
between \$45,522 and \$105,709. To calculate a partial amount, fill out the
line 3 section of Form TD1\sphinxhyphen{}WS. \textasciicircum{}

\sphinxAtStartPar
4. Pension income amount \sphinxhyphen{} If you will receive regular pension payments from
a pension plan or fund (not including Canada Pension Plan, Quebec Pension
Plan, old age security, or guaranteed income supplement payments), enter
whichever is less: \$2,000 or your estimated annual pension income. \textasciicircum{}

\sphinxAtStartPar
5. Tuition (full\sphinxhyphen{}time and part\sphinxhyphen{}time) \sphinxhyphen{} Fill in this section if you are a
student at a university or college, or an educational institution certified
by Employment and Social Development Canada, and you will pay more than \$100
per institution in tuition fees. Enter the total tuition fees that you will
pay if you are a full\sphinxhyphen{}time or part\sphinxhyphen{}time student. \textasciicircum{}

\sphinxAtStartPar
6. Disability amount \sphinxhyphen{} If you will claim the disability amount on your income
tax and benefit return by using Form T2201, Disability Tax Credit
Certificate, enter \$10,138. \textasciicircum{}

\sphinxAtStartPar
7. Spouse or common\sphinxhyphen{}law partner amount \sphinxhyphen{} Enter the difference between the
amount on line 1 (line 1 plus \$2,687 if your spouse or common\sphinxhyphen{}law partner is
infirm) and your spouse’s or common\sphinxhyphen{}law partner’s estimated net income for
the year if two of the following conditions apply:
\begin{itemize}
\item {} 
\sphinxAtStartPar
You are supporting your spouse or common\sphinxhyphen{}law partner who lives with you

\item {} 
\sphinxAtStartPar
Your spouse or common\sphinxhyphen{}law partner’s net income for the year will be less

\end{itemize}

\sphinxAtStartPar
than the amount on line 1 (line 1 plus \$2,687 if your spouse or common\sphinxhyphen{}law
partner is infirm)

\sphinxAtStartPar
In all cases, go to line 9 if your spouse or common\sphinxhyphen{}law partner is infirm and
has a net income for the year of \$28,798 or less. \textasciicircum{}

\sphinxAtStartPar
8. Amount for an eligible dependant \sphinxhyphen{} Enter the difference between the amount
on line 1 (line 1 plus \$2,687 if your eligible dependant is infirm) and your
eligible dependant’s estimated net income for the year if all of the
following conditions apply:
\begin{itemize}
\item {} 
\sphinxAtStartPar
You do not have a spouse or common\sphinxhyphen{}law partner, or you have a spouse or

\end{itemize}

\sphinxAtStartPar
common\sphinxhyphen{}law partner who does not live with you and who you are not supporting
or being supported by
\begin{itemize}
\item {} 
\sphinxAtStartPar
You are supporting the dependant who is related to you and lives with you

\item {} 
\sphinxAtStartPar
The dependant’s net income for the year will be less than the amount on

\end{itemize}

\sphinxAtStartPar
line 1 (line 1 plus \$2,687 if your dependant is infirm and you cannot claim
the Canada caregiver amount for infirm children under 18 years of age for
this dependant)

\sphinxAtStartPar
In all cases, go to line 9 if your dependant is18 years or older, infirm, and
has a net income for the year of \$28,798 or less. \textasciicircum{}

\sphinxAtStartPar
9. Canada caregiver amount for eligible dependant or spouse or common\sphinxhyphen{}law
partner \sphinxhyphen{} Fill out this section if, at any time in the year, you support an
infirm eligible dependant (aged 18 or older)or an infirm spouse or common\sphinxhyphen{}law
partner whose net income for the year will be \$28,798 or less. To calculate
the amount you may enter here, fill out the line 9 section of Form TD1\sphinxhyphen{}WS. \textasciicircum{}

\sphinxAtStartPar
10. Canada caregiver amount for dependant(s) age 18 or older \sphinxhyphen{} If, at any
time in the year, you support an infirm dependant age 18 or older (other than
the spouse or common\sphinxhyphen{}law partner or eligible dependant you claimed an amount
for on line 9 or could have claimed an amount for if their net income were
under \$18,816) whose net income for the year will be \$20,197 or less, enter
\$8,601. You may enter a partial amount if their net income for the year will
be between \$20,197 and \$28,798. To calculate a partial amount, fill out the
line 10 section of Form TD1\sphinxhyphen{}WS. This worksheet may also be used to calculate
your part of the amount if you are sharing it with another caregiver who
supports the same dependant. You may claim this amount for more than one
infirm dependant age 18 or older. \textasciicircum{}

\sphinxAtStartPar
11. Amounts transferred from your spouse or common\sphinxhyphen{}law partner \sphinxhyphen{} If your
spouse or common\sphinxhyphen{}law partner will not use all of their age amount, pension
income amount, tuition amount, or disability amount on their income tax and
benefit return, enter the unused amount. \textasciicircum{}

\sphinxAtStartPar
12. Amounts transferred from a dependant \sphinxhyphen{} If your dependant will not use all
of their disability amount on their income tax and benefit return, enter the
unused amount. If your or your spouse’s or common\sphinxhyphen{}law partner’s dependent
child or grandchild will not use all of their tuition amount on their income
tax and benefit return, enter the unused amount. \textasciicircum{}

\sphinxAtStartPar
13. TOTAL CLAIM AMOUNT \sphinxhyphen{} Add lines 1 to 12. Your employer or payer will use
this amount to determine the amount of your tax deductions. \textasciicircum{}

\sphinxstepscope


\chapter{CALCULATING NET PAY}
\label{\detokenize{calculating-net-pay:calculating-net-pay}}\label{\detokenize{calculating-net-pay::doc}}

\section{Salary}
\label{\detokenize{calculating-net-pay:salary}}

\section{Commission}
\label{\detokenize{calculating-net-pay:commission}}

\section{Pension}
\label{\detokenize{calculating-net-pay:pension}}
\sphinxstepscope


\chapter{RATES FOR 2025}
\label{\detokenize{rates_2025:rates-for-2025}}\label{\detokenize{rates_2025::doc}}

\section{CANADA / QUEBEC PENSION PLAN (CPP / QPP)}
\label{\detokenize{rates_2025:canada-quebec-pension-plan-cpp-qpp}}

\begin{savenotes}\sphinxattablestart
\sphinxthistablewithglobalstyle
\raggedright
\sphinxcapstartof{table}
\sphinxthecaptionisattop
\sphinxcaption{CANADA / QUEBEC PENSION PLAN (CPP / QPP)}\label{\detokenize{rates_2025:id1}}
\sphinxaftertopcaption
\begin{tabular}[t]{\X{130}{190}\X{30}{190}\X{30}{190}}
\sphinxtoprule
\sphinxstyletheadfamily 
\sphinxAtStartPar
Description
&\sphinxstyletheadfamily 
\sphinxAtStartPar
CPP
&\sphinxstyletheadfamily 
\sphinxAtStartPar
QPP
\\
\sphinxmidrule
\sphinxtableatstartofbodyhook
\sphinxAtStartPar
Yearly maximum pensionable earnings
&
\sphinxAtStartPar
\$71,300
&
\sphinxAtStartPar
\$
\\
\sphinxhline
\sphinxAtStartPar
Annual maximum contributory earnings
&
\sphinxAtStartPar
\$67,800
&
\sphinxAtStartPar
\$
\\
\sphinxhline
\sphinxAtStartPar
Annual maximum contribution
&
\sphinxAtStartPar
\$3,500
&
\sphinxAtStartPar
\$
\\
\sphinxhline
\sphinxAtStartPar
Employee contribution rate
&
\sphinxAtStartPar
5.95\%
&\\
\sphinxhline
\sphinxAtStartPar
Employer contribution rate
&
\sphinxAtStartPar
5.95\%
&\\
\sphinxhline
\sphinxAtStartPar
Basic exemption (Annual)
&
\sphinxAtStartPar
\$3,500
&\\
\sphinxhline\begin{quote}

\sphinxAtStartPar
Basic exemption (Monthly, 12)
\end{quote}
&
\sphinxAtStartPar
\$291.67
&
\sphinxAtStartPar
\$
\\
\sphinxhline\begin{quote}

\sphinxAtStartPar
Basic exemption (Weekly, 52)
\end{quote}
&
\sphinxAtStartPar
\$67.31
&
\sphinxAtStartPar
\$
\\
\sphinxhline\begin{quote}

\sphinxAtStartPar
Basic exemption (Weekly, 53)
\end{quote}
&
\sphinxAtStartPar
\$66.04
&
\sphinxAtStartPar
\$
\\
\sphinxhline\begin{quote}

\sphinxAtStartPar
Basic exemption (Semi\sphinxhyphen{}monthly, 24)
\end{quote}
&
\sphinxAtStartPar
\$145.83
&
\sphinxAtStartPar
\$
\\
\sphinxhline\begin{quote}

\sphinxAtStartPar
Basic exemption (Bi\sphinxhyphen{}weekly, 26)
\end{quote}
&
\sphinxAtStartPar
\$134.61
&
\sphinxAtStartPar
\$
\\
\sphinxbottomrule
\end{tabular}
\sphinxtableafterendhook\par
\sphinxattableend\end{savenotes}


\section{CPP2 CONTRIBUTION RATES MAXIMUMS}
\label{\detokenize{rates_2025:cpp2-contribution-rates-maximums}}

\begin{savenotes}\sphinxattablestart
\sphinxthistablewithglobalstyle
\raggedright
\sphinxcapstartof{table}
\sphinxthecaptionisattop
\sphinxcaption{CPP2 Contribution Rates Maximums}\label{\detokenize{rates_2025:id2}}
\sphinxaftertopcaption
\begin{tabular}[t]{\X{130}{160}\X{30}{160}}
\sphinxtoprule
\sphinxstyletheadfamily 
\sphinxAtStartPar
Description
&\sphinxstyletheadfamily 
\sphinxAtStartPar
Ammount
\\
\sphinxmidrule
\sphinxtableatstartofbodyhook
\sphinxAtStartPar
Additional maximum annual pensionable earnings
&
\sphinxAtStartPar
\$81,200
\\
\sphinxhline
\sphinxAtStartPar
Employee and employer contribution rate
&
\sphinxAtStartPar
4\%
\\
\sphinxhline
\sphinxAtStartPar
Maximum employee and employer contribution
&
\sphinxAtStartPar
\$396
\\
\sphinxhline
\sphinxAtStartPar
Maimum annual self\sphinxhyphen{}employed contribution
&
\sphinxAtStartPar
\$792
\\
\sphinxbottomrule
\end{tabular}
\sphinxtableafterendhook\par
\sphinxattableend\end{savenotes}


\section{References}
\label{\detokenize{rates_2025:references}}
\sphinxAtStartPar
\sphinxhref{https://laws-lois.justice.gc.ca/eng/acts/C-8/page-5.html\#docCont}{CPP Maximum contributory earnings}

\sphinxAtStartPar
\sphinxhref{https://laws-lois.justice.gc.ca/eng/acts/C-8/page-3.html\#docCont}{Second additional CPP contributions}

\sphinxstepscope


\chapter{REFERENCES}
\label{\detokenize{references:references}}\label{\detokenize{references::doc}}
\sphinxAtStartPar
\sphinxhref{https://www.canada.ca/en/revenue-agency/services/e-services/digital-services-businesses/payroll-deductions-online-calculator.html}{Payroll Deductions Online Calculator}

\sphinxstepscope


\chapter{Errors and Errata}
\label{\detokenize{errata:errors-and-errata}}\label{\detokenize{errata::doc}}
\sphinxstepscope


\chapter{TITLE \#}
\label{\detokenize{syntax:title}}\label{\detokenize{syntax::doc}}

\section{TITLE =}
\label{\detokenize{syntax:id1}}

\chapter{Glossary}
\label{\detokenize{index:glossary}}\begin{itemize}
\item {} 
\sphinxAtStartPar
\DUrole{xref}{\DUrole{std}{\DUrole{std-ref}{genindex}}}

\end{itemize}


\chapter{Canadian Payroll Administration}
\label{\detokenize{index:canadian-payroll-administration}}
\begin{sphinxVerbatim}[commandchars=\\\{\}]
Python 3.12.3
\end{sphinxVerbatim}

\begin{sphinxVerbatim}[commandchars=\\\{\}]
5
\end{sphinxVerbatim}



\renewcommand{\indexname}{Index}
\printindex
\end{document}