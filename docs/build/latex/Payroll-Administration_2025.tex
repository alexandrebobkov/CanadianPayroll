%% Generated by Sphinx.
\def\sphinxdocclass{report}
\documentclass[letterpaper,10pt,english]{sphinxmanual}
\ifdefined\pdfpxdimen
   \let\sphinxpxdimen\pdfpxdimen\else\newdimen\sphinxpxdimen
\fi \sphinxpxdimen=.75bp\relax
\ifdefined\pdfimageresolution
    \pdfimageresolution= \numexpr \dimexpr1in\relax/\sphinxpxdimen\relax
\fi
%% let collapsible pdf bookmarks panel have high depth per default
\PassOptionsToPackage{bookmarksdepth=5}{hyperref}

\PassOptionsToPackage{booktabs}{sphinx}
\PassOptionsToPackage{colorrows}{sphinx}

\PassOptionsToPackage{warn}{textcomp}
\usepackage[utf8]{inputenc}
\ifdefined\DeclareUnicodeCharacter
% support both utf8 and utf8x syntaxes
  \ifdefined\DeclareUnicodeCharacterAsOptional
    \def\sphinxDUC#1{\DeclareUnicodeCharacter{"#1}}
  \else
    \let\sphinxDUC\DeclareUnicodeCharacter
  \fi
  \sphinxDUC{00A0}{\nobreakspace}
  \sphinxDUC{2500}{\sphinxunichar{2500}}
  \sphinxDUC{2502}{\sphinxunichar{2502}}
  \sphinxDUC{2514}{\sphinxunichar{2514}}
  \sphinxDUC{251C}{\sphinxunichar{251C}}
  \sphinxDUC{2572}{\textbackslash}
\fi
\usepackage{cmap}
\usepackage[T1]{fontenc}
\usepackage{amsmath,amssymb,amstext}
\usepackage{babel}



\usepackage{tgtermes}
\usepackage{tgheros}
\renewcommand{\ttdefault}{txtt}



\usepackage[Bjarne]{fncychap}
\usepackage{sphinx}

\fvset{fontsize=auto}
\usepackage{geometry}


% Include hyperref last.
\usepackage{hyperref}
% Fix anchor placement for figures with captions.
\usepackage{hypcap}% it must be loaded after hyperref.
% Set up styles of URL: it should be placed after hyperref.
\urlstyle{same}

\addto\captionsenglish{\renewcommand{\contentsname}{Table of Contents:}}

\usepackage{sphinxmessages}
\setcounter{tocdepth}{1}



\title{Canadian Payroll Administration (2025)}
\date{Jun 27, 2025}
\release{Fall 2025}
\author{Alexandre Bobkov}
\newcommand{\sphinxlogo}{\vbox{}}
\renewcommand{\releasename}{Release}
\makeindex
\begin{document}

\ifdefined\shorthandoff
  \ifnum\catcode`\=\string=\active\shorthandoff{=}\fi
  \ifnum\catcode`\"=\active\shorthandoff{"}\fi
\fi

\pagestyle{empty}
\sphinxmaketitle
\pagestyle{plain}
\sphinxtableofcontents
\pagestyle{normal}
\phantomsection\label{\detokenize{index::doc}}


\sphinxAtStartPar
Alexander Bobkov (Alex) is the author of this comprehensive and practical study guide for Payroll Administration,
drawing on nearly two decades of hands\sphinxhyphen{}on experience in the accounting field. From 2005 to 2022, Alexander successfully
operated his own accounting firm, offering bookkeeping, accounting, and payroll services to a diverse clientele in the National Capital Regions.
With a rich educational background that spans from a college diploma to a Master’s degree in Business, he brings both academic insight
and practical expertise to his work. For the past five years, Alexander has focused specifically on the payroll sector. This study guide reflects
his long\sphinxhyphen{}standing goal: to help professional bookkeepers and business managers to build a solid foundation in payroll administration while easing
the anxiety often associated with its complexity. Designed to be clear, practical, and empowering, the guide equips readers with the skills needed
to confidently perform essential payroll functions encountered in day\sphinxhyphen{}to\sphinxhyphen{}day operations.

\sphinxstepscope


\chapter{PREFACE}
\label{\detokenize{preface:preface}}\label{\detokenize{preface::doc}}
\sphinxAtStartPar
Through this material, students will gain a comprehensive understanding of core payroll principles and practices. They will explore legislative compliance requirements and the role of key regulatory bodies that govern payroll operations in Canada.

\sphinxAtStartPar
Students will learn how to:
\begin{itemize}
\item {} 
\sphinxAtStartPar
Accurately calculate net pay for salaried, hourly, commissioned, and contract employees.

\item {} 
\sphinxAtStartPar
Identify and meet payroll\sphinxhyphen{}related obligations for businesses.

\item {} 
\sphinxAtStartPar
Navigate the administrative aspects of human resource management that intersect with payroll responsibilities.

\item {} 
\sphinxAtStartPar
Apply payroll procedures using computerized payroll software through practical, hands\sphinxhyphen{}on exercises.

\item {} 
\sphinxAtStartPar
Payroll’s responsibilities from hiring through to termination.

\item {} 
\sphinxAtStartPar
Payroll compliance legislation in practical scenarios.

\item {} 
\sphinxAtStartPar
Individual pay calculation process.

\end{itemize}


\section{Learning Outcomes}
\label{\detokenize{preface:learning-outcomes}}
\sphinxAtStartPar
The material of this study guide aim to make students to be be able to:
\begin{itemize}
\item {} 
\sphinxAtStartPar
Calculate regular individual pay

\item {} 
\sphinxAtStartPar
Calculate non\sphinxhyphen{}regular individual pay

\item {} 
\sphinxAtStartPar
Calculate termination payments

\item {} 
\sphinxAtStartPar
Complete a Record of Employment (ROE)

\item {} 
\sphinxAtStartPar
Apply federal and provincial legislation to payroll, including:
\sphinxhyphen{} The Canada Pension Plan Act
\sphinxhyphen{} The Employment Insurance Act
\sphinxhyphen{} The Income Tax Act
\sphinxhyphen{} Employment Standards legislation
\sphinxhyphen{} Workers’ Compensation Acts
\sphinxhyphen{} Québec\sphinxhyphen{}specific legislation

\end{itemize}


\section{Recommended Course Material}
\label{\detokenize{preface:recommended-course-material}}

\section{Material Structure Overview}
\label{\detokenize{preface:material-structure-overview}}\begin{enumerate}
\sphinxsetlistlabels{\arabic}{enumi}{enumii}{}{.}%
\item {} 
\sphinxAtStartPar
Introduction to Canadian Payroll

\item {} 
\sphinxAtStartPar
Labour and Employment Standards

\item {} 
\sphinxAtStartPar
Accounting for Payroll

\item {} 
\sphinxAtStartPar
Calculating Gross Pay

\item {} 
\sphinxAtStartPar
Pensionable, Insurable, and Taxable Earnings

\item {} 
\sphinxAtStartPar
Calculating Net Pay

\item {} 
\sphinxAtStartPar
Calculating Employer’s Source Deduction Remittances

\item {} 
\sphinxAtStartPar
Termination of Employment:
\begin{itemize}
\item {} 
\sphinxAtStartPar
Record of Employment (ROE)

\item {} 
\sphinxAtStartPar
Termination Payments

\item {} 
\sphinxAtStartPar
Retirement Pay

\end{itemize}

\end{enumerate}

\sphinxAtStartPar
In other words, the material covers the foundational knowledge and technical skills needed to confidently perform payroll tasks in a variety of employment settings.

\sphinxstepscope


\chapter{INTRODUCTION}
\label{\detokenize{introduction:introduction}}\label{\detokenize{introduction::doc}}

\section{Outcomes}
\label{\detokenize{introduction:outcomes}}
\sphinxAtStartPar
Applying federal and provincial payroll legislation, regulations, and policies to ensure compliance with the legal framework governing payroll in Canada.
\begin{itemize}
\item {} 
\sphinxAtStartPar
CPP/QPP

\item {} 
\sphinxAtStartPar
EI

\item {} 
\sphinxAtStartPar
Income Tax (Federal, ON and QC)

\end{itemize}

\sphinxAtStartPar
Calculating regular individual pay

\sphinxAtStartPar
Calculating non\sphinxhyphen{}regular individual pay

\sphinxAtStartPar
Calculating termination pay

\sphinxAtStartPar
Completing a Record of Employment (ROE)


\subsection{Payroll Legal Framework}
\label{\detokenize{introduction:payroll-legal-framework}}
\sphinxAtStartPar
The Canadian Payroll Administration system is designed to ensure compliance with the legal framework governing payroll in Canada. This includes adherence to federal and provincial regulations regarding employee compensation, deductions, and reporting requirements.
The system is built to handle various payroll scenarios, including different employment types, tax calculations, and benefit deductions, while ensuring that all transactions are accurately recorded and reported in accordance with the law.

\sphinxstepscope


\chapter{Payroll Accounting}
\label{\detokenize{payroll_accounting:payroll-accounting}}\label{\detokenize{payroll_accounting::doc}}

\section{Journal Entries}
\label{\detokenize{payroll_accounting:journal-entries}}

\subsection{Accounting Recap}
\label{\detokenize{payroll_accounting:accounting-recap}}
\begin{center}\(\Sigma \text{ Total Debits} = \Sigma \text{ Total Credits}\)
\end{center}
\begin{center}\(\text{Assets} = \text{Liabilities} + \text{Equity}\)
\end{center}\begin{equation}\label{equation:payroll_accounting:AccountingEquation}
\begin{split}Assets = Liabilities + Equity\end{split}
\end{equation}
\sphinxAtStartPar
Furthermore, we know that:

\begin{center}\(\text{Equity = Revenue - Expenses}\)
, which leads us to:
\end{center}
\begin{center}\(\text{Assets = Liabilities + (Revenues - Expenses)}\)
\end{center}
\sphinxAtStartPar
Accounting equation \eqref{equation:payroll_accounting:AccountingEquation}

\sphinxAtStartPar
Payroll accounting is a critical component of the Canadian Payroll Administration system. It involves the systematic recording, analysis, and reporting of payroll transactions to ensure that all financial aspects of employee compensation are accurately reflected in the organization’s financial statements.
Payroll accounting includes the management of employee wages, tax withholdings, benefit deductions, and other payroll\sphinxhyphen{}related expenses. The system is designed to automate these processes, ensuring accuracy and compliance with Canadian payroll regulations.


\subsection{Journal Entries}
\label{\detokenize{payroll_accounting:id1}}
\sphinxAtStartPar
Journal entries are a key part of payroll accounting, as they document the financial impact of payroll transactions on the organization’s accounts. Each payroll run generates a series of journal entries that reflect the distribution of wages, taxes, and deductions across various accounts.
These entries are essential for maintaining accurate financial records and ensuring that the organization’s financial statements reflect the true cost of employee compensation. The Canadian Payroll Administration system automates the generation of these journal entries, reducing the risk of errors and ensuring compliance with accounting standards.

\begin{DUlineblock}{0em}
\item[] DR    Payroll Expenses    \$10,500.00
\item[]
\begin{DUlineblock}{\DUlineblockindent}
\item[] CR  Payroll Payable   \$10,500.00
\end{DUlineblock}
\end{DUlineblock}

\sphinxstepscope


\chapter{REVIEW QUESTIONS}
\label{\detokenize{review_questions:review-questions}}\label{\detokenize{review_questions::doc}}
\sphinxAtStartPar
This section contains review questions for the material covered in the course. These questions are designed to test your understanding and help reinforce the concepts learned.


\section{New Employee Information}
\label{\detokenize{review_questions:new-employee-information}}
\sphinxAtStartPar
Which one of the following is correct?
\begin{itemize}
\item {} 
\sphinxAtStartPar
\sphinxstylestrong{a.} Choice A

\item {} 
\sphinxAtStartPar
\sphinxstylestrong{b.} Choice B

\item {} 
\sphinxAtStartPar
\sphinxstylestrong{c.} Choice C

\end{itemize}

\sphinxstepscope


\chapter{OBNOARDING EMPLOYEE}
\label{\detokenize{onboarding_employee:obnoarding-employee}}\label{\detokenize{onboarding_employee::doc}}

\section{Employment Standards Requirements}
\label{\detokenize{onboarding_employee:employment-standards-requirements}}
\sphinxAtStartPar
Each province/territory, as well as the federal government, sets minimum employment standards, including:
\begin{itemize}
\item {} 
\sphinxAtStartPar
Minimum wage

\item {} 
\sphinxAtStartPar
Minimum age (may also be governed by other legislation)

\item {} 
\sphinxAtStartPar
Required pay statement information:
\sphinxhyphen{} Employee name
\sphinxhyphen{} Pay period date
\sphinxhyphen{} Rates of pay and hours worked
\sphinxhyphen{} Gross earnings
\sphinxhyphen{} Itemized deductions
\sphinxhyphen{} Net pay

\end{itemize}


\section{Internal Forms}
\label{\detokenize{onboarding_employee:internal-forms}}
\sphinxAtStartPar
Typical commencement package forms include:
\begin{itemize}
\item {} 
\sphinxAtStartPar
Authorization for hiring

\item {} 
\sphinxAtStartPar
Direct deposit agreement

\item {} 
\sphinxAtStartPar
Union membership application

\item {} 
\sphinxAtStartPar
Benefits enrollment (e.g., health/dental, pension)

\item {} 
\sphinxAtStartPar
Confidentiality agreement

\end{itemize}


\subsection{Authorization for Hiring}
\label{\detokenize{onboarding_employee:authorization-for-hiring}}
\sphinxAtStartPar
This internal document includes:
\begin{itemize}
\item {} 
\sphinxAtStartPar
New employee’s basic info

\item {} 
\sphinxAtStartPar
Start date, department, salary

\item {} 
\sphinxAtStartPar
Probation details

\item {} 
\sphinxAtStartPar
Hiring authority’s signature

\end{itemize}

\sphinxAtStartPar
\sphinxstylestrong{Important:} Employer must obtain a valid SIN. A SIN starting with 9 must have a valid expiry date and associated work permit.


\subsection{Union Membership}
\label{\detokenize{onboarding_employee:union-membership}}
\sphinxAtStartPar
For unionized workplaces:
\begin{itemize}
\item {} 
\sphinxAtStartPar
Union dues are deducted

\item {} 
\sphinxAtStartPar
Employees sign authorization for deduction

\item {} 
\sphinxAtStartPar
Exemptions may apply, but dues equivalent still required

\end{itemize}


\subsection{Benefit Enrollment Forms}
\label{\detokenize{onboarding_employee:benefit-enrollment-forms}}
\sphinxAtStartPar
Forms cover group insurance and pension plans:
\begin{itemize}
\item {} 
\sphinxAtStartPar
Employee indicates coverage type

\item {} 
\sphinxAtStartPar
Signatures authorize payroll deductions

\end{itemize}


\subsection{Confidentiality Agreement}
\label{\detokenize{onboarding_employee:confidentiality-agreement}}
\sphinxAtStartPar
A legally binding agreement protecting sensitive company info:
\begin{itemize}
\item {} 
\sphinxAtStartPar
Defines proprietary data

\item {} 
\sphinxAtStartPar
Outlines responsibilities, penalties, and timeframe

\end{itemize}


\section{Required Federal and Provincial/Territorial Forms}
\label{\detokenize{onboarding_employee:required-federal-and-provincial-territorial-forms}}
\sphinxAtStartPar
\sphinxstylestrong{Purpose:} Determine correct income tax withholdings.

\sphinxAtStartPar
Forms:
\begin{itemize}
\item {} 
\sphinxAtStartPar
TD1 (federal)

\item {} 
\sphinxAtStartPar
TD1 (provincial/territorial)

\item {} 
\sphinxAtStartPar
Québec employees: also TP\sphinxhyphen{}1015.3\sphinxhyphen{}V

\end{itemize}

\sphinxAtStartPar
\sphinxstylestrong{Provincial/territorial withholding} is based on \sphinxstyleemphasis{province of employment}, but tax liability is based on \sphinxstyleemphasis{province of residence}.

\sphinxAtStartPar
\sphinxstylestrong{Adjustments:}
\begin{itemize}
\item {} 
\sphinxAtStartPar
Request extra withholding via TD1 or TP\sphinxhyphen{}1015.3\sphinxhyphen{}V

\item {} 
\sphinxAtStartPar
Request reduction using CRA Form T1213 or RQ Form TP\sphinxhyphen{}1016\sphinxhyphen{}V

\end{itemize}

\sphinxAtStartPar
Essential Info on All Forms:
\begin{itemize}
\item {} 
\sphinxAtStartPar
Employee name

\item {} 
\sphinxAtStartPar
Date of birth

\item {} 
\sphinxAtStartPar
Social Insurance Number

\end{itemize}


\subsection{Tax Credits (TD1)}
\label{\detokenize{onboarding_employee:tax-credits-td1}}\begin{enumerate}
\sphinxsetlistlabels{\arabic}{enumi}{enumii}{}{.}%
\item {} 
\sphinxAtStartPar
Basic personal amount

\item {} 
\sphinxAtStartPar
Canada caregiver (infirm children)

\item {} 
\sphinxAtStartPar
Age amount

\item {} 
\sphinxAtStartPar
Pension income

\item {} 
\sphinxAtStartPar
Tuition

\item {} 
\sphinxAtStartPar
Disability

\item {} 
\sphinxAtStartPar
Spouse/common\sphinxhyphen{}law partner amount

\item {} 
\sphinxAtStartPar
Eligible dependant

\item {} 
\sphinxAtStartPar
Caregiver for infirm spouse or dependant

\item {} 
\sphinxAtStartPar
Caregiver for dependant age 18+

\item {} 
\sphinxAtStartPar
Transfers from spouse

\item {} 
\sphinxAtStartPar
Transfers from dependant

\item {} 
\sphinxAtStartPar
Total

\end{enumerate}

\sphinxAtStartPar
Additional Instructions:
\begin{itemize}
\item {} 
\sphinxAtStartPar
Fill out TD1 only if claiming more than basic credit

\item {} 
\sphinxAtStartPar
Québec employees must always complete TP\sphinxhyphen{}1015.3\sphinxhyphen{}V

\end{itemize}


\subsection{Tax Credits (TP\sphinxhyphen{}1015.3\sphinxhyphen{}V \textendash{} Québec)}
\label{\detokenize{onboarding_employee:tax-credits-tp-1015-3-v-quebec}}\begin{itemize}
\item {} 
\sphinxAtStartPar
Basic amount

\item {} 
\sphinxAtStartPar
Transfer from spouse

\item {} 
\sphinxAtStartPar
Amount for dependants

\item {} 
\sphinxAtStartPar
Impairment in mental/physical function

\item {} 
\sphinxAtStartPar
Age amount, retirement income, living alone

\item {} 
\sphinxAtStartPar
Career extension

\end{itemize}

\sphinxAtStartPar
Deductions:
\begin{itemize}
\item {} 
\sphinxAtStartPar
Remote area housing

\item {} 
\sphinxAtStartPar
Deductible support payments

\end{itemize}


\section{Content Review Highlights}
\label{\detokenize{onboarding_employee:content-review-highlights}}\begin{itemize}
\item {} 
\sphinxAtStartPar
Consent is required for personal info collection

\item {} 
\sphinxAtStartPar
TD1 and TP\sphinxhyphen{}1015.3\sphinxhyphen{}V are used to calculate source deductions

\item {} 
\sphinxAtStartPar
Claim amounts may differ between federal and provincial forms

\item {} 
\sphinxAtStartPar
Employers must keep the forms on file (do not send to CRA/RQ)

\end{itemize}


\section{Review Questions (Sample)}
\label{\detokenize{onboarding_employee:review-questions-sample}}\begin{enumerate}
\sphinxsetlistlabels{\arabic}{enumi}{enumii}{}{.}%
\item {} 
\sphinxAtStartPar
What does an offer letter signature signify?

\item {} 
\sphinxAtStartPar
What documents are included in a commencement package?

\item {} 
\sphinxAtStartPar
Name three common internal forms

\item {} 
\sphinxAtStartPar
What must payroll verify on a hiring form?

\item {} 
\sphinxAtStartPar
What must be checked for SINs starting with “9”?

\item {} 
\sphinxAtStartPar
True/False: Union dues can be deducted without consent.

\item {} 
\sphinxAtStartPar
What authorizes benefit premium deductions?

\end{enumerate}


\section{Example Evaluations}
\label{\detokenize{onboarding_employee:example-evaluations}}
\sphinxAtStartPar
\sphinxstylestrong{Gloria Meyer (Alberta):}
\sphinxhyphen{} Claimed: Basic, eligible dependant, transferred tuition
\sphinxhyphen{} Appears accurate

\sphinxAtStartPar
\sphinxstylestrong{Luc Laframboise (Québec):}
\sphinxhyphen{} Claimed: Basic, spouse, dependant in school, tuition transfer
\sphinxhyphen{} Appropriate provincial and federal claims made

\sphinxAtStartPar
\sphinxstylestrong{Ingrid Johansson (Alberta, Single Parent):}
\sphinxhyphen{} Claimed credits for two children
\sphinxhyphen{} \sphinxstylestrong{Overclaimed} dependant credit \textendash{} only one is eligible
\sphinxhyphen{} Needs correction on federal and AB TD1 forms

\sphinxstepscope


\chapter{RATES FOR 2025}
\label{\detokenize{rates_2025:rates-for-2025}}\label{\detokenize{rates_2025::doc}}

\section{CANADA / QUEBEC PENSION PLAN (CPP / QPP)}
\label{\detokenize{rates_2025:canada-quebec-pension-plan-cpp-qpp}}

\begin{savenotes}\sphinxattablestart
\sphinxthistablewithglobalstyle
\raggedright
\sphinxcapstartof{table}
\sphinxthecaptionisattop
\sphinxcaption{CANADA / QUEBEC PENSION PLAN (CPP / QPP)}\label{\detokenize{rates_2025:id1}}
\sphinxaftertopcaption
\begin{tabular}[t]{\X{130}{190}\X{30}{190}\X{30}{190}}
\sphinxtoprule
\sphinxstyletheadfamily 
\sphinxAtStartPar
Description
&\sphinxstyletheadfamily 
\sphinxAtStartPar
CPP
&\sphinxstyletheadfamily 
\sphinxAtStartPar
QPP
\\
\sphinxmidrule
\sphinxtableatstartofbodyhook
\sphinxAtStartPar
Yearly maximum pensionable earnings
&
\sphinxAtStartPar
\$71,300
&
\sphinxAtStartPar
\$
\\
\sphinxhline
\sphinxAtStartPar
Annual maximum contributory earnings
&
\sphinxAtStartPar
\$67,800
&
\sphinxAtStartPar
\$
\\
\sphinxhline
\sphinxAtStartPar
Annual maximum contribution
&
\sphinxAtStartPar
\$3,500
&
\sphinxAtStartPar
\$
\\
\sphinxhline
\sphinxAtStartPar
Employee contribution rate
&
\sphinxAtStartPar
5.95\%
&\\
\sphinxhline
\sphinxAtStartPar
Employer contribution rate
&
\sphinxAtStartPar
5.95\%
&\\
\sphinxhline
\sphinxAtStartPar
Basic exemption (Annual)
&
\sphinxAtStartPar
\$3,500
&\\
\sphinxhline\begin{quote}

\sphinxAtStartPar
Basic exemption (Monthly, 12)
\end{quote}
&
\sphinxAtStartPar
\$291.67
&
\sphinxAtStartPar
\$
\\
\sphinxhline\begin{quote}

\sphinxAtStartPar
Basic exemption (Weekly, 52)
\end{quote}
&
\sphinxAtStartPar
\$67.31
&
\sphinxAtStartPar
\$
\\
\sphinxhline\begin{quote}

\sphinxAtStartPar
Basic exemption (Weekly, 53)
\end{quote}
&
\sphinxAtStartPar
\$66.04
&
\sphinxAtStartPar
\$
\\
\sphinxhline\begin{quote}

\sphinxAtStartPar
Basic exemption (Semi\sphinxhyphen{}monthly, 24)
\end{quote}
&
\sphinxAtStartPar
\$145.83
&
\sphinxAtStartPar
\$
\\
\sphinxhline\begin{quote}

\sphinxAtStartPar
Basic exemption (Bi\sphinxhyphen{}weekly, 26)
\end{quote}
&
\sphinxAtStartPar
\$134.61
&
\sphinxAtStartPar
\$
\\
\sphinxbottomrule
\end{tabular}
\sphinxtableafterendhook\par
\sphinxattableend\end{savenotes}


\section{CPP2 CONTRIBUTION RATES MAXIMUMS}
\label{\detokenize{rates_2025:cpp2-contribution-rates-maximums}}

\begin{savenotes}\sphinxattablestart
\sphinxthistablewithglobalstyle
\raggedright
\sphinxcapstartof{table}
\sphinxthecaptionisattop
\sphinxcaption{CPP2 Contribution Rates Maximums}\label{\detokenize{rates_2025:id2}}
\sphinxaftertopcaption
\begin{tabular}[t]{\X{130}{160}\X{30}{160}}
\sphinxtoprule
\sphinxstyletheadfamily 
\sphinxAtStartPar
Description
&\sphinxstyletheadfamily 
\sphinxAtStartPar
Ammount
\\
\sphinxmidrule
\sphinxtableatstartofbodyhook
\sphinxAtStartPar
Additional maximum annual pensionable earnings
&
\sphinxAtStartPar
\$81,200
\\
\sphinxhline
\sphinxAtStartPar
Employee and employer contribution rate
&
\sphinxAtStartPar
4\%
\\
\sphinxhline
\sphinxAtStartPar
Maximum employee and employer contribution
&
\sphinxAtStartPar
\$396
\\
\sphinxhline
\sphinxAtStartPar
Maimum annual self\sphinxhyphen{}employed contribution
&
\sphinxAtStartPar
\$792
\\
\sphinxbottomrule
\end{tabular}
\sphinxtableafterendhook\par
\sphinxattableend\end{savenotes}


\section{References}
\label{\detokenize{rates_2025:references}}
\sphinxAtStartPar
\sphinxhref{https://laws-lois.justice.gc.ca/eng/acts/C-8/page-5.html\#docCont}{CPP Maximum contributory earnings}

\sphinxAtStartPar
\sphinxhref{https://laws-lois.justice.gc.ca/eng/acts/C-8/page-3.html\#docCont}{Second additional CPP contributions}

\sphinxstepscope


\chapter{REFERENCES}
\label{\detokenize{references:references}}\label{\detokenize{references::doc}}
\sphinxstepscope


\chapter{Errors and Errata}
\label{\detokenize{errata:errors-and-errata}}\label{\detokenize{errata::doc}}
\sphinxstepscope


\chapter{TITLE \#}
\label{\detokenize{syntax:title}}\label{\detokenize{syntax::doc}}

\section{TITLE =}
\label{\detokenize{syntax:id1}}

\chapter{Glossary}
\label{\detokenize{index:glossary}}\begin{itemize}
\item {} 
\sphinxAtStartPar
\DUrole{xref}{\DUrole{std}{\DUrole{std-ref}{genindex}}}

\end{itemize}


\chapter{Canadian Payroll Administration}
\label{\detokenize{index:canadian-payroll-administration}}
\begin{sphinxVerbatim}[commandchars=\\\{\}]
Python 3.12.3
\end{sphinxVerbatim}

\begin{sphinxVerbatim}[commandchars=\\\{\}]
5
\end{sphinxVerbatim}



\renewcommand{\indexname}{Index}
\printindex
\end{document}