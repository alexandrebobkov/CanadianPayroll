%% Generated by Sphinx.
\def\sphinxdocclass{report}
\documentclass[letterpaper,10pt,english]{sphinxmanual}
\ifdefined\pdfpxdimen
   \let\sphinxpxdimen\pdfpxdimen\else\newdimen\sphinxpxdimen
\fi \sphinxpxdimen=.75bp\relax
\ifdefined\pdfimageresolution
    \pdfimageresolution= \numexpr \dimexpr1in\relax/\sphinxpxdimen\relax
\fi
%% let collapsible pdf bookmarks panel have high depth per default
\PassOptionsToPackage{bookmarksdepth=5}{hyperref}

\PassOptionsToPackage{booktabs}{sphinx}
\PassOptionsToPackage{colorrows}{sphinx}

\PassOptionsToPackage{warn}{textcomp}
\usepackage[utf8]{inputenc}
\ifdefined\DeclareUnicodeCharacter
% support both utf8 and utf8x syntaxes
  \ifdefined\DeclareUnicodeCharacterAsOptional
    \def\sphinxDUC#1{\DeclareUnicodeCharacter{"#1}}
  \else
    \let\sphinxDUC\DeclareUnicodeCharacter
  \fi
  \sphinxDUC{00A0}{\nobreakspace}
  \sphinxDUC{2500}{\sphinxunichar{2500}}
  \sphinxDUC{2502}{\sphinxunichar{2502}}
  \sphinxDUC{2514}{\sphinxunichar{2514}}
  \sphinxDUC{251C}{\sphinxunichar{251C}}
  \sphinxDUC{2572}{\textbackslash}
\fi
\usepackage{cmap}
\usepackage[T1]{fontenc}
\usepackage{amsmath,amssymb,amstext}
\usepackage{babel}



\usepackage{tgtermes}
\usepackage{tgheros}
\renewcommand{\ttdefault}{txtt}



\usepackage[Bjarne]{fncychap}
\usepackage{sphinx}

\fvset{fontsize=auto}
\usepackage{geometry}


% Include hyperref last.
\usepackage{hyperref}
% Fix anchor placement for figures with captions.
\usepackage{hypcap}% it must be loaded after hyperref.
% Set up styles of URL: it should be placed after hyperref.
\urlstyle{same}


\usepackage{sphinxmessages}
\setcounter{tocdepth}{2}



\title{Canadian Payroll Administration (2025)}
\date{Aug 26, 2025}
\release{Fall 2025 (v25.08.23)}
\author{Alexandre Bobkov}
\newcommand{\sphinxlogo}{\vbox{}}
\renewcommand{\releasename}{Release}
\makeindex
\begin{document}

\ifdefined\shorthandoff
  \ifnum\catcode`\=\string=\active\shorthandoff{=}\fi
  \ifnum\catcode`\"=\active\shorthandoff{"}\fi
\fi

\pagestyle{empty}
\sphinxmaketitle
\pagestyle{plain}
\sphinxtableofcontents
\pagestyle{normal}
\phantomsection\label{\detokenize{index::doc}}
\sphinxstepscope



\sphinxAtStartPar
Payroll is a necessary function in every organization that has employees,
as each employee expects to be paid for the work they perform. While the
amount of maximum remuneration that an employee receives for their work is
not legislated by any government (unless the employee is a federal or
provincial/territorial civil servant), there is legislation in place at
both the federal and provincial/territorial levels that governs many
aspects of processing employees’ pay, their taxable benefits and observing
their rights as employees.

\sphinxAtStartPar
It is important to note that this course deals with payroll, the function of paying employees
for work performed for employers. Self\sphinxhyphen{}employed workers or contractors, who submit
invoices for the work they perform and receive payment through accounts payable and not
payroll, are not employees.

\sphinxAtStartPar
Both the federal and the Québec governments provide factors that can be used to determine
whether an employee\sphinxhyphen{}employer relationship exists. It is crucial to know how to determine the
type of relationship that exists between the worker and the organization and to ensure that
any payments made comply with legislation.


\chapter{Payroll Objectives and Definitions}
\label{\detokenize{1_introduction:payroll-objectives-and-definitions}}
\sphinxAtStartPar
The \sphinxstylestrong{primary objective of the payroll function} in every organization is to
pay employees accurately and on time, in compliance with legislative
requirements, for a full annual payroll cycle.

\sphinxAtStartPar
Every employee expects to receive their pay on the day it is due in the
manner arranged with their employer, either by cheque or direct deposit.
In addition to ensuring that employees have been paid, payroll
practitioners must also be able to communicate payroll information to
all stakeholders.
\begin{itemize}
\item {} 
\sphinxAtStartPar
\sphinxstylestrong{Payroll} is the process of paying employees in exchange for the services they perform.

\item {} 
\sphinxAtStartPar
\sphinxstylestrong{Legislation} refers to laws enacted by a legislative body. In Canada

\end{itemize}

\sphinxAtStartPar
there are many legislative sources that payroll practitioners must comply
with at two separate levels: the federal and the provincial/territorial
governments. Later in the chapter we will explore the compliance
requirements for the various pieces of legislation from these sources.
\begin{itemize}
\item {} 
\sphinxAtStartPar
\sphinxstylestrong{Compliance} is the observance of official requirements. For payroll

\end{itemize}

\sphinxAtStartPar
practitioners, this means performing payroll functions according to
federal and provincial/territorial legislative and non\sphinxhyphen{}governmental
stakeholder requirements.

\sphinxAtStartPar
The legislative requirements are termed \sphinxstylestrong{statutory}. This means they are
enacted, created, or regulated by statute, a law enacted by the legislative
branch of a government. Fines and penalties can be imposed if an organization
is not in compliance with the legislative requirements in each jurisdiction.

\sphinxAtStartPar
When dealing with federal and provincial/territorial government agencies,
payroll administrator must know the many pieces of legislation that
regulate their work and the compliance requirements associated with each.
Payroll administrators are responsible for ensuring their organization is
compliant with all payroll related legislation, thus eliminating the
potential for any fines or penalties.

\sphinxAtStartPar
In payroll, there are also compliance requirements from other non\sphinxhyphen{}government stakeholders,
for example, union collective agreements or group insurance policies. Payroll administrator
must therefore ensure the organization is compliant with all stakeholder requirements.

\sphinxAtStartPar
The responsibilities of the payroll administrator will differ depending on
the size of the organization, the number of jurisdictions in which they pay,
the reporting structure under which they work, and whether there are other
related departments, such as human resources, finance and administration in
the organization.

\sphinxAtStartPar
Small and medium\sphinxhyphen{}sized organizations may have payroll administrators whose
positions include other functions that, in a larger organization, would
fall under other departments. This payroll practitioner may be required to
handle multiple tasks, such as employee recruitment, human resource policy
development, benefits administration, accounts payable, accounts receivable,
budgets and/or administration.

\sphinxAtStartPar
Larger organizations may have a distinct payroll department with specific
payroll positions, in addition to separate human resources, accounting and
administration groups. Even in a multi\sphinxhyphen{}departmental organization, payroll
administrators must have knowledge of the various stages of the life cycle
of an employee. From hiring through termination of employment, many of
these stages will impact how to produce the employee’s pay and prepare
required reports.

\sphinxAtStartPar
The payroll department in a large organization may have:
\begin{itemize}
\item {} 
\sphinxAtStartPar
payroll administrators who are responsible for entering payroll data into the system and making required payroll remittances

\item {} 
\sphinxAtStartPar
payroll coordinators who are responsible for preparing the payroll journal entries and reconciling the payroll related accounts

\item {} 
\sphinxAtStartPar
payroll managers who manage the payroll function, the payroll staff and represent payroll at the management level

\end{itemize}


\section{Legislation vs. regulation}
\label{\detokenize{1_introduction:legislation-vs-regulation}}
\sphinxAtStartPar
\sphinxstylestrong{The legislation} specifies the \sphinxstyleemphasis{requirements}.

\sphinxAtStartPar
\sphinxstylestrong{The regulation} specifies the \sphinxstyleemphasis{methods} of applying the legislation.


\section{Payroll Content Knowledge}
\label{\detokenize{1_introduction:payroll-content-knowledge}}
\sphinxAtStartPar
Payroll administrators should know the following to effectively perform
their duties:
\begin{itemize}
\item {} 
\sphinxAtStartPar
\sphinxstylestrong{Payroll Compliance Legislation:} the Income Tax Act, the Employment Insurance Act, the Canada Pension Plan Act, Employment/Labour Standards, privacy legislation, Workers’ Compensation and provincial/territorial payroll\sphinxhyphen{}specific legislation

\item {} 
\sphinxAtStartPar
\sphinxstylestrong{Payroll Processes:} the remuneration and deduction components of payroll and how to use these components to calculate a net pay in both regular and non\sphinxhyphen{}regular circumstances

\item {} 
\sphinxAtStartPar
\sphinxstylestrong{Payroll Reporting:} how to calculate and remit amounts due to government agencies, insurance companies, unions and other third parties.

\end{itemize}

\sphinxAtStartPar
In addition, payroll reporting includes accounting for payroll expenses and accruals to internal financial systems and federal and provincial/territorial year\sphinxhyphen{}end reporting.


\section{Technical Skills}
\label{\detokenize{1_introduction:technical-skills}}
\sphinxAtStartPar
The technical skills required by payroll professionals include proficiency in
computer programs such as payroll software and financial systems,
spreadsheets, databases and word processing.

\sphinxAtStartPar
Organizations often change their payroll and business systems to meet new
technology requirements and corporate reporting needs. It is important for
payroll personnel to have the ability to be adaptable to changing systems.
As a payroll practitioner, you must be prepared and willing to embrace
continuous learning.


\section{Personal and Professional Skills}
\label{\detokenize{1_introduction:personal-and-professional-skills}}
\sphinxAtStartPar
The following personal and professional skills will assist payroll
administrators in dealing with the various stakeholders involved in the
payroll process:
\begin{itemize}
\item {} 
\sphinxAtStartPar
written communication skills, such as preparing employee emails and memos, management reports, policies and procedures and correspondence with various levels of government

\item {} 
\sphinxAtStartPar
verbal communication skills, to be able to respond to internal and external stakeholder inquiries

\item {} 
\sphinxAtStartPar
the ability to read, understand and interpret legal terminology found in documents such as collective agreements, benefit contracts and government regulations

\item {} 
\sphinxAtStartPar
excellent mathematical skills to perform various calculations

\item {} 
\sphinxAtStartPar
problem solving, decision\sphinxhyphen{}making, time management and organizational skills

\end{itemize}


\section{Behavioural and Ethical Standards}
\label{\detokenize{1_introduction:behavioural-and-ethical-standards}}
\sphinxAtStartPar
Behaviour and ethics are two areas that build on the skills that an effective payroll practitioner
must have. Effective payroll professionals should be:
\begin{quote}
\begin{itemize}
\item {} 
\sphinxAtStartPar
trustworthy, as the potential for fraud is ever present

\item {} 
\sphinxAtStartPar
conscientious, with a keen attention to detail

\item {} 
\sphinxAtStartPar
discreet, due to the confidential nature of information being handled

\item {} 
\sphinxAtStartPar
tactful in dealing with employees who can be very sensitive when

\end{itemize}

\sphinxAtStartPar
discussing their financial issues
\begin{itemize}
\item {} 
\sphinxAtStartPar
perceptive, able to understand all sides of an issue

\item {} 
\sphinxAtStartPar
able to work under the pressures of absolute deadlines

\item {} 
\sphinxAtStartPar
able to use common sense in order to recognize problems quickly and

\end{itemize}

\sphinxAtStartPar
apply sound solutions
\begin{itemize}
\item {} 
\sphinxAtStartPar
able to remain objective and maintain a factual perspective when

\end{itemize}

\sphinxAtStartPar
dealing with questions and inquiries
\end{quote}


\chapter{Payroll Stakeholders}
\label{\detokenize{1_introduction:payroll-stakeholders}}
\sphinxAtStartPar
\sphinxstylestrong{Stakeholders} are the individuals, groups and agencies, both internal
and external to the organization, who share an interest in the function and
output of the payroll department. Stakeholders can be considered customers of the payroll department and payroll practitioners
can take a proactive customer service approach to serving these individuals and groups.


\section{Payroll Management Stakeholders}
\label{\detokenize{1_introduction:payroll-management-stakeholders}}
\sphinxAtStartPar
\sphinxstylestrong{Payroll Management Stakeholders} are the federal and provincial/territorial governments,
the internal stakeholders and the external stakeholders. Internal stakeholders
include employees, employers and other departments within the organization.
External stakeholders include benefit carriers, courts, unions, pension
providers, charities, third party administrators and outsource/software vendors.


\section{Government Stakeholders}
\label{\detokenize{1_introduction:government-stakeholders}}
\sphinxAtStartPar
Government legislation provides the rules and regulations that the payroll function must
administer with respect to payments made to employees. For this reason, it is important for
the payroll practitioner to understand both the scope and the source of payroll\sphinxhyphen{}related
legislation.

\sphinxAtStartPar
Canada is ruled by a federal government with ten largely self\sphinxhyphen{}governing provinces and three
territories controlled by the federal government. Payroll practitioners have to be compliant
not only with the federal government legislation, but with the provincial and territorial
governments’ legislation as well.

\sphinxAtStartPar
As a result, payroll practitioners and their organizations are affected by the enactment of
legislation at both the federal and provincial/territorial level.
The federal parliament has the power to make laws for the peace, order and good government
of Canada. The federal cabinet is responsible for most of the legislation introduced by
parliament, and has the sole power to prepare and introduce tax legislation involving the
expenditure of public money.

\sphinxAtStartPar
The provincial/territorial legislatures have power over direct taxation in the province or
territory for the purposes of natural resources, prisons (except for federal penitentiaries),
charitable institutions, hospitals (except marine hospitals), municipal institutions, education,
licences for provincial/territorial and municipal revenue purposes, local works, incorporation
of provincial/territorial organizations, the creation of courts and the administration of justice,
fines and penalties for breaking provincial/territorial laws.

\sphinxAtStartPar
In the case of old age, disability, and survivor’s pensions, again both the federal and
provincial/territorial governments have power. In this instance, if their laws conflict, the
provincial/territorial power prevails.

\sphinxAtStartPar
The federal government cannot transfer any of its powers to a provincial/territorial
legislature, nor can a provincial/territorial legislature transfer any of its powers to the federal
government. The federal government can, however, delegate the administration of a federal
act to a provincial/territorial agency, and a provincial/territorial legislature can delegate the
administration of a provincial/territorial act to a federal agency.


\section{Federal Government}
\label{\detokenize{1_introduction:federal-government}}
\sphinxAtStartPar
The \sphinxstyleemphasis{Constitution Act of 1867} outlined the division of legislative power and authority between
federal and provincial/territorial jurisdictional governments. The exclusive legislative
authority of the Parliament of Canada extends to all matters regarding:
\begin{itemize}
\item {} 
\sphinxAtStartPar
the regulation of trade and commerce

\item {} 
\sphinxAtStartPar
the raising of money by any mode or system of taxation

\item {} 
\sphinxAtStartPar
the borrowing of money on the public credit

\item {} 
\sphinxAtStartPar
the postal service

\item {} 
\sphinxAtStartPar
fixing and providing salaries and allowances for civil and other officers of the Government of Canada

\item {} 
\sphinxAtStartPar
navigation and shipping

\item {} 
\sphinxAtStartPar
ferries between a province and any British or foreign country or between two provinces

\item {} 
\sphinxAtStartPar
criminal law, except the Constitution of Courts of Criminal Jurisdiction, but including the Procedure in Criminal Matters

\item {} 
\sphinxAtStartPar
anything not specifically assigned to the provinces under this Act

\end{itemize}

\sphinxAtStartPar
The Canada Labour Code is legislation that consolidates certain statutes respecting labour.
Part I deals with Industrial Relations, Part II deals with Occupational Health and Safety and
Part III deals with Labour Standards. The primary objective of Part III is to establish and
protect employees’ and employers’ rights to fair and equitable conditions of employment.

\sphinxAtStartPar
Part III provisions establish minimum requirements concerning the working conditions of
employees under federal jurisdiction in the following industries and organizations:
\begin{itemize}
\item {} 
\sphinxAtStartPar
industries and undertakings of inter\sphinxhyphen{}provincial/territorial, national, or international nature, that is, transportation, communications, radio and television broadcasting, banking, uranium mining, grain elevators, and flour and feed operations

\item {} 
\sphinxAtStartPar
organizations whose operations have been declared for the general advantage of Canada or two or more provinces, and such Crown corporations as Canada Post Corporation, and the Canadian Broadcasting Corporation (CBC)

\end{itemize}


\section{Provincial/Territorial Governments}
\label{\detokenize{1_introduction:provincial-territorial-governments}}
\sphinxAtStartPar
Under the \sphinxstyleemphasis{Constitution Act of 1867}, the exclusive legislative authority of the provinces and
territories exists over:
\begin{itemize}
\item {} 
\sphinxAtStartPar
all laws regarding property and civil rights, which give the provinces/territories the authority to enact legislation to establish employment standards for working conditions

\item {} 
\sphinxAtStartPar
employment in manufacturing, mining, construction, wholesale and retail trade, service industries, local businesses and any industry or occupation not specifically covered under federal jurisdiction

\end{itemize}

\sphinxAtStartPar
The existing divisions between federal and provincial/territorial control impact payroll when
dealing with employment/labour standards. Employment/labour standards are rules legislated
by each provincial/territorial jurisdiction that dictate issues such as hours of work, minimum
wage, overtime, vacation pay and termination pay requirements.

\sphinxAtStartPar
Employers must follow the employment/labour standards legislated by the jurisdiction in
which their employees work, unless they are governed by federal labour standards. Federal
labour standards apply to certain industries and organizations, regardless of where the
employees work.

\sphinxAtStartPar
The person or persons performing the payroll function must clearly understand under which
employment/labour standards jurisdiction the employees of the organization fall.
Organizations may have some employees who fall under federal jurisdiction and another
group of employees who fall under provincial/territorial legislation.


\section{Internal Stakeholders}
\label{\detokenize{1_introduction:internal-stakeholders}}
\sphinxAtStartPar
Internal stakeholders are those individuals or departments closely related to the organization
that the payroll department is serving. This group includes employers, employees and other
departments in the organization.

\sphinxAtStartPar
Employers \sphinxhyphen{} Management may require certain information from payroll to make sound
business decisions.

\sphinxAtStartPar
Employees \sphinxhyphen{} Employees require that their pay is received in a timely and accurate manner to
meet personal obligations. Employees must also be assured that their personal information is
kept confidential.

\sphinxAtStartPar
Other departments \sphinxhyphen{} Many departments interact with payroll, either for information or
reporting. According to the Canadian Payroll Association’s 2020 National Payroll Week
(NPW) Payroll Professional Research Survey, fifty\sphinxhyphen{}five percent of payroll practitioners
report through the finance department and thirty\sphinxhyphen{}two percent report through the human
resources department.

\sphinxAtStartPar
Information such as general ledger posting, payroll and benefit costs
and salary information must flow between payroll, human resources and finance in formats
needed for their various requirements.

\sphinxAtStartPar
In addition, other departments such as contracts and manufacturing often need payroll
information for budgeting, analytical and quality purposes.


\section{External Stakeholders}
\label{\detokenize{1_introduction:external-stakeholders}}
\sphinxAtStartPar
External stakeholders are organizations that are neither government nor internal stakeholders,
yet have a close working relationship with the payroll function. Compliance with external
stakeholder requirements is also a responsibility of the payroll department. In most cases,
compliance will require that payroll request a cheque from accounts payable and send it to
the external organization along with supporting documentation.

\sphinxAtStartPar
Benefit Carriers are insurance companies that provide benefit coverage to employees.
Payroll is responsible for deducting and remitting premiums for the insurance coverage to the
carriers and for providing reports on employee enrolment and coverage levels.

\sphinxAtStartPar
Courts and the CRA require payroll to accurately deduct and remit amounts ordered to be
withheld through garnishments, third party demands, requirements to pay and support
deduction orders.

\sphinxAtStartPar
\sphinxstylestrong{Unions} require that payroll accurately deduct and remit union dues and initiation fees, and to
ensure that the terms of the collective agreement are adhered to. It is estimated that just under
one\sphinxhyphen{}third of the workforce in Canada belongs to a trade union. Payroll professionals must be
familiar with the role and activities of trade unions and the responsibilities of the employer
and the payroll department in a unionized environment.

\sphinxAtStartPar
Trade unions negotiate with the employer, through collective bargaining, the wages, benefits,
allowances and other terms and conditions of employment on behalf of their member
employees. The outcome of negotiations is a collective agreement, which is a legally binding
contract between the employer, the union and the employees.

\sphinxAtStartPar
\sphinxstylestrong{Pension Providers} are third party pension plan providers that may require payroll to provide
enrolment reports on participating employees and length of service calculations, and to remit
employee deductions and employer contributions.

\sphinxAtStartPar
\sphinxstylestrong{Charities} have arrangements with some organizations to facilitate employee donations
through payroll deductions. Payroll is responsible for remitting these deductions to the
charity.

\sphinxAtStartPar
\sphinxstylestrong{Third Party Administrators} are organizations that affect the administration of the payroll
function. Examples of these external stakeholders are banking institutions or benefit
organizations that offer Group Registered Retirement Saving Plans (RRSP). Payroll is
responsible for deducting any employee contributions and remitting employer and employee
contributions to the plan administrator.

\sphinxAtStartPar
\sphinxstylestrong{Outsource/Software vendors} are payroll service providers or payroll software vendors that
work with the payroll department to ensure the payroll is being processed accurately and
efficiently.


\chapter{Content Review}
\label{\detokenize{1_introduction:content-review}}\begin{itemize}
\item {} 
\sphinxAtStartPar
The primary objective of the payroll function in every organization is to pay employees accurately and on time, in compliance with legislative requirements, for a full annual payroll cycle.

\item {} 
\sphinxAtStartPar
Payroll is the process of paying employees in exchange for the services they perform.

\item {} 
\sphinxAtStartPar
Legislation refers to laws enacted by a legislative body.

\item {} 
\sphinxAtStartPar
Compliance is the observance of official requirements.

\item {} 
\sphinxAtStartPar
Payroll practitioner knowledge consists of information on payroll compliance legislation, payroll processes and payroll reporting as well as technical, personal and professional skills.

\item {} 
\sphinxAtStartPar
Stakeholders are the individuals, groups and agencies, both internal and external to the organization, who share an interest in the function and output of the payroll department.

\item {} 
\sphinxAtStartPar
Payroll management stakeholders are the federal and provincial/territorial governments, the internal stakeholders, and the external stakeholders. Internal stakeholders include employees, employers and other departments within the organization.

\item {} 
\sphinxAtStartPar
External stakeholders include benefit carriers, courts, unions, pension providers, charities, third party administrators and outsource/software vendors.

\item {} 
\sphinxAtStartPar
The federal parliament has the power to make laws for the peace, order and good government of Canada.

\item {} 
\sphinxAtStartPar
The provincial/territorial legislatures have power over direct taxation in the province/territory for provincial/territorial purposes.

\item {} 
\sphinxAtStartPar
Federal control exists over industries and undertakings of inter\sphinxhyphen{}provincial/territorial, national, or international nature and organizations whose operations have been declared for the general advantage of Canada or two or more provinces and Crown corporations.

\item {} 
\sphinxAtStartPar
Provincial/territorial legislation exists over all laws regarding property and civil rights, and employment in manufacturing, mining, construction, wholesale and retail trade, service industries, local businesses and any industry or occupation not specifically covered under federal jurisdiction.

\item {} 
\sphinxAtStartPar
Employers must follow the employment/labour standards legislated by the jurisdiction in which their employees work, unless they are governed by federal labour standards.

\item {} 
\sphinxAtStartPar
Where legislation requires employer compliance, there are financial penalties or the possibility of legal action to encourage compliance.

\end{itemize}


\chapter{Review Questions}
\label{\detokenize{1_introduction:review-questions}}\begin{enumerate}
\sphinxsetlistlabels{\arabic}{enumi}{enumii}{}{.}%
\item {} 
\sphinxAtStartPar
What is the primary objective of the payroll department?

\item {} 
\sphinxAtStartPar
List the three types of payroll management stakeholders and provide an example of each.

\item {} 
\sphinxAtStartPar
Explain the difference between legislation and regulation.

\item {} 
\sphinxAtStartPar
List three external stakeholders and explain their compliance requirements.

\end{enumerate}

\sphinxstepscope


\chapter{PAYROLL COMPLIANCE}
\label{\detokenize{2_compliance:payroll-compliance}}\label{\detokenize{2_compliance::doc}}

\section{The Canada Revenue Agency}
\label{\detokenize{2_compliance:the-canada-revenue-agency}}
\sphinxAtStartPar
Under the Canada Pension Plan Act and the Employment Insurance Act, the CRA is
responsible for determining:
\begin{itemize}
\item {} 
\sphinxAtStartPar
whether or not an individual’s employment is pensionable under the Canada Pension

\end{itemize}

\sphinxAtStartPar
Plan Act or insurable under the Employment Insurance Act
\begin{itemize}
\item {} 
\sphinxAtStartPar
the types of earnings that are considered pensionable or insurable

\item {} 
\sphinxAtStartPar
how many hours an insured person has in insurable employment

\item {} 
\sphinxAtStartPar
the recovery of any debts owed as a result of an overpayment of Canada Pension

\end{itemize}

\sphinxAtStartPar
Plan, Employment Insurance, or Old Age Security benefits

\sphinxAtStartPar
The CRA is also responsible for ensuring that CPP contributions and EI premiums are
deducted, remitted and reported as required by legislation.

\sphinxAtStartPar
The Canada Revenue Agency (CRA) is a federal government agency that manages the
following business lines for the federal government: Tax Services and Benefit Programs.
The Tax Services business line assists over 25 million individuals, businesses, trusts, and
organizations to meet their obligations under the tax system. Each year, the CRA collects
approximately \$324 billion in gross taxes and excise duties on behalf of the federal and
provincial governments \sphinxhyphen{} the equivalent of about \$1.2 billion every working day. The CRA’s
mission is to promote compliance with Canada’s tax legislation and regulations through
communication, quality service, and responsible enforcement, thereby contributing to the
economic and social well\sphinxhyphen{}being of Canadians.

\sphinxAtStartPar
From this mission comes the CRA’s mandate to:
\begin{itemize}
\item {} 
\sphinxAtStartPar
collect revenues and administer tax laws for the federal government and for most provinces and territories

\item {} 
\sphinxAtStartPar
deliver various social and economic benefit incentive programs to Canadians

\end{itemize}

\sphinxAtStartPar
The CRA tracks the success of the first part of its mandate by measuring compliance in the following areas:
\begin{itemize}
\item {} 
\sphinxAtStartPar
\sphinxstylestrong{Filing:} the CRA’s goal is to have over 90\% of individual and corporate income tax and registered business’ goods and services tax/harmonized sales tax (GST/HST) returns filed on time.

\item {} 
\sphinxAtStartPar
\sphinxstylestrong{Registration:} the CRA measures its success in this area by ensuring that the majority of all known businesses are registered for various programs including corporate income tax, GST/HST, payroll deductions, and import/export accounts.

\item {} 
\sphinxAtStartPar
\sphinxstylestrong{Remittance:} the CRA’s goal is to have over 90\% of individual and corporate tax filers pay their taxes on time.

\item {} 
\sphinxAtStartPar
\sphinxstylestrong{Reporting:} the CRA measures reporting compliance through the information it receives on tax documents, for example, the T4 and T4A information slips.

\end{itemize}

\sphinxAtStartPar
The CRA’s program responsibilities that are specifically related to payroll include the administration of:
\begin{itemize}
\item {} 
\sphinxAtStartPar
the Canada Pension Plan (CPP) (shared responsibility with Employment and Social Development Canada and Service Canada)

\item {} 
\sphinxAtStartPar
Employment Insurance (EI) (shared responsibility with Employment and Social Development Canada and Service Canada)

\item {} 
\sphinxAtStartPar
Income Tax

\end{itemize}

\sphinxAtStartPar
Each of these programs requires compliance by employers to withhold deductions from their
employees’ pay for CPP contributions, EI premiums and income tax deductions. These
withholdings are termed statutory deductions as the deductions are required under legislative
statute. Statutory deductions are the first deductions to be withheld from an employee’s gross
pay.


\subsection{Canada Pension Plan (CPP)}
\label{\detokenize{2_compliance:canada-pension-plan-cpp}}
\sphinxAtStartPar
The Canada Pension Plan became operational on January 1, 1966. The plan was fully
effective in 1976 after a ten year transitional period.
The Canada Pension Plan is a social insurance program, legislated under the federal Canada
Pension Plan Act, designed to provide protection in the form of benefits to contributors and
their families against loss of income due to retirement.

\sphinxAtStartPar
In addition to retirement pension benefits, the plan provides supplementary benefits in the form of:
\begin{itemize}
\item {} 
\sphinxAtStartPar
surviving spouse pensions

\item {} 
\sphinxAtStartPar
disability benefits

\item {} 
\sphinxAtStartPar
benefits for orphans and children of disabled contributors

\item {} 
\sphinxAtStartPar
death benefits payable upon the death of a contributor

\end{itemize}

\sphinxAtStartPar
All employers are required by law to deduct CPP contributions from pensionable earnings
paid to their employees, and to remit these deductions, along with the employer’s portion, to
the CRA. The employer matches the employee’s contributions dollar for dollar.

\begin{sphinxadmonition}{note}{EXAMPLE}

\sphinxAtStartPar
Janet Frank has \$45.00 in CPP contributions deducted from her gross pay. Her employer,
Northern Skies, must match her contribution of \$45.00. A total of \$90.00 in CPP contributions must be remitted to the CRA.
\end{sphinxadmonition}

\sphinxAtStartPar
CPP contributions take priority over all other deductions and are therefore the first statutory
deduction to be withheld from an employee’s gross pay.


\subsection{Employment Insurance (EI)}
\label{\detokenize{2_compliance:employment-insurance-ei}}
\sphinxAtStartPar
The CRA’s responsibility for the Employment Insurance program is associated with the
collection of employee and employer premiums. It also makes decisions about which types of
remuneration are considered insurable and, therefore, subject to EI premiums.
All employers are required by law to deduct EI premiums from the insurable earnings paid to
their employees, and remit these deductions, along with the employer’s portion, to the CRA.
The employer’s portion is 1.4 times the employee’s portion.

\begin{sphinxadmonition}{note}{EXAMPLE}

\sphinxAtStartPar
Janet Frank has \$20.00 in EI premiums deducted from her gross pay. Her employer, Northern
Skies must contribute \$28.00 (\$20.00 x 1.4). A total of \$48.00 in EI premiums must be
remitted to the CRA.
\end{sphinxadmonition}

\sphinxAtStartPar
EI premiums are the second statutory deduction to be withheld from an employee’s pay.
Employers are also required to track the employee’s insurable earnings and insurable hours
by pay period for reporting purposes, such as completing the Record of Employment for a
terminated or inactive employee.


\subsection{Income Tax}
\label{\detokenize{2_compliance:income-tax}}
\sphinxAtStartPar
Income taxation began in Canada, and in many other countries, during World War I. In July
1917, the Government of Canada passed legislation which enabled the government to levy a
temporary tax on personal income. This tax was intended to help finance government
expenditures for World War I; however, it eventually became the basic tax on all incomes.

\sphinxAtStartPar
When income tax was first introduced, each person was responsible for paying their own
income tax directly to the federal government. In 1940, the federal government legislated
deductions at source, which meant that employers became responsible for withholding
income tax from remuneration paid to employees. Beginning January 1, 1962, all provinces
imposed personal income tax; prior to that date, only Québec imposed such a tax.

\sphinxAtStartPar
Income tax withholdings are calculated by applying a federal tax rate and a separate
provincial/territorial tax rate to the employee’s taxable income. The employee’s province of
employment determines which provincial/territorial tax rate to apply. The federal
government and all provinces and territories, except Québec, have the same definition of
taxable income.

\sphinxAtStartPar
All Canadian provinces/territories, except Québec, have entered into tax collection
agreements with the federal government. Under these agreements the CRA collects the
provincial/territorial income taxes on behalf of the provinces/territories. The CRA then
distributes the provincial/territorial income taxes it has collected through a series of transfer
payments to the provinces/territories. These transfer payments are based on the personal tax
returns filed by Canadian taxpayers.

\sphinxAtStartPar
As the federal government collects both the federal and the provincial/territorial portions of
tax from all employees working in a province/territory other than Québec, the two tax
withholdings, federal and provincial/territorial, are combined into one deduction amount. The
employee may only see one item \sphinxstyleemphasis{Income Tax} or \sphinxstyleemphasis{Federal Income Tax} listed on their pay
statement, however it is the total of two withholdings.

\sphinxAtStartPar
Québec collects its own provincial income tax. There are two separate income tax deductions
withheld from Québec employees — one for federal income tax and the other for Québec
provincial income tax. The federal income tax is remitted to the CRA and the Québec
provincial income tax is remitted to Revenu Québec (RQ). Québec employees will see
\sphinxstyleemphasis{Federal Income Tax} and \sphinxstyleemphasis{Québec Income Tax} listed separately on their pay statements.
RQ is discussed extensively in a later chapter.


\section{Non\sphinxhyphen{}Compliance Penalties}
\label{\detokenize{2_compliance:non-compliance-penalties}}
\sphinxAtStartPar
If an organization fails to deduct and remit the amounts withheld from employees for CPP
contributions, EI premiums and income tax, it may be left in the position of having to pay
both the employer’s and the employee’s portion of deductions not taken, as well as penalties
and interest charges on the outstanding amount.

\sphinxAtStartPar
An employer who remits withholdings or deductions late is subject to the following penalties:
\begin{itemize}
\item {} 
\sphinxAtStartPar
3\% will be applied to remittances that are 1 to 3 days late

\item {} 
\sphinxAtStartPar
5\% for remittances that are 4 or 5 days late

\item {} 
\sphinxAtStartPar
7\% for remittances that are 6 or 7 days late

\item {} 
\sphinxAtStartPar
10\% for remittances that are 8 or more days late

\end{itemize}

\sphinxAtStartPar
An employer who withholds the statutory deductions but does not remit them, or fails to
deduct the required deductions, will be subject to a 10\% penalty for the first occurrence on
the amount that should have been deducted and remitted. This penalty may increase to 20\%
for the second and each subsequent occurrence in the same calendar year if the failure was
made knowingly or under circumstances of gross negligence. Penalties will be applied to
amounts in excess of \$500; however, in the case of wilful delay or deficiency, these penalties
can be levied on amounts of less than \$500.

\sphinxAtStartPar
The Canada Revenue Agency (CRA) charges interest on any unpaid remittances and unpaid
penalties from the day the payment was due. The interest rate is determined every three
months, in accordance with the prescribed interest rates, and is available on the CRA
website.

\sphinxAtStartPar
As a payroll practitioner, you need to have a clear understanding of how and when to make
the required deductions and remittances to avoid these penalties and interest charges.

\sphinxAtStartPar
All monies deducted on behalf of the CRA are considered to be held “in trust” for the
Receiver General. The amount owed must be kept separate from the operating funds of the
organization. In the event of estate liquidation, assignment, receivership, or bankruptcy the
trust money for statutory deductions is still owed to the CRA.


\section{Employment and Social Development Canada (ESDC)}
\label{\detokenize{2_compliance:employment-and-social-development-canada-esdc}}
\sphinxAtStartPar
Employment and Social Development Canada (ESDC), a department of the Government of
Canada, is committed to building a stronger and more competitive Canada by supporting
Canadians in making choices that help them live productive and rewarding lives and to
improve their quality of life.

\sphinxAtStartPar
To do this, the department:
\begin{itemize}
\item {} 
\sphinxAtStartPar
develops policies that make Canada a society in which all can use their talents, skills and resources to participate in learning, work and their community

\item {} 
\sphinxAtStartPar
creates programs and support initiatives that help Canadians move through life’s transitions—from families with children to seniors, from school to work, from one job to another, from unemployment to employment, from the workforce to retirement

\item {} 
\sphinxAtStartPar
creates better outcomes for Canadians through service excellence with Service Canada and other partners

\end{itemize}

\sphinxAtStartPar
ESDC supports human capital development and labour market development; and is dedicated
to establishing a culture of lifelong learning for Canadians. Some of their specific program
responsibilities include:
\begin{itemize}
\item {} 
\sphinxAtStartPar
Canada Pension Plan and Old Age Security

\item {} 
\sphinxAtStartPar
Employment Insurance

\item {} 
\sphinxAtStartPar
Employment Programs

\item {} 
\sphinxAtStartPar
Youth Employment Strategies

\item {} 
\sphinxAtStartPar
Canada Education Savings Program

\item {} 
\sphinxAtStartPar
Canada Student Loans and Grants

\end{itemize}

\sphinxAtStartPar
ESDC is responsible for matters relating to:
\begin{itemize}
\item {} 
\sphinxAtStartPar
amending the regulations made under the Canada Pension Plan and the Employment \sphinxstyleemphasis{Insurance Act}

\item {} 
\sphinxAtStartPar
keeping records of each individual’s CPP contributions and pensionable earnings

\item {} 
\sphinxAtStartPar
the establishment of annual maximum insurable earnings

\item {} 
\sphinxAtStartPar
the administration of provisions related to Wage Loss plans

\item {} 
\sphinxAtStartPar
the administration of provisions regarding Job Creation programs

\end{itemize}

\sphinxAtStartPar
Almost all of today’s seniors receive income from Canada’s public pensions. Basic financial
support is also available to survivors, people who become too disabled to work, and their
children. These pensions and benefits are delivered through the Canada Pension Plan (CPP)
and Old Age Security (OAS) programs. Together, the CPP and OAS programs provide a
modest base upon which Canadians can build their retirement income.

\sphinxAtStartPar
The amount of CPP benefits is based on an individual’s CPP contributions. Employees
between the ages of 18 to 70 years old make contributions that are calculated on their annual
pensionable earnings between a minimum and a maximum amount. The minimum amount is
frozen at \$3,500, while the maximum pensionable earnings are set each January, based on
increases in the average wage in Canada.

\sphinxAtStartPar
Employment Insurance (EI) is the program with the greatest impact on payroll. This program
provides temporary financial assistance for unemployed Canadians while they look for work
or upgrade their skills. It also provides coverage for Canadians who are sick, pregnant or
caring for a newborn or adopted child. Individuals who must care for a family member who
is seriously ill with a significant risk of death may also be assisted by Employment Insurance
benefits. Application of the EI rules will be looked at in more detail in another chapter.
The first Unemployment Insurance (UI) Act was passed into law in 1940, and was based on
the British Unemployment Insurance Act, 1935. Since that time, the UI Act has been repealed
and replaced four times \sphinxhyphen{} in 1955, 1971, 1985, and most recently in 1996. Clarifying details
on how the act is to be applied are found in the EI Regulations, which are amended as
required.

\sphinxAtStartPar
The purpose of the act is to provide income support during a temporary interruption of
earnings with the emphasis on returning the unemployed to the labour force as quickly as
possible. Contributions to the plan and the amount of benefits are based on a percentage of
insurable earnings. The ceiling on insurable earnings is reviewed annually to keep pace with
increases in average income and the cost of living. The premium rates payable by an insured
employee and the employer are also determined annually.


\section{Service Canada}
\label{\detokenize{2_compliance:service-canada}}
\sphinxAtStartPar
Service Canada (SC) was created by the federal government in 2005 with the goal of
providing easy\sphinxhyphen{}to\sphinxhyphen{}access, one\sphinxhyphen{}stop personalized service to Canadians. The agency integrates
services from a number of federal departments to form a service delivery network. These
services often touch all aspects of the lives of Canadians: from parental and pension benefits,
to matching employers with job seekers, and obtaining a Social Insurance Number.
Service Canada serves as the government’s operational arm while Employment and Social
Development Canada (ESDC) operates as the policy\sphinxhyphen{}making body. ESDC makes the rules
for the various programs while Service Canada delivers the programs.

\sphinxAtStartPar
Some of Service Canada’s program responsibilities include:
\begin{itemize}
\item {} 
\sphinxAtStartPar
the issuance of Social Insurance Numbers (SIN) and the protection and security of SIN information

\item {} 
\sphinxAtStartPar
the delivery of services to employers, including Record of Employment on the Web the administration of Employment Insurance programs to individuals, including regular, illness, pregnancy/parental, critically ill or injured person and compassionate care benefits

\item {} 
\sphinxAtStartPar
the administration of the Employment Insurance Premium Reduction program, including granting qualified employers a reduced Employment Insurance premium rate

\item {} 
\sphinxAtStartPar
the administration of Canada Pension Plan benefits, including retirement, disability, survivor, children’s and death benefits

\item {} 
\sphinxAtStartPar
the administration of benefits for seniors, including the Old Age Security Program and the Guaranteed Income Supplement

\end{itemize}

\sphinxAtStartPar
Through the Canada Pension Plan program, SC administers the payment of CPP benefits. These payments have little impact on payroll.
The biggest impact Service Canada has on
payroll is through the administration of the Record of Employment program and the Social
Insurance Number.

\sphinxAtStartPar
Payroll is responsible for deducting and remitting EI premiums on behalf of employees and
employers. They are also responsible for capturing information related to insurable earnings
and hours, and reporting that information on the Record of Employment, which is explained
in detail in a later chapter.


\subsection{Social Insurance Number (SIN)}
\label{\detokenize{2_compliance:social-insurance-number-sin}}
\sphinxAtStartPar
The Social Insurance Number (SIN) Program was introduced by Parliament in 1964 to
register people with the Unemployment Insurance Commission (now known as Employment
Insurance) and the Canada Pension Plan. In 1967, the SIN also became a file identifier for
Revenue Canada (now known as the Canada Revenue Agency).

\sphinxAtStartPar
Under the Employment Insurance Act, every person who works in Canada is required to have
a Social Insurance Number. As an employer, you must ask new employees to provide their
SIN when they are hired. According to the Employment Insurance Regulations which came
into force on April 30, 2013, employees are required by law to provide their SIN; they may
do so by presenting a SIN card or the Confirmation of SIN letter. It is important that you
obtain the correct SIN of your employee so that payroll can report accurate statutory
withholdings for Canada Pension Plan contributions, Employment Insurance premiums, and
income tax on the employee’s information slips at the end of the year.

\sphinxAtStartPar
To apply for a Social Insurance Number, individuals must complete an application form that
can either be obtained from a SC office or downloaded from their website. Documents
proving the individual’s identity and status in Canada must also be submitted with the
application. The documents must be originals and written in English or French.

\sphinxAtStartPar
Social Insurance Numbers beginning with a “9” (commonly called 900\sphinxhyphen{}series) are issued to
individuals who are neither Canadian citizens nor permanent residents, but need a SIN for
employment purposes. All 900\sphinxhyphen{}series SIN cards/letters carry an expiry date that is the same
as the expiry date on the individual’s work permit. Individuals must apply for new
documentation prior to expiry. Employees who are not residents of Canada, who are in
regular continuous employment, and are in possession of a 900\sphinxhyphen{}series SIN, are subject to all
applicable statutory deductions.


\section{Statistics Canada}
\label{\detokenize{2_compliance:statistics-canada}}
\sphinxAtStartPar
Statistics Canada produces statistics that help Canadians better understand their country—its
population, resources, economy, society and culture.

\sphinxAtStartPar
In Canada, providing statistics is a federal responsibility. As Canada’s central statistical
agency, Statistics Canada is legislated under the Statistics Act to serve this function for the
whole of Canada and each of the provinces/territories.

\sphinxAtStartPar
Objective statistical information is vital to an open and democratic society. It provides a solid
foundation for informed decisions by elected representatives, businesses, unions and non\sphinxhyphen{}
profit organizations, as well as individual Canadians.

\sphinxAtStartPar
Statistics Canada is committed to protecting the confidentiality of all information entrusted to
them and to ensure that the information delivered is timely and relevant to Canadians.


\section{Personal Privacy}
\label{\detokenize{2_compliance:personal-privacy}}
\sphinxAtStartPar
The Canadian federal government and all provincial governments have legislation that sets
limits on the collection, use or disclosure of personal information. Private sector privacy laws
in Canada currently only cover the employee personal information of employees that work
for federally regulated companies or who are located in one of the four provinces with
provincial private sector privacy laws: Alberta, British Columbia, Manitoba and Québec.

\sphinxAtStartPar
Public sector employees have some privacy protection under all jurisdictions except Ontario
which excludes employee information from its public sector privacy legislation. Employees
who are covered by a collective agreement also have statutory privacy protection based on
arbitral jurisprudence and their particular collective agreement. Therefore, approximately
half of workers in Canada have privacy rights backed by legislation, while the remaining
50\% of the country’s more than 20 million or so workers have privacy rights that are either
voluntarily set in place by employers who have developed employee privacy codes or have
privacy rights because they have a collective agreement in place.

\sphinxAtStartPar
Employers should also be aware that egregious violations of privacy may open them up to
civil damages, including class action lawsuits. Legislatures and the courts are recognizing
privacy rights and providing opportunities for civil remedies.

\sphinxAtStartPar
In drawing up its legislation for the protection of personal information, the Canadian
government based its privacy provisions on a set of guidelines that had been developed by
the Canadian Standards Association in its Model Code for the Protection of Personal
Information.


\subsection{The Privacy Principles}
\label{\detokenize{2_compliance:the-privacy-principles}}
\sphinxAtStartPar
The Canadian Standards Association (CSA) Model Code is a set of principles that was
developed with input from organizations, governments, consumer associations and other
privacy stakeholders. They are incorporated in Federal private sector privacy legislation and
have become the generally accepted framework for evaluating privacy processes and systems
in Canada.


\subsubsection{Principle 1. Accountability}
\label{\detokenize{2_compliance:principle-1-accountability}}
\sphinxAtStartPar
An organization is responsible for personal information under its control and shall designate
an individual or individuals to be accountable for the organization’s compliance with the
following principles.


\subsubsection{Principle 2. Identifying Purposes}
\label{\detokenize{2_compliance:principle-2-identifying-purposes}}
\sphinxAtStartPar
The purposes for which personal information is collected shall be identified by the
organization at or before the time the information is collected.


\subsubsection{Principle 3. Consent}
\label{\detokenize{2_compliance:principle-3-consent}}
\sphinxAtStartPar
The knowledge and consent of the individual are required for the collection, use, or
disclosure of personal information, except where inappropriate. Note: In certain
circumstances, personal information can be collected, used, or disclosed without the
knowledge and consent of the individual. For example, legal, medical, or security reasons
may make it impossible or impractical to seek consent.


\subsubsection{Principle 4. Limiting Collection}
\label{\detokenize{2_compliance:principle-4-limiting-collection}}
\sphinxAtStartPar
The collection of personal information shall be limited to that which is necessary for the
purposes identified by the organization. Information shall be collected by fair and lawful
means.


\subsubsection{Principle 5. Limiting Use, Disclosure, and Retention}
\label{\detokenize{2_compliance:principle-5-limiting-use-disclosure-and-retention}}
\sphinxAtStartPar
Personal information shall not be used or disclosed for purposes other than those for which it
was collected, except with the consent of the individual or as required by law. Personal
information shall be retained only as long as is necessary for the fulfillment of those
purposes.


\subsubsection{Principle 6. Accuracy}
\label{\detokenize{2_compliance:principle-6-accuracy}}
\sphinxAtStartPar
Personal information shall be as accurate, complete, and up\sphinxhyphen{}to\sphinxhyphen{}date as is necessary for the
purposes for which it is to be used.


\subsubsection{Principle 7. Safeguards}
\label{\detokenize{2_compliance:principle-7-safeguards}}
\sphinxAtStartPar
Personal information shall be protected by security safeguards appropriate to the sensitivity
of the information.


\subsubsection{Principle 8. Openness}
\label{\detokenize{2_compliance:principle-8-openness}}
\sphinxAtStartPar
An organization shall make readily available to individuals specific information about its
policies and practices relating to the management of personal information.


\subsubsection{Principle 9. Individual Access}
\label{\detokenize{2_compliance:principle-9-individual-access}}
\sphinxAtStartPar
Upon request, an individual shall be informed of the existence, use and disclosure of his or
her personal information and shall be given access to that information. An individual shall be
able to challenge the accuracy and completeness of the information and have it amended as
appropriate. In certain situations, an organization may not be able to provide access to all the
personal information it holds about an individual. Exceptions to the access requirement
should be limited and specific. The reasons for denying access should be provided to the
individual upon request. Exceptions may include information that is prohibitively costly to
provide, information that contains references to other individuals, information that cannot be
disclosed for legal, security, or commercial proprietary reasons, and information that is
subject to solicitor\sphinxhyphen{}client or litigation privilege.


\subsubsection{Principle 10. Challenging Compliance}
\label{\detokenize{2_compliance:principle-10-challenging-compliance}}
\sphinxAtStartPar
An individual shall be able to address a challenge concerning compliance with the above
principles to the designated individual or individuals accountable for the organization’s
compliance.


\subsection{PIPEDA}
\label{\detokenize{2_compliance:pipeda}}
\sphinxAtStartPar
The federal government drew upon the CSA Privacy Principles in its drafting of the federal
Personal Information Protection and Electronic Documents Act (PIPEDA) and the spirit and
much of the wording of the principles can be found throughout PIPEDA.

\sphinxAtStartPar
The mandate of the Office of the Privacy Commissioner of Canada (OPC) is overseeing
compliance with both the Privacy Act, which covers the personal information\sphinxhyphen{}handling
practices of federal government departments and agencies (including employee data), and the
Personal Information Protection and Electronic Documents Act (PIPEDA), Canada’s private
sector privacy law.

\sphinxAtStartPar
PIPEDA has applied to federally regulated organizations such as banks, telecommunications
and transportation companies since January 2001 and applies to the collection, use or
disclosure of personal information in the course of any commercial activity within a province
that does not have its own privacy legislation, since January 2004.

\sphinxAtStartPar
While this protection of personal information legislation has a significant impact on how
organizations collect, use and disclose personal information relating to commercial
transactions (for example, customer/client lists and information), it is the effect of this
legislation on employee personal information that concerns the payroll and human resources
departments.

\sphinxAtStartPar
Employers collect personal employee information to conduct and protect their business, and
to comply with government legislation (for example, Employment/Labour Standards and
statutory deductions relating to CPP/QPP contributions, EI and QPIP premiums along with
income tax). As well, many employers provide benefits such as dental, medical and pension
plans that require the collection of even greater amounts of personal data.

\begin{sphinxadmonition}{note}{Note:}
\sphinxAtStartPar
PIPEDA does not require that employers obtain consent from prospective employees, current
employees, or terminated employees to collect, use, and disclose information about that
person where the information is necessary for the creation, maintenance, and termination of
the employment relationship. It is, however, the case that the employer will provide notice to
the employee so that they are knowledgeable with respect to the information that the
employer collects, uses, and discloses.

\sphinxAtStartPar
This notice should be provided to prospective employees as part of the recruitment process
and also as part of the on\sphinxhyphen{}boarding process. In addition, if there are changes to personal data
practices for employee information, employees should be informed about such changes in a
timely manner.
\end{sphinxadmonition}


\subsubsection{Consent}
\label{\detokenize{2_compliance:consent}}
\sphinxAtStartPar
According to PIPEDA, employers must obtain an employee’s consent before they collect
personal information where that information is not required for the employment relationship.
Further, the information collected must be for a specific purpose and must be destroyed once
that purpose is no longer valid.

\sphinxAtStartPar
There are two forms of consent that can be obtained from an employee \sphinxhyphen{} expressed and
implied:

\sphinxAtStartPar
\sphinxstylestrong{Expressed consent} should be used for particularly sensitive employee information such as
might be asked for in the case of a voluntary employee assistance program.

\sphinxAtStartPar
\sphinxstylestrong{Implied consent} means the employee is considered to have consented indirectly. An
example of implied consent is when an employee completes a form for an employer provided
but optional service such as a \sphinxstyleemphasis{social club} for birthday gifts and notices. Participating in this
club is not required for the employment relationship so consent is required. But the
information requested, and the context is not overly sensitive so consent for the collection
and use of employee data may be implied by the fact that the employee completed the
voluntary form. It doesn’t need an “I consent” checkbox.

\sphinxAtStartPar
In essence, the more sensitive the information, the more one should use express written
consent, which outlines in detail the specific purpose for which an employer is using the
information. It is critical for those working in payroll to be aware of the requirements of
privacy legislation that applies to their employees and to have the necessary procedures in
place to comply with the legislation. If an employee chooses not to disclose the information
and is not required to do so by law, an employer cannot force an employee to divulge it.


\subsubsection{Exceptions to Consent Requirement}
\label{\detokenize{2_compliance:exceptions-to-consent-requirement}}
\sphinxAtStartPar
Subparagraph 7(3) of the Personal Information Protection and Electronic Documents Act
(Bill C6) allows an employer to disclose personal information without the knowledge or
consent of the individual if the disclosure is made to a government institution which has
identified its lawful authority, and if the disclosure is for the purpose of administering any
law of Canada.

\sphinxAtStartPar
PIPEDA permits federal government agencies such as the CRA, ESDC, Service Canada and
provincial/territorial Ministries of Labour to obtain personal employee information needed to
administer programs or benefits, or to perform an audit. Legislation specifically provides
these government bodies with the right to request personal employee information and inspect
certain records and documents. As a result, the employer does not need to obtain the
employee’s permission to provide the information.

\sphinxAtStartPar
In addition to disclosures to government that are mandated by legislation and in relation to
employment, subparagraph 7.3 of PIPEDA states that an employer that is regulated by
federal labour codes can “…collect, use and disclose personal information without the consent of the individual if
(a) the collection, use or disclosure is necessary to establish, manage or terminate an
employment relationship between the federal work, undertaking or business and the
individual; and
(b) the federal work, undertaking or business has informed the individual that the personal
information will be or may be collected, used or disclosed for those purposes”.


\subsubsection{Use and Storage of Personal Information}
\label{\detokenize{2_compliance:use-and-storage-of-personal-information}}
\sphinxAtStartPar
According to PIPEDA, organizations can only use information for the purpose for which it
was collected. Employers must fully disclose in writing to the employee the reasons why
they require the information, as well as what will be done with it.

\sphinxAtStartPar
Personal information must not be disclosed to external stakeholders without the employee’s
consent and only for the purpose for which the information was collected. For example, if the
organization is being audited by a government agency, such as the CRA, the employee’s
medical information should not be included with the information provided for audit purposes.

\sphinxAtStartPar
There are times when employers are required to collect information about employees in order
to comply with employment/labour standards or human rights legislation. For example, to
accommodate an employee for religious days and holidays, an employer needs to know about
the employee’s religious beliefs. To seek out this type of information for any other reason
invades the individual’s right to privacy.


\subsubsection{Limitations on Use \sphinxhyphen{} the Social Insurance Number example}
\label{\detokenize{2_compliance:limitations-on-use-the-social-insurance-number-example}}
\sphinxAtStartPar
The purpose of a social insurance number (SIN) is to identify an individual for specific
government programs. This information may not be collected, stored, used or disclosed for
any other purpose without the employee’s consent. Where the SIN is to be used for purposes
of identification, an organization must provide a convenient method for the employee to
withdraw his/her consent for that use at any time.

\sphinxAtStartPar
Employers are authorized to collect a SIN from employees in order to produce Records of
Employment and income tax information slips. Unless the employee has provided a SIN for
another specific use, and has consented to that specific use in writing, an employer could be
subject to fines for each improper use of that number.

\sphinxAtStartPar
As a general rule, an employer may not communicate the number to a third party without the
employee’s specific consent to do so. Exceptions are cases in which it is the employer’s
obligation to report an employee’s SIN to RQ, CRA, ESDC or Service Canada.

\sphinxAtStartPar
The SIN should not be used on pay statements or communicated to unions or benefit carriers.
They should not be used as an identifier by any organization other than the government
agencies mentioned above, unless the employee provides written consent to do so.


\section{Pension Benefits Standards Act}
\label{\detokenize{2_compliance:pension-benefits-standards-act}}

\section{Canadian Human Rights Act}
\label{\detokenize{2_compliance:canadian-human-rights-act}}

\section{Employment Equity Act}
\label{\detokenize{2_compliance:employment-equity-act}}

\section{Summary}
\label{\detokenize{2_compliance:summary}}\begin{itemize}
\item {} 
\sphinxAtStartPar
Under the Canada Pension Plan Act and the Employment Insurance Act, the Canada Revenue Agency is responsible for determining:
\sphinxhyphen{} whether or not an individual’s employment is pensionable under the Canada Pension Plan Act or insurable under the Employment Insurance Act
\sphinxhyphen{} the types of earnings that are considered pensionable or insurable
\sphinxhyphen{} how many hours an insured person has in insurable employment
\sphinxhyphen{} the recovery of any debts owed as a result of an overpayment of Canada Pension Plan, Employment Insurance, or Old Age Security benefits

\item {} 
\sphinxAtStartPar
The Canada Revenue Agency is responsible for ensuring that Canada Pension Plan contributions and Employment Insurance premiums are deducted, remitted, and reported as required by legislation.

\item {} 
\sphinxAtStartPar
The Canada Revenue Agency collects provincial/territorial income taxes on behalf of all provinces/territories except Québec.

\item {} 
\sphinxAtStartPar
Revenu Québec collects the provincial income tax for the province of Québec.

\item {} 
\sphinxAtStartPar
Employers who remit withholdings or deductions late, withhold the statutory deductions but do not remit them, or fail to deduct the required deductions will be subject to penalties, which may increase on subsequent occurrences, plus interest charges.

\item {} 
\sphinxAtStartPar
All monies deducted on behalf of the Canada Revenue Agency are considered to be held “in trust” for the Receiver General.

\item {} 
\sphinxAtStartPar
Employment and Social Development Canada is responsible for matters relating to:
\begin{itemize}
\item {} 
\sphinxAtStartPar
amending the regulations made under the Canada Pension Plan and the Employment Insurance Act

\item {} 
\sphinxAtStartPar
keeping records of each individual’s Canada Pension Plan contributions and pensionable earnings

\item {} 
\sphinxAtStartPar
the establishment of annual maximum insurable earnings

\item {} 
\sphinxAtStartPar
the administration of provisions related to Wage Loss plans

\item {} 
\sphinxAtStartPar
the administration of provisions regarding Job Creation programs

\end{itemize}

\item {} 
\sphinxAtStartPar
Employment and Social Development Canada’s Employment Insurance program

\end{itemize}

\sphinxAtStartPar
provides temporary financial assistance for unemployed Canadians while they look
for work or upgrade their skills.
\begin{itemize}
\item {} 
\sphinxAtStartPar
Service Canada serves as the government’s operational arm while Employment and

\end{itemize}

\sphinxAtStartPar
Social Development Canada operates as the policy\sphinxhyphen{}making body.
\begin{itemize}
\item {} 
\sphinxAtStartPar
Service Canada is responsible for:
\begin{itemize}
\item {} 
\sphinxAtStartPar
the issuance of Social Insurance Numbers (SIN) and the protection and security of SIN information

\item {} 
\sphinxAtStartPar
the delivery of services to employers, including Record of Employment on the Web

\item {} 
\sphinxAtStartPar
the administration of Employment Insurance programs to individuals, including regular, illness, pregnancy/parental, critically ill or injured person and compassionate care benefits

\item {} 
\sphinxAtStartPar
the administration of the Employment Insurance Premium Reduction program, including granting qualified employers a reduced Employment Insurance premium rate

\item {} 
\sphinxAtStartPar
the administration of Canada Pension Plan benefits, including retirement, disability, survivor, children’s and death benefits

\item {} 
\sphinxAtStartPar
the administration of benefits for seniors, including the Old Age Security Program and the Guaranteed Income Supplement Payroll is responsible for deducting and remitting Employment Insurance premiums on behalf of employees and employers.

\end{itemize}

\item {} 
\sphinxAtStartPar
Payroll is responsible for deducting and remitting Employment Insurance premiums on behalf of employees and employers.

\item {} 
\sphinxAtStartPar
Payroll is responsible for capturing information related to insurable earnings and hours, and reporting that information on the Record of Employment.

\item {} 
\sphinxAtStartPar
The Canadian government based its privacy provisions in its legislation for the

\end{itemize}

\sphinxAtStartPar
protection of personal information on a set of guidelines called the Ten Privacy
Principles.
\begin{itemize}
\item {} 
\sphinxAtStartPar
The Personal Information Protection and Electronic Documents Act has applied to

\end{itemize}

\sphinxAtStartPar
federally\sphinxhyphen{}regulated organizations such as banks, telecommunications and
transportation companies since January 2001.
\begin{itemize}
\item {} 
\sphinxAtStartPar
Since January 2004 the Personal Information Protection and Electronic Documents

\end{itemize}

\sphinxAtStartPar
Act has applied to the collection, use or disclosure of personal information in the
course of any commercial activity within a province that does not have its own
privacy legislation.
\begin{itemize}
\item {} 
\sphinxAtStartPar
Express consent means the employee provides their consent either verbally (in which

\end{itemize}

\sphinxAtStartPar
case when and how the consent was received should be documented) or in writing.
\begin{itemize}
\item {} 
\sphinxAtStartPar
Implied consent means the employee is considered to have consented indirectly.

\item {} 
\sphinxAtStartPar
The employer does not need to obtain the employee’s permission to provide personal

\end{itemize}

\sphinxAtStartPar
information where legislation provides federal government agencies such as the
Canada Revenue Agency, Employment and Social Development Canada, Service
Canada and provincial/territorial Ministries of Labour with the right to request
personal employee information in order to administer programs or benefits, or in the
case of an audit.
\begin{itemize}
\item {} 
\sphinxAtStartPar
Other than an employer’s obligation to report an employee’s Social Insurance

\end{itemize}

\sphinxAtStartPar
Number to the Canada Revenue Agency, Employment and Social Development
Canada, Service Canada or Revenu Québec, an employer may not communicate the
number to a third party without the employee’s specific consent to do so.


\section{Review Questions}
\label{\detokenize{2_compliance:review-questions}}\begin{enumerate}
\sphinxsetlistlabels{\arabic}{enumi}{enumii}{}{.}%
\item {} 
\sphinxAtStartPar
What are the three main programs specifically related to payroll that the Canada Revenue Agency administers?

\item {} 
\sphinxAtStartPar
If an organization deducts \$27,400 in Canada Pension Plan contributions from its employees and \$21,200 in Employment Insurance premiums, how much would have to be remitted in total to the Canada Revenue Agency?

\item {} 
\sphinxAtStartPar
True or False: The Canada Revenue Agency collects provincial/territorial income taxes for all provinces and territories.

\item {} 
\sphinxAtStartPar
True or False: The emphasis of the Employment Insurance program is on paying people so they don’t have to return to work.

\item {} 
\sphinxAtStartPar
Indicate which federal department or agency would be responsible in the following scenarios.

\end{enumerate}
\begin{quote}

\sphinxAtStartPar
An employer is preparing the budget for the next year and wants to know the annual maximum insurable earnings amount.

\sphinxAtStartPar
An employee wants to retire and apply for Canada Pension Plan benefits.

\sphinxAtStartPar
There is a new type of earning in the new collective agreement. You are not sure if it is insurable.

\sphinxAtStartPar
The organization would like to apply for a reduction in its Employment Insurance premium rate.
\end{quote}

\sphinxAtStartPar
6. How does the Personal Information Protection and Electronic Documents Act
legislation affect the handling of employee personal information?

\sphinxAtStartPar
7. Explain the difference between implied and express employee consent and provide an
example of each.

\sphinxAtStartPar
8. The Personal Information Protection and Electronic Documents Act contains ten
privacy principles. Choose two and develop a statement for each that could be included
in your organization’s privacy policy.

\sphinxstepscope


\chapter{EMPLOYEE vs. INDEPENDENT CONTRACTOR}
\label{\detokenize{3_contracts:employee-vs-independent-contractor}}\label{\detokenize{3_contracts::doc}}

\section{The Employee\sphinxhyphen{}Employer Relationship}
\label{\detokenize{3_contracts:the-employee-employer-relationship}}
\sphinxAtStartPar
Determining the nature of the working relationship between an individual and an organization is essential in all employment situations. Whether the individual is classified as an employee or self\sphinxhyphen{}employed directly affects the statutory withholding requirements and the organization’s compliance with applicable legislation. To support this assessment, the Canada Revenue Agency (CRA) provides a set of guidelines designed to help distinguish between the two classifications. Importantly, the decision is not made by the worker but must be based on objective criteria and legal standards.

\sphinxAtStartPar
Payroll practitioners play an important role in promoting awareness of this distinction throughout the organization. By proactively communicating the significance of establishing a valid employee\sphinxhyphen{}employer relationship, payroll professionals help ensure that employment classifications are accurate and compliant.

\sphinxAtStartPar
Once an employee\sphinxhyphen{}employer relationship is confirmed, the payroll department becomes responsible for meeting compliance obligations related to statutory withholdings. This includes deducting the appropriate amounts—such as income tax, Canada Pension Plan contributions, and Employment Insurance premiums—from employee pay and remitting them to the government within the required timelines. Proper classification and adherence to these rules are key to maintaining legal and financial accountability.

\sphinxAtStartPar
Where an employee\sphinxhyphen{}employer relationship exists, the CRA requires the employer to:
\begin{itemize}
\item {} 
\sphinxAtStartPar
register with the Canada Revenue Agency for a Business Number (BN)

\item {} 
\sphinxAtStartPar
withhold the statutory deductions of income tax, Canada Pension Plan (CPP) contributions, and Employment Insurance (EI) premiums on amounts paid to employees

\item {} 
\sphinxAtStartPar
remit the amounts withheld as well as the required employer’s share of CPP contributions and EI premiums to the Canada Revenue Agency

\item {} 
\sphinxAtStartPar
report the employees’ income and deductions on the appropriate information return

\item {} 
\sphinxAtStartPar
give the employees copies of their T4 slips by the end of February of the following calendar year

\end{itemize}

\sphinxAtStartPar
Information on the factors to consider when determining whether an employee\sphinxhyphen{}employer relationship exists can be found in the
Canada Revenue Agency guide, Employee or Self\sphinxhyphen{}Employed? \sphinxhyphen{} RC4110. The guide is available on the CRA’s website,
\sphinxurl{https://www.canada.ca/en/revenue-agency.html}.


\subsection{Contract of Service (Employment)}
\label{\detokenize{3_contracts:contract-of-service-employment}}
\sphinxAtStartPar
A \sphinxstylestrong{contract of service} is an arrangement whereby an individual (the employee) agrees to
work on a full\sphinxhyphen{}time or part\sphinxhyphen{}time basis for an employer for a specified or indeterminate period
of time.

\sphinxAtStartPar
Under a contract of service, one party serves another in return for a salary or some other form
of remuneration.


\subsection{Contract for Service (Subcontracting)}
\label{\detokenize{3_contracts:contract-for-service-subcontracting}}
\sphinxAtStartPar
A \sphinxstylestrong{contract for service} is a business relationship whereby one party agrees to perform certain
specific work stipulated in the contract for another party. It usually calls for the
accomplishment of a clearly defined task but does not normally require that the contracting
party do anything him/herself. A person who carries out a contract for service may be
considered a contract worker, a self\sphinxhyphen{}employed person or an independent contractor.

\sphinxAtStartPar
A business relationship is formed when a self\sphinxhyphen{}employed individual enters into a verbal or written agreement to complete
specific work for a payer in exchange for compensation. This arrangement does not establish an employer\sphinxhyphen{}employee relationship;
instead, it represents a contract for services.

\sphinxAtStartPar
Under this type of agreement, the self\sphinxhyphen{}employed individual is responsible for delivering a final result within an agreed
timeframe, using methods of their own choosing. They are not subject to the direction or supervision of the payer while
completing the work, and they retain autonomy over how tasks are executed. In most cases, the payer does not participate in
or influence the work process, meaning control over the work lies entirely with the self\sphinxhyphen{}employed individual. This structure
reflects a high level of independence and flexibility, distinguishing it clearly from traditional employment relationships.

\sphinxAtStartPar
Under a contract for service, a self\sphinxhyphen{}employed individual accepts both the potential for profit and the risk of financial loss.
Prior to engagement, the individual agrees on the total cost of the work, uses personal tools and equipment, and assumes full
responsibility for how the work is performed. This means the individual bears any unforeseen costs or challenges that arise
during the project. Conversely, if the work is completed more efficiently than expected, the financial gain—through retained
profits—is greater.

\sphinxAtStartPar
Organizations often utilize contracts for service when they require tasks or projects that fall outside their normal business
operations. While the relationship between a payer and a self\sphinxhyphen{}employed contractor may resemble that of an employer and
employee, there is a key distinction. In a contract for service, the payer specifies the desired outcome or deliverable, but
not how the work should be completed. In contrast, a contract of service allows the employer to direct both the tasks and the
method by which they are carried out, establishing a more controlled, employee\sphinxhyphen{}based relationship.

\sphinxAtStartPar
Under a contract for service, the payer exercises general oversight to ensure that the agreed work is completed as specified.
However, this oversight does not extend to controlling how the work is performed. The self\sphinxhyphen{}employed individual retains
autonomy over the methods used to complete the tasks. Receiving general instructions from a project manager or similar
representative does not establish an employer\sphinxhyphen{}employee relationship.

\sphinxAtStartPar
An employee\sphinxhyphen{}employer relationship is recognized when an organization has the authority to direct and control both the work and
the manner in which it is performed. If there is uncertainty about whether such a relationship exists, the Canada Revenue
Agency (CRA) recommends submitting Form CPT1 — Request for a CPP/EI Ruling: Employee or Self\sphinxhyphen{}Employed? — to obtain
clarification. A sample of this form can be found at the end of this section.

\sphinxAtStartPar
Independent contractors or self\sphinxhyphen{}employed individuals are not classified as employees if no employer\sphinxhyphen{}employee relationship is
present. They typically submit invoices and receive payment through accounts payable. However, the act of submitting an
invoice alone is not sufficient to confirm self\sphinxhyphen{}employment status. Proper assessment of the working relationship is essential
sto ensure accurate classification and compliance with tax and labor regulations.


\section{Factors Determining the Type of Contract}
\label{\detokenize{3_contracts:factors-determining-the-type-of-contract}}
\sphinxAtStartPar
The CRA uses a two\sphinxhyphen{}step approach to examine the relationship between the worker and the
payer for relationships outside the province of Québec. The approach used for relationships
in the province of Québec will be discussed in a later chapter.

\sphinxAtStartPar
\sphinxstylestrong{Step 1:}
The first step is to establish what the intent was when the worker and the payer entered into
the working arrangement. Did they intend to enter into an employee\sphinxhyphen{}employer relationship
(contract of service) or did they intend to enter into a business relationship (contract for
service). The CRA must determine not only how the working relationship has been defined
but why it was defined that way.

\sphinxAtStartPar
\sphinxstylestrong{Step 2:}
The CRA then considers certain factors when determining if a contract of service or a
contract for service exists. In order to understand the working relationship and verify that the
intent of the worker and the payer is reflected in the facts, they will ask a series of questions
that relate to the following factors:
\begin{itemize}
\item {} 
\sphinxAtStartPar
the level of control the payer has over the worker

\item {} 
\sphinxAtStartPar
whether or not the worker provides the tools and equipment

\item {} 
\sphinxAtStartPar
whether the worker can subcontract the work or hire assistants

\item {} 
\sphinxAtStartPar
the degree of financial risk taken by the worker

\item {} 
\sphinxAtStartPar
the degree of responsibility for investment and management held by the worker

\item {} 
\sphinxAtStartPar
the worker’s opportunity for profit

\item {} 
\sphinxAtStartPar
any other relevant factors, such as written contracts

\end{itemize}

\sphinxAtStartPar
The CRA will look at the answers independently and then together and consider whether or
not they reflect the intent that was originally stated. Considered individually, the response to
each of these questions is not conclusive; however, when weighed together, certain
conclusions may be drawn. When there is no common intent, the CRA will decide if the
answers are more consistent with a contract of service or a contract for service.
Each of these factors will be discussed in the material and indicators showing whether the
worker is an employee or self\sphinxhyphen{}employed will be provided.


\subsection{Control}
\label{\detokenize{3_contracts:control}}
\sphinxAtStartPar
One of the key factors in determining a worker’s status is the extent to which the payer has the ability, authority, or
right to control both what work is performed and how it is carried out. Equally important is the level of independence the
worker maintains in performing their duties.

\sphinxAtStartPar
In evaluating the relationship, both the payer’s oversight of the worker’s day\sphinxhyphen{}to\sphinxhyphen{}day activities and their overall
influence are assessed. However, it is the payer’s right to exercise control—rather than whether that control is actively
used—that holds the most weight in determining the nature of the working relationship.

\sphinxAtStartPar
Worker is an \sphinxstyleemphasis{Employee} when:
\begin{itemize}
\item {} 
\sphinxAtStartPar
The relationship is one of subordination.

\item {} 
\sphinxAtStartPar
The payer will often direct, scrutinize, and effectively control many elements of how the work is performed.

\item {} 
\sphinxAtStartPar
The payer controls both the results of the work and the method used to do the work.

\item {} 
\sphinxAtStartPar
The payer determines what jobs the worker will do.

\item {} 
\sphinxAtStartPar
The worker receives training or direction from the payer on how to do the work.

\end{itemize}

\sphinxAtStartPar
Worker is a \sphinxstyleemphasis{Self\sphinxhyphen{}Employed} when:
\begin{itemize}
\item {} 
\sphinxAtStartPar
Individual usually works independently, does not have anyone overseeing them.

\item {} 
\sphinxAtStartPar
The worker is usually free to work when and for whom they choose and may provide their services to different payers at the same time.

\item {} 
\sphinxAtStartPar
The worker can accept or refuse work from the payer.

\item {} 
\sphinxAtStartPar
The working relationship between the payer and the worker does not present a degree of continuity, loyalty, security, subordination, or integration.

\end{itemize}


\subsection{Tools and Equipment}
\label{\detokenize{3_contracts:tools-and-equipment}}
\sphinxAtStartPar
Ownership of tools and equipment is not a definitive factor in determining the nature of a working relationship or the type
of contract in place. While self\sphinxhyphen{}employed individuals often use their own tools to fulfill contractual obligations—making
such ownership common in business relationships—this alone does not confirm self\sphinxhyphen{}employment. Employees may also be required
to supply their own tools, depending on the trade or occupation.

\sphinxAtStartPar
In typical employee\sphinxhyphen{}employer relationships, the employer provides the necessary equipment and assumes the costs associated
with its use, including maintenance, insurance, transportation, rental fees, and operational expenses such as fuel. However,
in certain industries—such as automotive repair, painting, carpentry, and technical fields like software development or
surveying—it is standard practice for employees to use their own tools or specialized instruments.

\sphinxAtStartPar
In contrast, self\sphinxhyphen{}employed individuals not only supply their own equipment but also bear the associated costs. Significant
financial investment in tools—especially those that require ongoing maintenance or replacement—can suggest a business
relationship, as these individuals assume both the potential for profit and the risk of loss.

\sphinxAtStartPar
Ultimately, the key consideration is the scale of the investment and the financial responsibility related to repairs,
replacement, and insurance, rather than mere ownership itself.

\sphinxAtStartPar
The worker is an employee when:
\begin{itemize}
\item {} 
\sphinxAtStartPar
The payer supplies most of the tools and equipment.

\item {} 
\sphinxAtStartPar
The payer retains the right of use over the tools and equipment provided to the worker.

\item {} 
\sphinxAtStartPar
The worker supplies the tools and equipment and the payer reimburses the worker for their use

\end{itemize}

\sphinxAtStartPar
The worker is a self\sphinxhyphen{}employed individual when:
\begin{itemize}
\item {} 
\sphinxAtStartPar
The worker provides the tools and equipment required and is responsible for the cost of repairs, insurance and maintenance and retains the right over the use of these assets.

\item {} 
\sphinxAtStartPar
The worker supplies his or her own workspace, is responsible for the costs to maintain it, and does substantial work from that site.

\end{itemize}


\subsection{Subcontracting Work or Hiring Assistants}
\label{\detokenize{3_contracts:subcontracting-work-or-hiring-assistants}}
\sphinxAtStartPar
As subcontracting work or hiring assistants can affect a worker’s chance of profit or risk of loss, this can help determine
the type of business relationship.

\sphinxAtStartPar
The worker is an employee when:
\begin{itemize}
\item {} 
\sphinxAtStartPar
The worker cannot hire helpers or assistants.

\item {} 
\sphinxAtStartPar
The worker must perform the services personally.

\end{itemize}

\sphinxAtStartPar
The worker is a self\sphinxhyphen{}employed individual when:
\begin{itemize}
\item {} 
\sphinxAtStartPar
The worker does not have to perform the service personally.

\item {} 
\sphinxAtStartPar
They can hire another party to complete the work, without consulting with the payer.

\end{itemize}


\subsection{Financial Risk}
\label{\detokenize{3_contracts:financial-risk}}
\sphinxAtStartPar
When evaluating the nature of a working relationship, the Canada Revenue Agency (CRA) considers whether the individual
incurs fixed, ongoing costs or unreimbursed expenses. In traditional employee arrangements, employers typically reimburse
expenses that arise as part of the job—for example, travel or business\sphinxhyphen{}related costs.

\sphinxAtStartPar
In contrast, self\sphinxhyphen{}employed individuals often assume greater financial risk by covering recurring operational costs
regardless of whether active work is being performed. These may include equipment leasing, office space rental, or other
business overheads. While both employees and contractors may receive reimbursement for certain expenses, the CRA places
particular emphasis on identifying costs that are not reimbursed. The presence of such expenses may indicate a business
relationship, reflecting the independence and financial responsibility characteristic of self\sphinxhyphen{}employment.

\sphinxAtStartPar
The worker is an employee when:
\begin{itemize}
\item {} 
\sphinxAtStartPar
The worker is not usually responsible for any operating expenses.

\item {} 
\sphinxAtStartPar
The worker is not financially liable if he or she does not fulfill the obligations of the contract.

\item {} 
\sphinxAtStartPar
The payer determines and controls the method and amount of pay.

\end{itemize}

\sphinxAtStartPar
The worker is a self\sphinxhyphen{}employed individual when:
\begin{itemize}
\item {} 
\sphinxAtStartPar
The worker is financially liable if he or she does not fulfill the obligations of the contract.

\item {} 
\sphinxAtStartPar
The worker does not receive any protection or benefits from the payer.

\item {} 
\sphinxAtStartPar
The worker hires helpers to assist and pays them.

\item {} 
\sphinxAtStartPar
The worker advertises the services offered.

\end{itemize}


\subsection{Responsibility for Investment and Management}
\label{\detokenize{3_contracts:responsibility-for-investment-and-management}}
\sphinxAtStartPar
When assessing whether a business relationship exists, one important indicator is the worker’s financial investment in the
services they provide. If an individual is required to invest in equipment, materials, or other resources to complete the
work, this suggests the presence of a contract for service rather than an employment relationship.

\sphinxAtStartPar
Another key factor is decision\sphinxhyphen{}making authority related to financial outcomes. When the worker independently makes business
decisions that influence their profit or loss—such as pricing, project selection, or service delivery methods—it further
supports the classification of a self\sphinxhyphen{}employed individual operating under a business arrangement. These characteristics
reflect the autonomy and financial risk typically associated with self\sphinxhyphen{}employment.

\sphinxAtStartPar
The worker is an employee when:
\begin{itemize}
\item {} 
\sphinxAtStartPar
The worker has no capital investment in the business.

\item {} 
\sphinxAtStartPar
The worker does not have a business presence.

\end{itemize}

\sphinxAtStartPar
The worker is a self\sphinxhyphen{}employed individual when:
\begin{itemize}
\item {} 
\sphinxAtStartPar
The worker has capital investment, manages his or her staff, hires and pays individuals to help perform the work, and has established a business presence.

\end{itemize}


\subsection{Opportunity for Profit}
\label{\detokenize{3_contracts:opportunity-for-profit}}
\sphinxAtStartPar
A business relationship is often indicated when a worker has the ability to realize a profit or incur a loss, reflecting their
control over the financial and operational aspects of the services they provide. Self\sphinxhyphen{}employed individuals typically
negotiate their own rates, choose which contracts to accept, and may take on multiple contracts simultaneously. To fulfill
contractual obligations, they often incur and manage expenses, which directly influence their potential for profit.

\sphinxAtStartPar
In contrast, employees generally do not bear financial risk or benefit from profit. While commission\sphinxhyphen{}based employees may
increase their earnings through performance, this does not represent profit in the traditional sense, as it does not reflect
income earned beyond expenses. Moreover, employees do not typically share in a business’s profits or losses.

\sphinxAtStartPar
When assessing worker classification, the Canada Revenue Agency (CRA) considers the extent to which the individual controls
their revenue and expenses. Another key factor is the method of payment: employees are usually compensated at a fixed rate
based on a consistent pay schedule (e.g., hourly, weekly, or annually). Self\sphinxhyphen{}employed individuals, however, are often paid a
flat rate for a specific job, especially when they absorb related costs—an arrangement that commonly signals a business
relationship.

\sphinxAtStartPar
The worker is an employee when:
\begin{itemize}
\item {} 
\sphinxAtStartPar
The worker is not in a position to realize a business profit or loss.

\item {} 
\sphinxAtStartPar
The worker is entitled to benefit plans that are normally only offered to employees.

\end{itemize}

\sphinxAtStartPar
The worker is a self\sphinxhyphen{}employed individual when:
\begin{itemize}
\item {} 
\sphinxAtStartPar
The worker is compensated by a flat fee.

\item {} 
\sphinxAtStartPar
The worker can hire and pay a substitute.

\end{itemize}

\sphinxAtStartPar
The worker is an employee when:

\sphinxAtStartPar
The worker is a self\sphinxhyphen{}employed individual when:


\section{Review Summary}
\label{\detokenize{3_contracts:review-summary}}
\sphinxAtStartPar
Employment relationships are defined through contractual arrangements. A contract of service refers to a traditional
employer\sphinxhyphen{}employee relationship, where an individual commits to working for an employer—either on a full\sphinxhyphen{}time or part\sphinxhyphen{}time
basis—for a specified or ongoing period. The employer has authority over both the duties and how they are executed.

\sphinxAtStartPar
Conversely, a contract for service reflects a business arrangement where an independent contractor agrees to perform specific
tasks, with discretion over how the work is completed. This signifies a client\sphinxhyphen{}provider relationship rather than an employment
one.

\sphinxAtStartPar
To assess worker classification—particularly outside Québec—the Canada Revenue Agency (CRA) employs a two\sphinxhyphen{}step evaluation. Key
considerations include:

\sphinxAtStartPar
Control: Whether the payer holds the right to determine what work is done and how it is executed.

\sphinxAtStartPar
Independence: The degree of autonomy exercised by the worker.

\sphinxAtStartPar
Ownership of Tools: Significant investment in tools and equipment, along with maintenance and insurance responsibilities, may indicate a business relationship.

\sphinxAtStartPar
Financial Risk: Ongoing operational costs or unreimbursed expenses reflect a higher likelihood of self\sphinxhyphen{}employment.

\sphinxAtStartPar
Revenue Control: The ability to manage pricing, accept multiple contracts, and influence earnings supports classification under a contract for service.

\sphinxAtStartPar
Collectively, these factors guide proper categorization for legal and tax purposes, helping organizations ensure compliance and mitigate potential risk.


\section{Review Questions}
\label{\detokenize{3_contracts:review-questions}}
\sphinxAtStartPar
What is the primary objective of the payroll department?
\begin{quote}

\sphinxAtStartPar
The primary objective of the payroll department is to pay employees accurately and
on time, in compliance with the legislative requirements for a full annual payroll
cycle.
\end{quote}

\sphinxAtStartPar
List four definitions of payroll.
\begin{itemize}
\item {} 
\sphinxAtStartPar
the department that administers the payroll

\item {} 
\sphinxAtStartPar
the total number of people employed by an organization

\item {} 
\sphinxAtStartPar
the wages and salaries paid out in a year

\item {} 
\sphinxAtStartPar
a list of employees to be paid and the amount due to each

\end{itemize}

\sphinxAtStartPar
List the three types of payroll management stakeholders and provide an example of each.
\begin{quote}

\sphinxAtStartPar
Payroll management stakeholders are government (federal and provincial/territorial), internal
(employees, employers and other departments) and external (benefit carriers, courts, unions, pension
providers, charities, third party administrators and outsource/software vendors).
\end{quote}

\sphinxAtStartPar
Explain the difference between legislation and regulation.
\begin{quote}

\sphinxAtStartPar
Legislation determines what the rules are, while regulations determine how the rules are to be applied.
\end{quote}

\sphinxAtStartPar
What are two examples of sources of information that you use (or could use) to keep upto\sphinxhyphen{}date on payroll compliance changes?
\begin{quote}

\sphinxAtStartPar
The Canadian Payroll Association offers Payroll InfoLine, a phone\sphinxhyphen{}in and e\sphinxhyphen{}mail information service for members
\begin{quote}
\begin{itemize}
\item {} 
\sphinxAtStartPar
The Canada Revenue Agency (CRA) produces guides, publications and Income Tax Bulletins, folios and Circulars, posts news bulletins and enables

\end{itemize}

\sphinxAtStartPar
participation on an electronic mailing list with e\sphinxhyphen{}mail alerts for new content to the site
\sphinxhyphen{} The Revenu Québec (RQ) website provides guides, publications, bulletins, forms, online services and enables participation on an electronic mailing list with e\sphinxhyphen{}mail notifications of tax news articles
\sphinxhyphen{} Employment/labour standards (federal, provincial and territorial) publications and websites
\sphinxhyphen{} Employment and Social Development Canada (ESDC) and Service Canada (SC) publications including information regarding the Employment Insurance (EI) program and the Social Insurance Number
\sphinxhyphen{} CCH Canada Limited publishes a series of volumes on employment and labour law, pensions and benefits, etc., that supplies information on legislation with regular updates as changes become law
\sphinxhyphen{} Carswell publishes The Canadian Payroll Manual and offers a phone\sphinxhyphen{}in service to subscribers
\end{quote}

\sphinxAtStartPar
Copies of legislation are available from the printing offices of the federal, provincial and territorial governments as well as through government websites.
\end{quote}

\sphinxAtStartPar
List three external stakeholders and explain their compliance requirements.
\begin{quote}

\sphinxAtStartPar
Benefit Carriers \sphinxhyphen{} Payroll is responsible for deducting and remitting premiums for the insurance coverage to the carriers and for providing reports on employee enrolment and coverage levels.
Courts and the CRA \sphinxhyphen{} Payroll must accurately deduct and remit amounts ordered to be withheld through garnishments, third party demands, requirements to pay and support deduction orders.
Unions \sphinxhyphen{} Payroll must accurately deduct and remit union dues and initiation fees, and ensure that the terms of the collective agreement are adhered to.
Pension Providers \sphinxhyphen{} Third party pension plan providers may require payroll to provide enrolment reports on participating employees and length of service calculations, and to remit employee deductions and employer contributions.
\end{quote}

\sphinxAtStartPar
Indicate the jurisdiction the following employees fall under:
\begin{itemize}
\item {} 
\sphinxAtStartPar
Canada Post Corporation (F)

\item {} 
\sphinxAtStartPar
An insurance company (P)

\item {} 
\sphinxAtStartPar
A uranium mining company (F)

\item {} 
\sphinxAtStartPar
Canadian Broadcasting Corporation (F)

\item {} 
\sphinxAtStartPar
A retail department store with locations in every province (P)

\item {} 
\sphinxAtStartPar
A chartered bank (F)

\end{itemize}

\sphinxAtStartPar
What is the difference between a contract of service and a contract for service?
\begin{quote}

\sphinxAtStartPar
A contract of service is an arrangement whereby an individual (the employee) agrees to work on a full\sphinxhyphen{}time or part\sphinxhyphen{}time basis for an employer for a specified or indeterminate period of time.

\sphinxAtStartPar
A contract for service is a business relationship whereby one party agrees to perform certain specific work stipulated in the contract for another party.
\end{quote}

\sphinxAtStartPar
What are the factors that the Canada Revenue Agency (CRA) considers when
determining if a contract of service or a contract for service exists?

\sphinxAtStartPar
Please consider the following scenario.
\begin{quote}

\sphinxAtStartPar
You are a payroll professional working for a large manufacturing company. Your
organization has had many change initiatives over the last number of years including
three mergers and two large group terminations. Your company endorses the use of
consultants rather than growing the number of permanent employees.

\sphinxAtStartPar
Write a memo to your supervisor, who is the Chief Financial Officer of the company, to
explain why your role must coordinate with the Accounts Payable Department to ensure
that these payments are being handled correctly. Please prepare your answer in a separate
document.

\sphinxAtStartPar
\sphinxstyleemphasis{At the last weekly Finance meeting, Tom and I discussed the increase in the number of contractor invoices being
processed through accounts payable (AP). We have some concerns as to whether these individuals would be considered truly
selfemployed by the Canada Revenue Agency (CRA), or whether the CRA would determine them to be employees.}

\sphinxAtStartPar
\sphinxstyleemphasis{If the worker is considered self\sphinxhyphen{}employed, then payment, on submission of an invoice, will continue to be handled by AP. If, however, the worker is considered an
employee, they would have to be set up on payroll, as they would be in receipt of income from employment, subject to all legislated statutory withholdings.}

\sphinxAtStartPar
\sphinxstyleemphasis{I have attached the CRA’s form Request for a CPP/EI Ruling \sphinxhyphen{} Employee or SelfEmployed? \sphinxhyphen{} CPT1 for your information. This form can be completed by the
company and sent with supporting documentation, such as the terms and conditions of the contract, for a ruling from the CRA on the individual’s status.}

\sphinxAtStartPar
\sphinxstyleemphasis{I think that Payroll must coordinate with the Accounts Payable Department to ensure that these payments are being handled correctly.}

\sphinxAtStartPar
\sphinxstyleemphasis{Tom and I would be pleased to meet with you to ensure the company is in
compliance with all legislative requirements. Would you be available next Friday
morning at 10:00 to discuss?}
\end{quote}

\sphinxstepscope

\sphinxAtStartPar
Membership or participation in the Canada Pension Plan (CPP) and Employment Insurance
Plan (EI) is compulsory for certain types of employment. As a person responsible for the payroll
you need to know which employees must participate in these plans, what amounts to withhold
from employees and how much the employer will have to remit or send to the Canada
Revenue Agency (CRA).

\sphinxAtStartPar
Payroll plays a pivotal role in administering statutory deductions, specifically the collection and remittance of
Canada Pension Plan (CPP) contributions and Employment Insurance (EI) premiums. These mandatory deductions, along with the
employer’s matching portion, must be accurately submitted to the Canada Revenue Agency (CRA) within prescribed timelines.

\sphinxAtStartPar
This chapter outlines the essential criteria used to identify pensionable and insurable earnings, and provides detailed
guidance on calculating both employee deductions and employer contributions for regular and non\sphinxhyphen{}regular pay periods.

\sphinxAtStartPar
In accordance with federal legislation, CPP contributions are the \sphinxstylestrong{first deduction} applied to employment income,
followed by EI premiums. Because these deductions are mandated by statute, they are classified as \sphinxstylestrong{statutory deductions},
underscoring their legal significance and the employer’s obligation to ensure full compliance.

\sphinxAtStartPar
\sphinxstylestrong{Learning Objectives}

\sphinxAtStartPar
Upon completion of this chapter, you should be able to explain:
\begin{enumerate}
\sphinxsetlistlabels{\arabic}{enumi}{enumii}{}{.}%
\item {} 
\sphinxAtStartPar
The requirements and calculations for Canada Pension Plan contributions

\item {} 
\sphinxAtStartPar
The requirements and calculations for Employment Insurance premiums

\item {} 
\sphinxAtStartPar
What Service Canada uses the information on a Record of Employment for

\end{enumerate}

\sphinxAtStartPar
This chapter will cover the following topics:
\begin{quote}
\begin{enumerate}
\sphinxsetlistlabels{\arabic}{enumi}{enumii}{}{.}%
\item {} 
\sphinxAtStartPar
Identify the following Canada Pension Plan components:

\end{enumerate}
\begin{itemize}
\item {} 
\sphinxAtStartPar
Who must contribute to the Canada Pension Plan

\item {} 
\sphinxAtStartPar
Types of employment subject to Canada Pension Plan contributions

\item {} 
\sphinxAtStartPar
Types of employment not subject to Canada Pension Plan contributions

\item {} 
\sphinxAtStartPar
Payments and benefits subject to Canada Pension Plan contributions

\item {} 
\sphinxAtStartPar
Payments and benefits not subject to Canada Pension Plan contributions

\end{itemize}
\begin{enumerate}
\sphinxsetlistlabels{\arabic}{enumi}{enumii}{}{.}%
\setcounter{enumi}{1}
\item {} 
\sphinxAtStartPar
Calculate Canada Pension Plan contributions at an individual level

\item {} 
\sphinxAtStartPar
Identify the following Employment Insurance components:

\end{enumerate}
\begin{itemize}
\item {} 
\sphinxAtStartPar
Who must pay Employment Insurance premiums

\item {} 
\sphinxAtStartPar
Types of employment subject to Employment Insurance premiums

\item {} 
\sphinxAtStartPar
Types of employment not subject to Employment Insurance premiums

\item {} 
\sphinxAtStartPar
Payments and benefits subject to Employment Insurance premiums

\item {} 
\sphinxAtStartPar
Payments and benefits not subject to Employment Insurance premiums

\end{itemize}
\begin{enumerate}
\sphinxsetlistlabels{\arabic}{enumi}{enumii}{}{.}%
\setcounter{enumi}{3}
\item {} 
\sphinxAtStartPar
Calculate Employment Insurance premiums at an individual level

\item {} 
\sphinxAtStartPar
Describe the purpose of a Record of Employment

\item {} 
\sphinxAtStartPar
Identify when the Record of Employment must be completed

\end{enumerate}
\end{quote}


\chapter{Canada Pension Plan}
\label{\detokenize{cpp-and-ei:canada-pension-plan}}\label{\detokenize{cpp-and-ei::doc}}
\sphinxAtStartPar
Objective of this section is to enable you to identify the following Canada Pension Plan components:
\begin{itemize}
\item {} 
\sphinxAtStartPar
Who must contribute to the Canada Pension Plan

\item {} 
\sphinxAtStartPar
Types of employment subject to Canada Pension Plan contributions

\item {} 
\sphinxAtStartPar
Types of employment not subject to Canada Pension Plan contributions

\item {} 
\sphinxAtStartPar
Payments and benefits subject to Canada Pension Plan contributions

\item {} 
\sphinxAtStartPar
Payments and benefits not subject to Canada Pension Plan contributions

\item {} 
\sphinxAtStartPar
Calculate Canada Pension Plan contributions at an individual level

\end{itemize}

\sphinxAtStartPar
The \sphinxstylestrong{Canada Pension Plan} (CPP) is a federally legislated social insurance program established under the Canada Pension Plan
Act. Its primary purpose is to provide financial protection to contributors and their families in the event of retirement,
disability, or death. The program is funded through mandatory payroll deductions from employees, which are matched equally by
employers. These employee contributions are specifically referred to as Canada Pension Plan contributions.

\sphinxAtStartPar
In addition to the CPP, employers may offer private or non\sphinxhyphen{}government pension plans, which may also involve payroll
deductions from employees. The specific payroll withholding requirements for these supplementary pension plans will be discussed
in more detail in the later chapters; it is important to note that the CPP is often one of multiple retirement savings
vehicles available within an organization’s compensation structure.

\begin{sphinxadmonition}{note}{Note:}
\sphinxAtStartPar
Employers located in Quebec are responsible for deducting Québec Pension Plan (QPP) contributions, instead
of CPP contributions, from their Québec employees and remitting those contributions to Revenu Québec (RQ).
\end{sphinxadmonition}

\sphinxAtStartPar
The Canada Pension Plan (CPP) was designed as an income replacement program for individuals who have been in pensionable
employment during their working life. A CPP retirement pension is a monthly benefit paid to people who have contributed to
the Canada Pension Plan. The pension is designed to replace about 25 percent of the earnings on which a person’s contributions
were based. Individuals can apply for their CPP retirement pension when they turn 60.

\sphinxAtStartPar
There are three Canada Pension Plan benefits:
\begin{itemize}
\item {} 
\sphinxAtStartPar
retirement pension

\item {} 
\sphinxAtStartPar
disability benefits (for contributors with a disability and their dependent children)

\item {} 
\sphinxAtStartPar
survivor benefits (including the death benefit, the survivor’s pension and the children’s benefit)

\end{itemize}

\sphinxAtStartPar
The CPP operates throughout Canada while the province of Québec administers its own program for workers in Québec called the
\sphinxstylestrong{Québec Pension Plan} (QPP). The two plans work together to ensure that all contributors are protected, no matter where the
individual lives. Québec Pension Plan requirements will be covered later in this course.


\section{Who Must Contribute to the Canada Pension Plan}
\label{\detokenize{cpp-and-ei:who-must-contribute-to-the-canada-pension-plan}}
\sphinxAtStartPar
The CPP is a \sphinxstylestrong{contributory plan}. This means that all costs are covered by the financial contributions paid by employees,
employers and self\sphinxhyphen{}employed workers, and from revenue earned on CPP investments. The CPP is not funded through general tax
revenues.

\sphinxAtStartPar
Canada Pension Plan contributions must be withheld from employees who:
\begin{quote}
\begin{enumerate}
\sphinxsetlistlabels{\arabic}{enumi}{enumii}{}{.}%
\item {} 
\sphinxAtStartPar
CPP contributions must be withheld from employees who have reached the age of 18 but are under the age of 70.

\item {} 
\sphinxAtStartPar
CPP contributions must be withheld from employees who are in pensionable employment.

\item {} 
\sphinxAtStartPar
CPP contributions must be withheld from employees in pensionable employment who are not considered to be disabled by either Service Canada or Retraite Québec.

\end{enumerate}

\sphinxAtStartPar
4. CPP contributions must be withheld from employees who are 65 years of age but are under the age of 70 and are in receipt of the Canada or Québec Pension Plan retirement
pension, but have not filed an election to stop paying CPP contributions.
\end{quote}

\sphinxAtStartPar
In principle, employees who do not fall within the categories listed previously would not make CPP contributions. However, it
is not always clear what constitutes pensionable earnings and pensionable employment. To clarify eligibility, the CRA has
developed a list of the types of employment that are not subject to CPP contributions. This information can also be found in
the Employers’ Guide \sphinxhyphen{} Payroll Deductions and Remittances \sphinxhyphen{} T4001, which is published by the CRA.

\sphinxAtStartPar
The following types of employment are excluded by legislation and therefore do not constitute pensionable employment. Payments arising from such employment are not subject
to CPP contributions:
\begin{itemize}
\item {} 
\sphinxAtStartPar
employment in agriculture, or an agricultural enterprise, horticulture, fishing, hunting, trapping, forestry, logging, or lumbering, by an employer:
\begin{itemize}
\item {} 
\sphinxAtStartPar
who pays the employee less than \$250 in cash remuneration in a calendar year; or

\item {} 
\sphinxAtStartPar
employs the employee for a period of less than 25 working days in the same year on terms providing for payment of cash remuneration—the working days do not have to be consecutive

\end{itemize}

\item {} 
\sphinxAtStartPar
employment of a casual nature other than for the purpose of the employer’s usual trade or business

\item {} 
\sphinxAtStartPar
employment of a person, other than as an entertainer, in connection with a circus, fair, parade, carnival, exposition, exhibition, or other similar activity, if that person is:
\begin{itemize}
\item {} 
\sphinxAtStartPar
not regularly employed by that employer, and

\item {} 
\sphinxAtStartPar
employed by that employer for less than seven days in a year

\end{itemize}

\item {} 
\sphinxAtStartPar
employment of a person by a government body as an election worker, if that person:
\begin{itemize}
\item {} 
\sphinxAtStartPar
is not a regular employee of the government body, and

\item {} 
\sphinxAtStartPar
works for less than 35 hours in a calendar year

\end{itemize}

\item {} 
\sphinxAtStartPar
employment as a teacher on exchange from a foreign country

\item {} 
\sphinxAtStartPar
employment of a spouse or common\sphinxhyphen{}law partner if the employer cannot deduct the remuneration paid as an expense under the Income Tax Act

\item {} 
\sphinxAtStartPar
employment of a member of a religious order who has taken a vow of perpetual poverty. This applies whether the remuneration is paid directly to the order or paid by the member to the order.

\item {} 
\sphinxAtStartPar
employment for which no cash remuneration is paid, where the employee is the child of, or is maintained by, the employer

\item {} 
\sphinxAtStartPar
employment of a person who helps the employer in a disaster or in a rescue operation if the employee is not regularly employed by the employer

\end{itemize}


\chapter{Employment Insurance}
\label{\detokenize{cpp-and-ei:employment-insurance}}
\sphinxAtStartPar
Objective of this section is to enable you to identify the following Employment Insurance components:
\begin{itemize}
\item {} 
\sphinxAtStartPar
Who must pay Employment Insurance premiums

\item {} 
\sphinxAtStartPar
Types of employment subject to Employment Insurance premiums

\item {} 
\sphinxAtStartPar
Types of employment not subject to Employment Insurance premiums

\item {} 
\sphinxAtStartPar
Payments and benefits subject to Employment Insurance premiums

\item {} 
\sphinxAtStartPar
Payments and benefits not subject to Employment Insurance premiums

\item {} 
\sphinxAtStartPar
Calculate Employment Insurance premiums at an individual level

\end{itemize}

\sphinxAtStartPar
\sphinxstylestrong{Employment Insurance} (EI) is a federally legislated social insurance program established under the Employment Insurance Act.
It provides temporary financial support to individuals who are unemployed while seeking new employment or engaging in skill
development. In addition to regular benefits, EI offers special provisions for workers who take leave due to significant life
events such as illness, pregnancy, caring for a newborn or newly adopted child, supporting a critically ill or injured person,
or tending to a family member facing a serious health condition with a risk of death.

\sphinxAtStartPar
The EI program is funded through payroll contributions made by employees, known as Employment Insurance premiums. Employers
also contribute by paying a premium that is calculated based on their employees’ deductions.

\sphinxAtStartPar
While Employment Insurance is a government\sphinxhyphen{}mandated program, it may not be the only insurance plan available in the workplace.
Many organizations offer private or non\sphinxhyphen{}government insurance options such as life and disability coverage, which are funded by
employers, employees, or both. Although this chapter focuses specifically on the federally legislated EI program, additional
information about private insurance plans will be covered in the later chapters.


\chapter{Record of Employment}
\label{\detokenize{cpp-and-ei:record-of-employment}}
\sphinxAtStartPar
The \sphinxstylestrong{Record of Employment} (ROE) is the form used by Service Canada to determine an individual’s qualification to collect
Employment Insurance benefits when their employment is interrupted, how much the benefit will be and how long they will
collect it. As payroll is responsible for completing the ROE, the form will be illustrated in this chapter, along with an
explanation of what payroll information must be tracked for ROE reporting purposes.

\sphinxstepscope


\chapter{CALCULATING NET EARNINGS}
\label{\detokenize{compensation:calculating-net-earnings}}\label{\detokenize{compensation::doc}}

\section{Employment Income}
\label{\detokenize{compensation:employment-income}}

\section{Allowances}
\label{\detokenize{compensation:allowances}}

\section{Expenses}
\label{\detokenize{compensation:expenses}}

\section{Benefits}
\label{\detokenize{compensation:benefits}}
\sphinxstepscope


\chapter{OBNOARDING EMPLOYEE}
\label{\detokenize{onboarding_employee:obnoarding-employee}}\label{\detokenize{onboarding_employee::doc}}
\sphinxAtStartPar
In the context of Canadian payroll administration, onboarding an employee refers to the formal process of integrating a new hire into
both the organizational and payroll systems. It ensures that the employee is properly registered, legally compliant, and ready to be paid
accurately and on time.

\sphinxAtStartPar
Key Steps on Onboarding an Employee:
\begin{itemize}
\item {} 
\sphinxAtStartPar
Collect Required Personal Information: Includes full legal name, address, date of birth, and Social Insurance Number (SIN). The SIN is

\end{itemize}

\sphinxAtStartPar
critical for tax reporting to the CRA (Canada Revenue Agency).
\begin{itemize}
\item {} 
\sphinxAtStartPar
Obtain Federal \& Provincial Tax Forms: New employees must complete Form TD1 (Federal and possibly a Provincial version) to declare tax

\end{itemize}

\sphinxAtStartPar
credits and determine income tax withholdings.
\begin{itemize}
\item {} 
\sphinxAtStartPar
Set Up Banking Info for Direct Deposit: Employees usually provide a void cheque or bank form to set up electronic payments.

\item {} 
\sphinxAtStartPar
Register the Employee in the Payroll System: Involves entering all personal and job\sphinxhyphen{}related data, assigning a payroll ID, and verifying

\end{itemize}

\sphinxAtStartPar
employment status (e.g. full\sphinxhyphen{}time, part\sphinxhyphen{}time, contract).
\begin{itemize}
\item {} 
\sphinxAtStartPar
Enroll in Benefits or Pension Programs: If applicable, the employee may be signed up for group insurance, retirement savings plans (like

\end{itemize}

\sphinxAtStartPar
RRSP or pension plans), and other benefits. These deductions must be accurately reflected in payroll.
\begin{itemize}
\item {} 
\sphinxAtStartPar
Assign Statutory Deductions that Employers must withhold and remit
\begin{itemize}
\item {} 
\sphinxAtStartPar
CPP (Canada Pension Plan)

\item {} 
\sphinxAtStartPar
EI (Employment Insurance)

\item {} 
\sphinxAtStartPar
Income Tax (based on TD1 form)

\end{itemize}

\item {} 
\sphinxAtStartPar
Confirm Employment Agreement \& Start Date

\end{itemize}

\sphinxAtStartPar
Compliance \& Record Keeping
\begin{itemize}
\item {} 
\sphinxAtStartPar
Employers in Canada are responsible for keeping accurate records of employee data, pay stubs, deductions, and remittances for

\end{itemize}

\sphinxAtStartPar
at least 6 years.
\begin{itemize}
\item {} 
\sphinxAtStartPar
If audited by CRA, these documents must be readily available.

\item {} 
\sphinxAtStartPar
Employers must also provide T4 slips by end of February each year to summarize annual earnings and deductions for tax filing.

\end{itemize}
\begin{quote}

\sphinxAtStartPar
Employment Standards Requirements
\end{quote}


\bigskip\hrule\bigskip


\sphinxAtStartPar
Each province/territory, as well as the federal government, sets minimum employment standards, including:
\begin{itemize}
\item {} 
\sphinxAtStartPar
Minimum wage

\item {} 
\sphinxAtStartPar
Minimum age (may also be governed by other legislation)

\item {} 
\sphinxAtStartPar
Required pay statement information:
\sphinxhyphen{} Employee name
\sphinxhyphen{} Pay period date
\sphinxhyphen{} Rates of pay and hours worked
\sphinxhyphen{} Gross earnings
\sphinxhyphen{} Itemized deductions
\sphinxhyphen{} Net pay

\end{itemize}


\section{Internal Forms}
\label{\detokenize{onboarding_employee:internal-forms}}
\sphinxAtStartPar
Typical commencement package forms include:
\begin{itemize}
\item {} 
\sphinxAtStartPar
Authorization for hiring

\item {} 
\sphinxAtStartPar
Direct deposit agreement

\item {} 
\sphinxAtStartPar
Union membership application

\item {} 
\sphinxAtStartPar
Benefits enrollment (e.g., health/dental, pension)

\item {} 
\sphinxAtStartPar
Confidentiality agreement

\end{itemize}


\subsection{Authorization for Hiring}
\label{\detokenize{onboarding_employee:authorization-for-hiring}}
\sphinxAtStartPar
This internal document includes:
\begin{itemize}
\item {} 
\sphinxAtStartPar
New employee’s basic info

\item {} 
\sphinxAtStartPar
Start date, department, salary

\item {} 
\sphinxAtStartPar
Probation details

\item {} 
\sphinxAtStartPar
Hiring authority’s signature

\end{itemize}

\sphinxAtStartPar
\sphinxstylestrong{Important:} Employer must obtain a valid SIN. A SIN starting with 9 must have a valid expiry date and associated work permit.


\subsection{Union Membership}
\label{\detokenize{onboarding_employee:union-membership}}
\sphinxAtStartPar
For unionized workplaces:
\begin{itemize}
\item {} 
\sphinxAtStartPar
Amount of union dues to be deducted

\item {} 
\sphinxAtStartPar
Employees signature authorization for deduction

\item {} 
\sphinxAtStartPar
Exemptions may apply, but dues equivalent still required

\end{itemize}


\subsection{Benefit Enrollment Forms}
\label{\detokenize{onboarding_employee:benefit-enrollment-forms}}
\sphinxAtStartPar
Forms cover group insurance and pension plans:
\begin{itemize}
\item {} 
\sphinxAtStartPar
Employee indicates coverage type

\item {} 
\sphinxAtStartPar
Signatures authorize payroll deductions

\end{itemize}


\subsection{Confidentiality Agreement}
\label{\detokenize{onboarding_employee:confidentiality-agreement}}
\sphinxAtStartPar
A legally binding agreement protecting sensitive company info:
\begin{itemize}
\item {} 
\sphinxAtStartPar
Defines proprietary data

\item {} 
\sphinxAtStartPar
Outlines responsibilities, penalties, and timeframe

\end{itemize}


\section{Required Federal and Provincial/Territorial Forms}
\label{\detokenize{onboarding_employee:required-federal-and-provincial-territorial-forms}}
\sphinxAtStartPar
\sphinxstylestrong{Purpose:} Determine correct income tax withholdings.

\sphinxAtStartPar
Forms:
\begin{itemize}
\item {} 
\sphinxAtStartPar
TD1 (Federal)

\item {} 
\sphinxAtStartPar
TD1 (Provincial/Territorial)

\item {} 
\sphinxAtStartPar
TP\sphinxhyphen{}1015.3\sphinxhyphen{}V (Québec employees)

\end{itemize}

\sphinxAtStartPar
\sphinxstylestrong{Provincial/territorial withholding} is based on \sphinxstyleemphasis{province of employment}, but tax liability is based on \sphinxstyleemphasis{province of residence}.

\sphinxAtStartPar
\sphinxstylestrong{Adjustments:}
\begin{itemize}
\item {} 
\sphinxAtStartPar
Request extra withholding via TD1 or TP\sphinxhyphen{}1015.3\sphinxhyphen{}V

\item {} 
\sphinxAtStartPar
Request reduction using CRA Form T1213 or RQ Form TP\sphinxhyphen{}1016\sphinxhyphen{}V

\end{itemize}

\sphinxAtStartPar
Essential Info on All Forms:
\begin{itemize}
\item {} 
\sphinxAtStartPar
Employee name

\item {} 
\sphinxAtStartPar
Date of birth

\item {} 
\sphinxAtStartPar
Social Insurance Number

\end{itemize}


\subsection{Tax Credits (TD1)}
\label{\detokenize{onboarding_employee:tax-credits-td1}}\begin{enumerate}
\sphinxsetlistlabels{\arabic}{enumi}{enumii}{}{.}%
\item {} 
\sphinxAtStartPar
Basic personal amount

\item {} 
\sphinxAtStartPar
Canada caregiver (infirm children)

\item {} 
\sphinxAtStartPar
Age amount

\item {} 
\sphinxAtStartPar
Pension income

\item {} 
\sphinxAtStartPar
Tuition

\item {} 
\sphinxAtStartPar
Disability

\item {} 
\sphinxAtStartPar
Spouse/common\sphinxhyphen{}law partner amount

\item {} 
\sphinxAtStartPar
Eligible dependant

\item {} 
\sphinxAtStartPar
Caregiver for infirm spouse or dependant

\item {} 
\sphinxAtStartPar
Caregiver for dependant age 18+

\item {} 
\sphinxAtStartPar
Transfers from spouse

\item {} 
\sphinxAtStartPar
Transfers from dependant

\item {} 
\sphinxAtStartPar
Total

\end{enumerate}

\sphinxAtStartPar
Additional Instructions:
\begin{itemize}
\item {} 
\sphinxAtStartPar
Fill out TD1 only if claiming more than basic credit

\item {} 
\sphinxAtStartPar
Québec employees must always complete TP\sphinxhyphen{}1015.3\sphinxhyphen{}V

\end{itemize}


\subsection{Tax Credits (TP\sphinxhyphen{}1015.3\sphinxhyphen{}V \sphinxhyphen{} Québec)}
\label{\detokenize{onboarding_employee:tax-credits-tp-1015-3-v-quebec}}\begin{itemize}
\item {} 
\sphinxAtStartPar
Basic amount

\item {} 
\sphinxAtStartPar
Transfer from spouse

\item {} 
\sphinxAtStartPar
Amount for dependants

\item {} 
\sphinxAtStartPar
Impairment in mental/physical function

\item {} 
\sphinxAtStartPar
Age amount, retirement income, living alone

\item {} 
\sphinxAtStartPar
Career extension

\end{itemize}

\sphinxAtStartPar
Deductions:
\begin{itemize}
\item {} 
\sphinxAtStartPar
Remote area housing

\item {} 
\sphinxAtStartPar
Deductible support payments

\end{itemize}


\section{Entering Employee Information into Sage50}
\label{\detokenize{onboarding_employee:entering-employee-information-into-sage50}}
\sphinxAtStartPar
To enter a new employee into the Sage 50 Payroll module (Canada edition), start by navigating to the Employees \& Payroll section in the
Home window. Right\sphinxhyphen{}click the Employees icon and choose “Add Employee” to begin creating employee’s record. Input the employee’s full legal
name. Then, proceed to fill in the personal and payroll details across several tabs: the Personal tab for birth date and contact info, the Taxes tab to select the appropriate provincial
tax table, the Income tab to configure their pay frequency, and the Deductions tab to define benefit or pension deductions. You’ll also
want to enter their bank details for direct deposit. For compliance, be sure to complete and store signed TD1 forms (Federal and Provincial)
separately, as Sage50 does not automatically generate these. You’ll also need to set up EI, CPP, and Income Tax deductions and link them to remittance vendors in the system. Once all
information is reviewed for accuracy, save and close the record to finalize setup. If you prefer a guided approach, Sage50 also offers
an Employee Wizard to walk you through these steps.

\begin{sphinxadmonition}{important}{Important:}
\sphinxAtStartPar
To maintain accuracy and compliance in Sage 50 Payroll, carefully verify that all employee information entered into the system,
including full legal name, Social Insurance Number (SIN), residential address, and compensation details, matches the data provided on
official documentation such as the signed employment contract and government\sphinxhyphen{}issued identification (e.g. driver’s licence, Employment Contract).
Double\sphinxhyphen{}checking these entries helps prevent administrative errors and ensures that payroll records remain consistent with legal and
regulatory standards.
\end{sphinxadmonition}

\noindent\sphinxincludegraphics{{onboarding-employee_001}.png}

\noindent\sphinxincludegraphics{{onboarding-employee_002}.png}

\noindent\sphinxincludegraphics{{onboarding-employee_003}.png}

\noindent\sphinxincludegraphics{{onboarding-employee_004}.png}

\noindent\sphinxincludegraphics{{onboarding-employee_005}.png}

\noindent\sphinxincludegraphics{{onboarding-employee_006}.png}


\subsection{Review Questions}
\label{\detokenize{onboarding_employee:review-questions}}\begin{enumerate}
\sphinxsetlistlabels{\arabic}{enumi}{enumii}{}{.}%
\item {} 
\sphinxAtStartPar
What is the significance of accurately entering the “Date Hired” field when setting up a new employee profile in Sage 50?

\sphinxAtStartPar
\sphinxstyleemphasis{Accurately entering the “Date Hired” in Sage 50 is a critical step in ensuring the integrity of payroll records and overall HR
compliance. This field defines the employee’s official start date, which determines pay cycle alignment, benefit entitlement periods,
and the correct application of mandatory deductions such as CPP and EI. It also plays a pivotal role in historical payroll
reporting—including audit readiness and the generation of year\sphinxhyphen{}end T4 slips. Furthermore, the hire date is essential when preparing a
Record of Employment (ROE), as it establishes the starting point for the employee’s insurable earnings and service duration.}

\end{enumerate}

\sphinxAtStartPar
2. Within the scope of Payroll Administration, how should the department ethically and legally respond when a supervisor requests access
to an employee’s date of birth for the purpose of workplace recognition, given that this personal information is already held by payroll?
\begin{quote}

\sphinxAtStartPar
\sphinxstyleemphasis{Under Canadian payroll administration and the Personal Information Protection and Electronic Documents Act (PIPEDA), sharing an
employee’s date of birth — even for positive intentions like workplace recognition, is not legally or ethically appropriate.}

\sphinxAtStartPar
\sphinxstyleemphasis{PIPEDA requires employers to:}
\begin{itemize}
\item {} 
\sphinxAtStartPar
Limit the collection, use, and disclosure of personal information to what is necessary for clearly identified business purposes.

\item {} 
\sphinxAtStartPar
Obtain meaningful consent before using personal data for any purpose beyond what it was originally collected for—such as payroll or benefits administration.

\item {} 
\sphinxAtStartPar
Protect employee privacy by restricting access to personal information on a strict need\sphinxhyphen{}to\sphinxhyphen{}know basis.

\end{itemize}

\sphinxAtStartPar
\sphinxstyleemphasis{In this case, using the date of birth for celebrations or acknowledgments is outside the scope of payroll processing. Even if the
Payroll department holds this information, it cannot be disclosed to supervisors or other staff.}
\end{quote}


\section{Content Review Highlights}
\label{\detokenize{onboarding_employee:content-review-highlights}}\begin{itemize}
\item {} 
\sphinxAtStartPar
Consent is required for personal info collection

\item {} 
\sphinxAtStartPar
TD1 and TP\sphinxhyphen{}1015.3\sphinxhyphen{}V are used to calculate source deductions

\item {} 
\sphinxAtStartPar
Claim amounts may differ between federal and provincial forms

\item {} 
\sphinxAtStartPar
Employers must keep the forms on file (do not send to CRA/RQ)

\end{itemize}


\section{Review Questions (Sample)}
\label{\detokenize{onboarding_employee:review-questions-sample}}\begin{enumerate}
\sphinxsetlistlabels{\arabic}{enumi}{enumii}{}{.}%
\item {} 
\sphinxAtStartPar
What does an offer letter signature signify?

\item {} 
\sphinxAtStartPar
What documents are included in a commencement package?

\item {} 
\sphinxAtStartPar
Name three common internal forms

\item {} 
\sphinxAtStartPar
What must payroll verify on a hiring form?

\item {} 
\sphinxAtStartPar
What must be checked for SINs starting with “9”?

\item {} 
\sphinxAtStartPar
True/False: Union dues can be deducted without consent.

\item {} 
\sphinxAtStartPar
What authorizes benefit premium deductions?

\end{enumerate}


\section{Example Evaluations}
\label{\detokenize{onboarding_employee:example-evaluations}}
\sphinxAtStartPar
\sphinxstylestrong{Gloria Meyer (Alberta):}
\sphinxhyphen{} Claimed: Basic, eligible dependant, transferred tuition
\sphinxhyphen{} Appears accurate

\sphinxAtStartPar
\sphinxstylestrong{Luc Laframboise (Québec):}
\sphinxhyphen{} Claimed: Basic, spouse, dependant in school, tuition transfer
\sphinxhyphen{} Appropriate provincial and federal claims made

\sphinxAtStartPar
\sphinxstylestrong{Ingrid Johansson (Alberta, Single Parent):}
\sphinxhyphen{} Claimed credits for two children
\sphinxhyphen{} \sphinxstylestrong{Overclaimed} dependant credit \textendash{} only one is eligible
\sphinxhyphen{} Needs correction on federal and AB TD1 forms

\begin{sphinxadmonition}{note}{ONBOARDING EMPLOYEE EXERCISE}

\sphinxAtStartPar
Using MS Forms, create a questionaire for gathering all required information for onboarding a new employee at Quebec\sphinxhyphen{}based company for the payroll purposes.
\end{sphinxadmonition}

\sphinxstepscope


\chapter{Payroll Accounting}
\label{\detokenize{payroll_accounting:payroll-accounting}}\label{\detokenize{payroll_accounting::doc}}

\section{Journal Entries}
\label{\detokenize{payroll_accounting:journal-entries}}

\subsection{Accounting Recap}
\label{\detokenize{payroll_accounting:accounting-recap}}
\begin{center}\(\Sigma \text{ Total Debits} = \Sigma \text{ Total Credits}\)
\end{center}
\begin{center}\(\text{Assets} = \text{Liabilities} + \text{Equity}\)
\end{center}\begin{equation}\label{equation:payroll_accounting:AccountingEquation}
\begin{split}Assets = Liabilities + Equity\end{split}
\end{equation}
\sphinxAtStartPar
Furthermore, we know that:

\begin{center}\(\text{Equity = Revenue - Expenses}\)
, which leads us to:
\end{center}
\begin{center}\(\text{Assets = Liabilities + (Revenues - Expenses)}\)
\end{center}
\sphinxAtStartPar
Accounting equation \eqref{equation:payroll_accounting:AccountingEquation}

\sphinxAtStartPar
Payroll accounting is a critical component of the Canadian Payroll Administration system. It involves the systematic recording, analysis, and reporting of payroll transactions to ensure that all financial aspects of employee compensation are accurately reflected in the organization’s financial statements.
Payroll accounting includes the management of employee wages, tax withholdings, benefit deductions, and other payroll\sphinxhyphen{}related expenses. The system is designed to automate these processes, ensuring accuracy and compliance with Canadian payroll regulations.


\subsection{Journal Entries}
\label{\detokenize{payroll_accounting:id1}}
\sphinxAtStartPar
Journal entries are a key part of payroll accounting, as they document the financial impact of payroll transactions on the organization’s accounts. Each payroll run generates a series of journal entries that reflect the distribution of wages, taxes, and deductions across various accounts.
These entries are essential for maintaining accurate financial records and ensuring that the organization’s financial statements reflect the true cost of employee compensation. The Canadian Payroll Administration system automates the generation of these journal entries, reducing the risk of errors and ensuring compliance with accounting standards.
\begin{quote}

\sphinxAtStartPar
DR    Payroll Expenses    \$10,500.00

\sphinxAtStartPar
CR    Payroll Payable     (\$10,500.00)
\end{quote}


\begin{savenotes}\sphinxattablestart
\sphinxthistablewithglobalstyle
\centering
\sphinxcapstartof{table}
\sphinxthecaptionisattop
\sphinxcaption{Mar. 15, 2025}\label{\detokenize{payroll_accounting:id2}}
\sphinxaftertopcaption
\begin{tabular}[t]{\X{25}{60}\X{25}{60}\X{10}{60}}
\sphinxtoprule
\sphinxstyletheadfamily 
\sphinxAtStartPar
DR Account
&\sphinxstyletheadfamily 
\sphinxAtStartPar
CR Account
&\sphinxstyletheadfamily 
\sphinxAtStartPar
Amount
\\
\sphinxmidrule
\sphinxtableatstartofbodyhook
\sphinxAtStartPar
Payroll Expenses
&&
\sphinxAtStartPar
\$10,500.00
\\
\sphinxhline&
\sphinxAtStartPar
Payroll Payable
&
\sphinxAtStartPar
\$10,500.00
\\
\sphinxhline
\sphinxAtStartPar
\sphinxstyleemphasis{To record accrual of payroll expenses for the period.}
&&\\
\sphinxbottomrule
\end{tabular}
\sphinxtableafterendhook\par
\sphinxattableend\end{savenotes}

\sphinxstepscope


\chapter{REVIEW QUESTIONS}
\label{\detokenize{review_questions:review-questions}}\label{\detokenize{review_questions::doc}}
\sphinxAtStartPar
This section contains review questions for the material covered in the course. These questions are designed to test your understanding and help reinforce the concepts learned.


\section{New Employee Information}
\label{\detokenize{review_questions:new-employee-information}}
\sphinxAtStartPar
Which one of the following is correct?
\begin{enumerate}
\sphinxsetlistlabels{\arabic}{enumi}{enumii}{}{.}%
\item {} 
\sphinxAtStartPar
Choice A

\item {} 
\sphinxAtStartPar
\sphinxstylestrong{Choice B}

\item {} 
\sphinxAtStartPar
Choice C

\end{enumerate}

\sphinxstepscope


\chapter{TERMINOLOGY}
\label{\detokenize{terminology:terminology}}\label{\detokenize{terminology::doc}}

\section{Pensionable Earnings}
\label{\detokenize{terminology:pensionable-earnings}}

\section{Insurable Earnings}
\label{\detokenize{terminology:insurable-earnings}}
\sphinxstepscope


\chapter{RATES FOR 2025}
\label{\detokenize{rates_2025:rates-for-2025}}\label{\detokenize{rates_2025::doc}}

\section{CANADA / QUEBEC PENSION PLAN (CPP / QPP)}
\label{\detokenize{rates_2025:canada-quebec-pension-plan-cpp-qpp}}

\begin{savenotes}\sphinxattablestart
\sphinxthistablewithglobalstyle
\raggedright
\sphinxcapstartof{table}
\sphinxthecaptionisattop
\sphinxcaption{CANADA / QUEBEC PENSION PLAN (CPP / QPP)}\label{\detokenize{rates_2025:id1}}
\sphinxaftertopcaption
\begin{tabular}[t]{\X{130}{190}\X{30}{190}\X{30}{190}}
\sphinxtoprule
\sphinxstyletheadfamily 
\sphinxAtStartPar
Description
&\sphinxstyletheadfamily 
\sphinxAtStartPar
CPP
&\sphinxstyletheadfamily 
\sphinxAtStartPar
QPP
\\
\sphinxmidrule
\sphinxtableatstartofbodyhook
\sphinxAtStartPar
Yearly maximum pensionable earnings
&
\sphinxAtStartPar
\$71,300
&
\sphinxAtStartPar
\$
\\
\sphinxhline
\sphinxAtStartPar
Annual maximum contributory earnings
&
\sphinxAtStartPar
\$67,800
&
\sphinxAtStartPar
\$
\\
\sphinxhline
\sphinxAtStartPar
Annual maximum contribution
&
\sphinxAtStartPar
\$3,500
&
\sphinxAtStartPar
\$
\\
\sphinxhline
\sphinxAtStartPar
Employee contribution rate
&
\sphinxAtStartPar
5.95\%
&\\
\sphinxhline
\sphinxAtStartPar
Employer contribution rate
&
\sphinxAtStartPar
5.95\%
&\\
\sphinxhline
\sphinxAtStartPar
Basic exemption (Annual)
&
\sphinxAtStartPar
\$3,500
&\\
\sphinxhline\begin{quote}

\sphinxAtStartPar
Basic exemption (Monthly, 12)
\end{quote}
&
\sphinxAtStartPar
\$291.67
&
\sphinxAtStartPar
\$
\\
\sphinxhline\begin{quote}

\sphinxAtStartPar
Basic exemption (Weekly, 52)
\end{quote}
&
\sphinxAtStartPar
\$67.31
&
\sphinxAtStartPar
\$
\\
\sphinxhline\begin{quote}

\sphinxAtStartPar
Basic exemption (Weekly, 53)
\end{quote}
&
\sphinxAtStartPar
\$66.04
&
\sphinxAtStartPar
\$
\\
\sphinxhline\begin{quote}

\sphinxAtStartPar
Basic exemption (Semi\sphinxhyphen{}monthly, 24)
\end{quote}
&
\sphinxAtStartPar
\$145.83
&
\sphinxAtStartPar
\$
\\
\sphinxhline\begin{quote}

\sphinxAtStartPar
Basic exemption (Bi\sphinxhyphen{}weekly, 26)
\end{quote}
&
\sphinxAtStartPar
\$134.61
&
\sphinxAtStartPar
\$
\\
\sphinxbottomrule
\end{tabular}
\sphinxtableafterendhook\par
\sphinxattableend\end{savenotes}


\section{CPP2 CONTRIBUTION RATES MAXIMUMS}
\label{\detokenize{rates_2025:cpp2-contribution-rates-maximums}}

\begin{savenotes}\sphinxattablestart
\sphinxthistablewithglobalstyle
\raggedright
\sphinxcapstartof{table}
\sphinxthecaptionisattop
\sphinxcaption{CPP2 Contribution Rates Maximums}\label{\detokenize{rates_2025:id2}}
\sphinxaftertopcaption
\begin{tabular}[t]{\X{130}{160}\X{30}{160}}
\sphinxtoprule
\sphinxstyletheadfamily 
\sphinxAtStartPar
Description
&\sphinxstyletheadfamily 
\sphinxAtStartPar
Ammount
\\
\sphinxmidrule
\sphinxtableatstartofbodyhook
\sphinxAtStartPar
Additional maximum annual pensionable earnings
&
\sphinxAtStartPar
\$81,200
\\
\sphinxhline
\sphinxAtStartPar
Employee and employer contribution rate
&
\sphinxAtStartPar
4\%
\\
\sphinxhline
\sphinxAtStartPar
Maximum employee and employer contribution
&
\sphinxAtStartPar
\$396
\\
\sphinxhline
\sphinxAtStartPar
Maimum annual self\sphinxhyphen{}employed contribution
&
\sphinxAtStartPar
\$792
\\
\sphinxbottomrule
\end{tabular}
\sphinxtableafterendhook\par
\sphinxattableend\end{savenotes}

\sphinxstepscope


\chapter{REFERENCES}
\label{\detokenize{references:references}}\label{\detokenize{references::doc}}
\sphinxAtStartPar
\sphinxhref{https://www.canada.ca/en/revenue-agency/services/e-services/digital-services-businesses/payroll-deductions-online-calculator.html}{Payroll Deductions Online Calculator}

\sphinxAtStartPar
\sphinxhref{https://laws-lois.justice.gc.ca/eng/acts/C-8/page-5.html\#docCont}{CPP Maximum contributory earnings}

\sphinxAtStartPar
\sphinxhref{https://laws-lois.justice.gc.ca/eng/acts/C-8/page-3.html\#docCont}{Second additional CPP contributions}

\sphinxstepscope


\chapter{Errors and Errata}
\label{\detokenize{errata:errors-and-errata}}\label{\detokenize{errata::doc}}

\chapter{Glossary}
\label{\detokenize{index:glossary}}\begin{itemize}
\item {} 
\sphinxAtStartPar
\DUrole{xref}{\DUrole{std}{\DUrole{std-ref}{genindex}}}

\end{itemize}

\sphinxAtStartPar
\sphinxstylestrong{sys variables}

\begin{sphinxVerbatim}[commandchars=\\\{\}]
Python 3.12.3
\end{sphinxVerbatim}

\begin{sphinxVerbatim}[commandchars=\\\{\}]
5
\end{sphinxVerbatim}



\renewcommand{\indexname}{Index}
\printindex
\end{document}